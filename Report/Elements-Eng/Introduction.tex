\section{Introduction}
%%%%%%%%%%%%%%%%%%%%%%%%%%%%%%%%%%

Existing literature has well established that stocks comove with each other. One collection of studies suggests that commonality in fundamental values generates comovement in stock return. This suggestion is derived from economics without frictions and rational investors. For example, \cite{shiller1989comovements} provides evidence that the comovement of real dividends could be accounted for the comovement in real stock prices.
	Although, in recent years, it has been recognized that the comovement rises from non-fundamental sources. {\cite{barberis2003style} and \cite{barberis2005comovement}} provided theoretical models for predicting a comovement between fundamentally unrelated companies. Trying to explain factors affecting comovement, {\cite{AntonPolk}} examined the effect of common ownership\footnote{The common ownership concept has been observed in financial literature in recent years. There has been a surge in the popularity of index investing in the United States, which has led to an increase in common ownership. 
			For instance, \cite{azar2018anticompetitive} claims that an increase in mutual ownership in airline companies leads to less competitive ticket pricing. However, this subject is controversial, and many papers discuss whether mutual ownership affects companies' behavior.
			For example, \cite{lewellen2021does} realized that in previous investigations, other effective factors have wrongly been replaced by the mutual ownership effect.
		} on comovement\footnote{	The followings are some of the other sources of comovement. Index inclusion ({\cite{barberis2005comovement}}), investors' attention to the companies ({\cite{wu2014investor}}), Investment banks' underwriting ({\cite{grullon2014comovement}}), correlated beliefs ({\cite{david2016correlated}}), shareholders' coordination ({\cite{pantzalis2017shareholder}}), and preference for companies' dividends ({\cite{HAMEED2019103}}) are among contributing factors to comovement that have been identified by researchers.}. This paper uses mutual funds' ownership and suggests that comovement increases by increasing common ownership. Also, it is shown in the paper that the comovement increases when there is a significant net flow, either in or out-flow in the months.
	
	Subsequently, according to {\cite{Liquidity2016}} companies show comovement considering their owners' correlation in their liquidity needs. The author also adds that companies with higher mutual fund ownership have a more liquidity correlation than others. This paper contends that in order for companies to have comovement, there is no need for common ownership. Plus, common ownership can explain companies'  liquidity correlation. 
	
	While most of the prior investigations on factors affecting common-ownership have focused on the fund, the role of the block-holders as one the most important factors in firms' governance has remained a black box\footnote{A long literature surveyed by \cite{holderness2003survey}, \cite{edmans2014blockholders}, and \cite{edmans2017blockholders} considers the potential role of blockholders in firm governance}. This type of owner performs particular types of behavior due to their needs and the fact that they are intermediates. Nevertheless, in Iran, the block holders' daily ownership data, including mutual fund ownership, is publicly accessible. So research through this data can show whether common ownership other than mutual fund ownership can lead to comovement or not. Following \cite{AntonPolk}, we are the first study that uses block-holder ownership to investigate the relationship between common ownership and comovement.
		
Despite the presence of the business group in both emerging economies, e.g., Brazil, Chile, China, India, Indonesia, South Korea, and developed countries, e.g., Italy, Sweden, (\cite{khanna2007business}), there is no evidence on whether being at the same business group can lead to the comovement. Business groups consist of legally independent firms operating across diverse industries different from commonly held firms. Although researchers have identified comovement among stock
returns,  to the best of our knowledge, we are the first comprehensive study about the different roles of the business groups and common ownership on comovement. 


%%%%%%%%%%%%%%%%%%%%%%%%%%%%%%%%%%
%%%%%%%%%%%%%%%%%%%%%%%%%%%%%


%	
%	Both {\cite{cho2015stock} and \cite{kim2015stock}} studied the South Korean market and suggested two different sources for the comovement in business groups. The first paper attributed comovement to the companies' fundamentals. However, the second paper presents that the investors' category/habitat behavior is responsible for comovement.
%	
%	In this paper, we consider the comovement of the companies in business groups. Best of our knowledge, it is the first study that compares direct and indirect common ownership.
%	A modified measurement is introduced in this paper to calculate the common ownership of the companies. 


We hypothesize that stocks with a high level of common ownership and the same ultimate owner exhibit strong comovement. In fact, when we talk about the presence of two stocks in the same business group, we talk about a high level of invisible common ownership between two stocks that we cannot measure by common ownership measurements. 

%% Result
We test this hypothesis using \cite{AntonPolk} methodology and realize that common ownership is crucial for predicting the comovement. Business groups play a more critical role in predicting the correlation of companies' returns than common ownership, with a coefficient about six times as large as the common ownership coefficient. Furthermore, We show that common ownership can predict comovement only inside the business groups.



We extend our analysis to validate the prominence of business groups. First, we find that the average of common ownership is five times larger for the firms in the business group. So, we restrict the study to the high level of common ownership to distinguish the effect of common ownership and business groups. In this subsample, like the mentioned ones, business groups significantly impact firms' comovement. Second, if business groups affect comovement, there is no need to restrict our investigation to firms with common owners, and also this would affect all the firms in the market. To address these concerns and distinguish the impact of common ownership and business group, we extend our investigation to all the firms in the market and show that the business group can increase firms' comovement. 
	
	Finally, we show that correlated trade in business groups is the channel of comovement. We provide evidence that the volume and direction of trades in business groups are related, and firms in the business groups with higher relation in trade have a higher level of comovement.
	
	
	%  The business groups might not have as much influence as the market has on many companies' turnover fluctuations. However, it is influential.
	% The companies' presence in business groups can explain the correlation in turnover. 
	% We extracted both yearly companies' turnover average and monthly market turnover from monthly companies' turnover. Then we realized that there is less monthly residual dispersion in business
	% groups analyzing the deduced data.
	% More comovement is detected for groups with less dispersion.
	% 
	% Institutional imbalance:
	% Institutional imbalance dispersion must be low for these companies.
	% In general, the imbalance dispersion average in business groups is less than companies that are not included in the groups.
	% A more precise study illustrated that determining business groups with low dispersion, we expect the comovement to increase.
	% In low dispersion groups, the comovement increases with either a decrease in dispersion or with higher common ownership.
	% We searched for fundamental factors, too. But they had no impact on price comovements.-
	

%%%%%%%%%%%%%%%%%%%%%%%%%%%%%
%\section{Introduction}
در سال های اخیر با
\begin{itemize}
\item
هم حرکتی مورد توجه تحلیلگران بازار و محققان قرار گرفته است
\item
بعد از بحران مالی سال 2007 مدل های برآورد ریسک اهمیت پیدا کرده است
\item
در این مدل ها هم بستگی قیمت دارایی ها نقش تاثیرگذاری در برآورد ریسک دارد
\item
پاسخ سنتی به دلیل هم حرکتی بازده شرکت ها عوامل بنیادی دو شرکت بوده است  برای مثال 
 \lr{\cite{shiller1989comovements}}
 \item 
 ولی در سال های اخیر نشان داده شده است که هم حرکتی می تواند از عواملی غیر بنیادی به وجود بیاید. 
 \item
 مدل های تئوری برای  هم حرکتی میان بازده شرکت های غیر مرتبط از لحاظ بنیادی
 \lr{\cite{barberis2003style},\cite{barberis2005comovement}}
 \item
 معرفی عوامل دیگر برای هم حرکتی قیمت شرکت ها
 \begin{itemize}
 \item 
 عضو بودن شرکت ها در شاخص
  \lr{S\&P500} [\lr{\cite{barberis2005comovement}}]
      \item
      توجه سرمایه گذاران به شرکت ها
      [\lr{\cite{wu2014investor}}]  
        \item 
      پذیره نویسی توسط بانک سرمایه گذاری 
      (\lr{investment bank})
      
      [\lr{\cite{grullon2014comovement}}]  
      \item
      باور های یکسان و مرتبط
      [\lr{\cite{david2016correlated}}] 
         \item
         هم زمان بودن نیاز های نقدینگی سهامداران شرکت ها
         [\lr{\cite{pantzalis2017shareholder}}]  
            \item
         پرداخت سود تقسیمی توسط شرکت ها
         [\lr{\cite{HAMEED2019103}}]  
 \end{itemize}



\end{itemize}

\begin{itemize}
\item 
از طرف دیگر در سال های اخیر مسئله مالکیت مشترک ادبیات مالی مورد توجه قرار گرفته است
\footnote{
	با توجه به افزایش صندوق های سرمایه گذاری  دنبال کننده شاخص در آمریکا، مسئله مالکیت مشترک در میان شرکت های آمریکا افزایش داشته است و این امر سبب شده است که در ادبیات مسئله بررسی مالکیت مشترک و عملکرد شرکت ها و همچنین رفتار بازده ای شرکت ها مورد توجه قرار گیرد. 
	برای مثال 
	\lr{\cite{azar2018anticompetitive}}
	با افزایش مالکیت مشترک میان شرکت های هواپیمایی رقابت قیمتی شرکت ها کاهش پیدا می کند.  اما در این رابطه بحث و گفت و گو همچنان ادامه دارد و مقالات زیادی در رد و تایید اثر مالکیت مشترک بر روی رفتار شرکت ها وجود دارد. برای مثال مقاله
	\lr{\cite{lewellen2021does}}
	مقالات سال های گذشته را بررسی کرده است و یافته است که در بررسی های گذشته، اثر دیگر فاکتور های تاثیر گذار به اشتباه به مالکیت مشترک مرتبط شده است.
}
 و
\lr{\cite{AntonPolk}}
اثر مالکیت مشترک را بر هم حرکتی را بررسی کرده است.  

\begin{itemize}
\item \lr{\cite{AntonPolk}}
\begin{itemize}
\item
یافته است که با افزایش مالکیت مشترک هم حرکتی شرکت ها افزایش پیدا می کند. 
%\item
%علاوه بر این با توجه به دسترسی به داده های مالکیت صندوق های سرمایه گذاری مقاله نشان داده است که هم حرکتی شرکت ها هنگامی که جریان خروجی و ورودی قوی ای در صندوق ها وجود داشته باشد افزایش پیدا می کند. 
\item
این مقاله بررسی خود را محدود به صندوق های سرمایه گذاری فعال 
(\lr{Active mutual funds}) 
و شرکت های بزرگ (ارزش بازاری بالاتر از میانه ارزش شرکت ها) محدود کرده است.
\end{itemize}
\end{itemize}
\item
در ادامه 
\lr{\cite{Liquidity2016}}
نشان داده است که شرکت ها با توجه به هم بستگی نیاز های نقدینگی مالکان خود، با یکدیگر هم حرکتی نشان می دهند
\begin{itemize}
\item \lr{\cite{Liquidity2016}}:
\begin{itemize} 
\item
 نشان می دهد که شرکت های دارای سطح بالایی از مالکیت صندوق های سرمایه گذاری همراهی نقدشوندگی بالاتری نسبت به بقیه شرکت ها دارند.
 \item
 نشان می دهد که برای هم حرکتی قیمت شرکت ها نیاز به مالکیت مشترک نیست
 \item
 همچنین نشان می دهد که مالکیت مشترک می تواند هم بستگی در نقد شوندگی سهام شرکت را توضیح دهد
\end{itemize}

\item
نتایج به دست آمده در پژوهش های قبلی با استفاده از داده های مالکیت صندوق های سرمایه گذاری بدست آمده است
\item
در نتیجه نتایج بدست آمده محدود به این نوع مالکیت می باشد
\item
در صورتی که این نوع به خصوص مالکیت با توجه به نیاز ها و واسطه بودن، رفتار های به خصوصی انجام می دهند
\footnote{\lr{\cite{coval2007asset}} 
	نشان داده است که جریان ورود و خروج مالی صندوق ها می تواند سبب ایجاد فشار قیمتی بر سهام شرکت ها شود و قیمت شرکت ها رو تحت تاثیر قرار دهد و این مسئله با موصوع مالکیت مشترک که می تواند سبب تغییر رفتار مدیران شرکت شود می تواند متفاوت باشد.}

\end{itemize}

\item 
از طرفی دیگر 
یکی دیگر از ویژگی های بازار سرمایه ایران وجود گروه های کسب و کار است. گروه های کسب و کار حدود 85\% از ارزش بازار ایران را در اختیار دارند.
\begin{itemize}
	\item
	گروه های کسب و کار پدیده مهمی هستند در کشور های در حال توسعه یافته و در حال توسعه وجود دارند.
	\item
	هم حرکتی شرکت ها را در گروه های کسب و کار بررسی می کنیم
	\item
	دو مقاله در ادبیات به این موضوع پرداخته اند و هم حرکتی قیمت شرکت ها در گروه های کسب و کار را بررسی کرده اند
	\item
	هم حرکتی قیمت شرکت ها در گروه های کسب و کار تایید شده است ولی کانال این هم حرکتی مشخص نشده است
	\begin{itemize}
		\item \lr{\cite{cho2015stock},\cite{kim2015stock}}:
		\begin{itemize}
			دو پاسخ متفاوت به دلایل هم حرکتی شرکت ها در گروه های کسب و کار داده اند. هر دو گروه های کسب و کار موجود در بازار کره جنوبی را بررسی کرده اند و مقاله اول عوامل بنیادی مرتبط شرکت ها در گروه های کسب و کار را به عنوان دلیل هم حرکتی شرکت ها معرفی کرده است ولی مقاله دوم دسته بندی شرکت های عضو گروه را به عنوان دلیل هم حرکتی بیان کرده است. 
		\end{itemize}
		
	\end{itemize}

\end{itemize}
\item
در این بستر می توان اثر مالکیت مشترک مستقیم و غیر مستقیم را بررسی کرد
\end{itemize}

%%%% Our work
\begin{itemize}
\item 
در این پژوهش 
\item
اولین بررسی هم حرکتی ناشی از مالکیت مشترک صرف نظر از اینکه مالک شرکت صندوق سرمایه گذاری بوده باشد

\item 
در ایران داده های مالکیت های بالای یک درصد به صورت روزانه وجود دارد که محدود به مالکیت صندوق های سرمایه گذاری نیست.
\item
می توان به این سوال پاسخ داد که مالکیت مشترک فارغ از نوع مالکان سبب هم حرکتی می شود و یا خیر
\end{itemize}

%با توجه به محدودیت های دیتای موجود در آمریکا و  تنها موجود بودن داده های مالکیت های صندوق های سرمایه گذاری، بخشی از بررسی های این حوزه محدود به اثر مالکیت صندوق های سرمایه گذاری بر شرکت ها می باشد. برای مثال مقاله 
%\lr{\cite{coval2007asset}} 
%نشان داده است که جریان ورود و خروج مالی صندوق ها می تواند سبب ایجاد فشار قیمتی بر سهام شرکت ها شود و قیمت شرکت ها رو تحت تاثیر قرار دهد و این مسئله با موصوع مالکیت مشترک که می تواند سبب تغییر رفتار مدیران شرکت شود می تواند متفاوت باشد.


%در ایران با توجه به دسترسی به داده های مالکیت های بالای یک درصد شرکت ها به صورت روزانه، می توان با دقتی بالاتر و مشخص تر اثر مالکیت مشترک را بر روی شرکت های ثبت شده در بازار بورس مشخص کرد.  از طرف دیگر یکی دیگر از ویژگی های بازار سرمایه ایران وجود گروه های کسب و کار است. گروه های کسب و کار حدود 85\% از ارزش بازار ایران را در اختیار دارند.  در بازار های در حال توسعه در اکثر نقاط دنیا و حتی بعضی از کشرو های توسعه یافته نیز گروه های کسب و کار وجود دارند. گروه های کسب و کار مجموعه از شرکت های مرتبط با یکدیگر هستند که دارای مالک نهایی یکسان می باشند. در ادبیات موارد متعددی در رابطه با تاثیر گروه های کسب و کار بر روی شرکت ها معرفی  شده است. در رابطه با بررسی هم حرکتی در شرکت ها دو مقاله 
%\lr{\cite{cho2015stock},\cite{kim2015stock}}
%این موضوع را بررسی کرده اند و دو پاسخ متفاوت به دلایل هم حرکتی شرکت ها در گروه های کسب و کار داده اندد. هر دو گروه های کسب و کار موجود در بازار کره جنوبی را بررسی کرده اند و مقاله اول مسائل بنیادی مرتبط شرکت ها در گروه های کسب و کار را به عنوان دلیل هم حرکتی شرکت ها معرفی کرده است ولی مقاله دوم دسته بندی شرکت های عضو گروه را به عنوان دلیل هم حرکتی بیان کرده است.


\begin{itemize}
\item
در این مقاله سعی شده است تا هم حرکتی میان شرکت های درون گروه های کسب و کار بررسی شود.

\item 
و برای اولین میان مالکیت مشترک مستفیم و مالکیت مشترک غیر مستقیم مقایسه انجام شود

\end{itemize}
 
 
 \begin{itemize}
 \item
برای محاسبه مالکیت مشترک میان شرکت ها از ملاکی اصلاح شده استفاده کرده ایم که در این پژوهش معرفی کرده ایم

 \end{itemize}


\begin{itemize}
\item
مالکیت مشترک برای پیش بینی هم حرکتی قیمت شرکت ها اهمیت دارد
\item 
گروه های کسب و کار برای پیش بینی هم حرکتی قیمت شرکت ها اهمیت دارد
\item
اهمیت گروه های کسب و کار برای پیش بینی هم حرکتی قیمت شرکت ها بیشتر از مالکیت مشترک است
\item
 در گروه های کسب و کار، مالکیت مشترک سبب افزایش هم حرکتی قیمت می شود ولی در خارج از گروه های کسب و کار اهمیت ندارد
\end{itemize}



\begin{itemize}
\item 
انواع بررسی ها برای تایید اهمیت گروه های کسب و کار انجام شده است 
\begin{itemize}

\item
فقط جفت  های د ارای مالکیت بالا را بررسی کردیم
\begin{itemize}
\item 
در این زیر مجموعه هم گروه های کسب و کار بیشترین تاثیر را دارند 
\item
مالکیت مشترک صرفا در گروه های کسب و کار اهمیت دارند.

\end{itemize}
\item
بررسی ها محدود به جفت های دارای مالک مشترک بوده است:
\begin{itemize}
\item
بررسی اثر گروه کسب و کار نیاز به مالکیت مشترک ندارد
\item
اثر گروه کسب و کار و مالکیت مشترک را نمی توان جدا کرد
\item 
همه ی جفت های بازار را ساختیم و اهمیت گروه ها کسب و کار نسبت به مالکیت مشترک مستقیم تایید شده است.
%\begin{itemize}
%\item
%نتایج اولیه تایید شد
%\item 
%جفت های حاضر در گروه های کسب و کار سطح مالکیت مشترک اهمیت ندارد و صرفا سطح بالایی از مالکیت مشترک اهمیت دارد
%\item
%برای جفت های بیرون یک گروه کسب وکار، سطح مالکیت در واقع وجود مالکیت مشترک اهمیت دارد و نه مقدار قابل توجه آن
%\item
%مالکیت مشترک خارج از گروه های کسب و کار نیز اهمیت دارد
%\item
%تاثیر یکسان بودن گروه های کسب و کار بیشتر است
%\end{itemize}
\end{itemize}

\end{itemize}


	\item
کانال تاثیر: معامله هم زمان شرکت ها با یکدیگر در گروه های کسب و کار
\begin{itemize}

	\item
	از دو پراکسی برای بررسی اثر معاملات هم زمان در گروه های کسب و کار 
	\item
	\lr{turnover} 
	و ناترازی خرید و فروش حقوقی استفاده کرده ایم
	\item
	پراکسی اول معاملات هم زمان را تایید می کند
	\item
	پراکسی دوم هم جهتی معاملات را تایید می کند
	%\begin{itemize}
	%\item
	%\lr{turnover}
	%\begin{itemize}
	%\item
	%بخش قابل توجهی از تغییرات 
	%\lr{turnover}
	%شرکت ها علاوه بر بازار از گروه های کسب و کار ناشی می شود
	%\item
	%حضور شرکت ها در گروه های کسی و کار می تواند هم بستگی 
	%\lr{turnover}
	%را توضیح دهد.
	%\item
	%از 
	%\lr{turnover}
	%ماهانه شرکت ها 
	%میانگین 
	%\lr{turnover}
	%سالانه شرکت و 
	%\lr{turnover}
	%ماهانه بازار را خارج کردیم و بررسی کردیم درگروه های کسب و کار میزان پراکندگی باقی مانده ماهانه کمتر است
	%\item
	%برای گروه های با پراکندگی کمتر هم حرکتی بیشتر است
	%
	%% گروه های کسب و کار بزرگ
	%% \begin{itemize}
		%% 	\item 
		%% 	اگر معامله گران شرکت های در یک گروه کسب و کار را در یک دسته قرار می دهند نیاز است تا اعضای گروه های بزرگ هم حرکتی بیشتری داشته باشند
		%% 	\item
		%% 	علاوه بر مورد فوق باید رابطه هم بستگی 
		%% 	\lr{turnover}
		%% 	و هم حرکتی بازده نیز مثبت باشد. 
		%% 	
		%% 		\item 
		%% 		شرکت های عضو گروه کسب و کار به همراه یکدیگر معامله بشوند
		%% 		
		%% 	
		%% \item
		%% بررسی کردیم و نتایج نشان داد که شرکت های در گروه بزرگ هم حرکتی بیشتری دارند و علاوه بر این تاثیر هم حرکتی در 
		%% \lr{turnover}
		%% نیز در گروه های بزرگ از دیگر گروه ها بیشتر است.
		%% \end{itemize}
	%\end{itemize}
	%\item 
	%ناترازی خرید حقوقی
	%\begin{itemize}
	\item
	پراکندگی ناترازی خرید و فروش حقوقی در این شرکت ها باید کم باشد
	\item 
	به صورت کلی در گروه های کسب و کار میانگین پراکندگی شاخص ناترازی کمتر از شرکت های بیرون گروه است
	\item
	بررسی دقیق تر نشان داد با مشخص کردن گروه های کسب و کار دارای پراکندگی کم انتظار داریم با کاهش پراکندگی، هم حرکتی افزایش پیدا کند
	\item
	در گروه های با پراکندگی کم، هم حرکتی شرکت ها افزایش پیدا می کند و با افزایش مالکیت مشترک نیز هم حرکتی افزایش پیدا می کند
	
	\item
	کانال عوامل بنیادی را بررسی کردیم ولی تاثیری بر هم حرکتی قیمت ها یافت نشد
	
\end{itemize}



\end{itemize}






