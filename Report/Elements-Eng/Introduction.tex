\section{Introduction}
%%%%%%%%%%%%%%%%%%%%%%%%%%%%%%%%%%

It is well established that stock prices comove with each other. Earlier studies explain stock return comovements through commonality in fundamentals. For example, \cite{shiller1989comovements} provides evidence that comovement of dividends could account for stock price comovement. More recently,\change{It is 2022, 2003 is not recent anymore} it has been recognized that return comovement could rise between fundamentally unrelated stocks (see for example, {\cite{barberis2003style} and \cite{barberis2005comovement}} for theoretical models predicting comovement between fundamentally unrelated companies.) The rise in popularity of index and passive investing around the world in the last two decades has led to an increase in a phenomnon known as common ownership, which has attracted the attention of academics as well as policy makers. For example, \cite{azar2018anticompetitive} show that an increase in mutual ownership in airline companies leads to less competitive ticket pricing.  While the anticompetitive effect of common ownership has sparred somewhat of a controversial debate, {\cite{AntonPolk}} provide less disputable evidence showing that having common shareholders drive stock return comovement\footnote{
	The followings are some of the other explanations for return comovement: index inclusion ({\cite{barberis2005comovement}}), investors' attention to the companies ({\cite{wu2014investor}}), Investment banks' underwriting ({\cite{grullon2014comovement}}), correlated beliefs ({\cite{david2016correlated}}), shareholders' coordination ({\cite{pantzalis2017shareholder}}), and preference for companies' dividends ({\cite{HAMEED2019103}}) are among contributing factors to comovement that have been identified by researchers.
}.\improvement{Is not better to talk about co-movement literature in the main body and the common ownership literature in footnote?}
\improvement[inline]{Connect common ownership literature and stocks comovement}
 They use mutual funds' ownership and suggest that comovement increases by increasing common ownership. Also, they show that the comovement increases when there is a significant net flow, either in or out-flow in the months for mutual funds.
Subsequently, according to {\cite{Liquidity2016}} companies show comovement considering their owners' correlation in their liquidity needs. The authors also add that companies with higher mutual fund ownership have a more liquidity correlation than others. This paper contends that firms without a common owner would comove with each other. 
	
While most of the prior investigations on factors affecting common-ownership have focused on the fund, the role of the block-holders as one of the most important factors in firms' governance has remained a black box\footnote{A long literature surveyed by \cite{holderness2003survey}, \cite{edmans2014blockholders}, and \cite{edmans2017blockholders} considers the potential role of blockholders in firm governance}. 
The fact that funds are intermediates and behave differently due to their needs, making it difficult to generalize these results to other types of ownership. \info{We should talk more about their differences and refer to a paper} \cite{edmans2014governance} provide a theoretical model to investigate the implications of common ownership of block holders for corporate governance and asset pricing. \improvement{Talk more about paper} Regardless, the block holders' daily ownership data, including mutual fund ownership, is publicly accessible in Iran.\info{Talk more about our unique data} So research through this data can show whether common ownership other than mutual fund ownership can lead to comovement or not. Following\unsure{What do you mean by “Following”?
	Explain how your study is different or complement it } \cite{AntonPolk}, we are the first study that uses block-holder ownership to investigate the relationship between common ownership and comovement. \info[inline]{I realized that at previous paragraph we talk about our main contribution, are we clearly explain our contributions?} 
\improvement[inline]{Talk about our institutional setting and then explain about BG. We jump to the next paragraph.}
\improvement[inline]{I think that we should pin-point  our contributions here and sell it}
\improvement[inline]{What is the difference between motivation and contribution?! }
	
	

Despite the presence of the business group in both emerging economies, e.g., Brazil, Chile, China, India, Indonesia, South Korea, and developed countries, e.g., Italy, Sweden, (\cite{khanna2007business}), there is no evidence on whether being at the same business group can lead to the comovement. \unsure{Why is this important? Talk more about the importance of BG and also this question} Business groups consist of legally independent firms operating across diverse industries different from commonly held firms. Although researchers have identified comovement among stock
returns,  to the best of our knowledge, we are the first comprehensive study about the different roles of the business groups and common ownership on comovement. 
\info[inline]{There are other weak papers that answered this question. It is not better to talk about them at least in the footnote? }
\info[inline]{There is a paper that talks about asset pricing effects of BG}


\info[inline]{The reader is puzzled: Block-holders; Business
	Group; Ultimate Owner}
\info[inline]{OK, we have two types of common ownership:
	visible and invisible
	But earlier you compare “intermediaries” and block-holders}

%%%%%%%%%%%%%%%%%%%%%%%%%%%%%%%%%%
%%%%%%%%%%%%%%%%%%%%%%%%%%%%%


%	
%	Both {\cite{cho2015stock} and \cite{kim2015stock}} studied the South Korean market and suggested two different sources for the comovement in business groups. The first paper attributed comovement to the companies' fundamentals. However, the second paper presents that the investors' category/habitat behavior is responsible for comovement.
%	
%	In this paper, we consider the comovement of the companies in business groups. Best of our knowledge, it is the first study that compares direct and indirect common ownership.
%	A modified measurement is introduced in this paper to calculate the common ownership of the companies. 


We hypothesize that stocks with a high level\unsure{High level is right?} of common ownership and the same ultimate owner exhibit strong comovement. In fact, when we talk about the presence of two stocks in the same business group, we talk about a high level of invisible \unsure{Is it really invisible?} common ownership between two stocks that we cannot measure by common ownership measurements. 
\unsure[inline]{Should not we talk about the channel?}
\unsure[inline]{Should not we talk about the identification challenges?}

%% Result
\reversemarginpar
We test this hypothesis using \cite{AntonPolk} methodology \info{ We do not have their identification method } and realize that common ownership is crucial for predicting the comovement. Business groups play a more critical role in predicting the correlation of companies' returns than common ownership, with a coefficient about seven times as large as the common ownership coefficient.
\unsure[inline]{Is this correct? We talked about this before. We believe that we cannot compare these two coefs.} Furthermore, We show that common ownership can predict comovement only inside the business groups.\unsure[inline]{Can we say this? It's not consistent with our second sentence in paragraph}


We extend our analysis to validate the prominence of business groups. First, we find that the average of common ownership is five times larger for the firms in the business group. So, we restrict the study to the high level of common ownership to distinguish the effect of common ownership\change{High level of common ownership} and business groups. In this subsample, like the mentioned ones, business groups significantly impact firms' comovement. Second, if business groups affect comovement, there is no need to restrict our investigation to firms with common owners, and also this would affect all the firms in the market. To address these concerns and distinguish the impact of common ownership and business group, we extend our investigation to all the firms in the market and show that the business group can increase firms' comovement for all the firms in the market. 
	
{Finally, we investigate different \unsure{Is it right? We just propose a channel and check it. There was some investigation but we did not include them in the paper} sources of business groups' comovement. We show that correlated trade in business groups is the channel of comovement. We provide evidence that the volume and direction of trades in business groups are related, and firms in the business groups with higher relation in trade have a higher level of comovement.}
	\improvement[inline]{Tell about sections}
	
	
	
	
	%  The business groups might not have as much influence as the market has on many companies' turnover fluctuations. However, it is influential.
	% The companies' presence in business groups can explain the correlation in turnover. 
	% We extracted both yearly companies' turnover average and monthly market turnover from monthly companies' turnover. Then we realized that there is less monthly residual dispersion in business
	% groups analyzing the deduced data.
	% More comovement is detected for groups with less dispersion.
	% 
	% Institutional imbalance:
	% Institutional imbalance dispersion must be low for these companies.
	% In general, the imbalance dispersion average in business groups is less than companies that are not included in the groups.
	% A more precise study illustrated that determining business groups with low dispersion, we expect the comovement to increase.
	% In low dispersion groups, the comovement increases with either a decrease in dispersion or with higher common ownership.
	% We searched for fundamental factors, too. But they had no impact on price comovements.-
