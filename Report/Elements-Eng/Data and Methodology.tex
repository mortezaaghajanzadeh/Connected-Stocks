% !TeX spellcheck = <none>

\section{{Data and Methodology}}



\subsection{{Data and Sample}}



Our data have several unique features which distinguish our paper from the existing literature. Stock ownership data for Iranian public firms is available with daily frequency reported at the end of each trading day. This data includes all blockholders, defined as having at least one percent of the outstanding shares in a firm for all investor types- institutional as well as individuals- and is automatically reported by a central authority named Tehran Securities Exchange Technology Management Corporation (TSETMC), which is a subsidiary of Tehran Stock Exchange. This eliminates potential issues in self-reported data (such as those existing in US institutional holding data, known as 13F filings). In order to put together our dataset, we compile daily stock ownership tables (available in separate tables for each firm), which are publicly available starting from 2010.

We use data on business groups, defined as a group of listed firms with interconnected ownership structures controlled by an ultimate owner. Business groups are a common organizational structure in corporate ownership in Iran as well as many other parts of the world. Two-thirds of Iranian public firms are part of complex interlinked ownership networks, each governed by an ultimate owner, which sits at the top of a multi-layer pyramid ownership structure ({\cite{Aliabadi2022}}). Unlike countries like South Korea, Japan, and India that formally announce business groups, we do not have officially-defined business groups in Iran. We use data provided by {\cite{Aliabadi2022}} which builds a comprehensive dataset of all Iranian business groups. Using either of the two different methodologies, one introduced by \cite{almeida2011structure} with a 40\% threshold for control rights, and the other by \cite{aminadav2011rebuilding} which is based on Shapley-Shubik index (\cite{shapley1954method}) generates the same business group definitions in Iranian public sector({\cite{Aliabadi2022}}).\footnote{For further discussion see appendix \ref{BGDef}} \info{I think that we should talk more about the creation of BG. There is a great concern that they are endogenous.} Our business group data covers the period 2015-2020.  



	We also gather stock returns, trading volume, firm-level trading data, and accounting information from Codal (equivalent to SEC's EDGAR) \footnote{\href{http://www.codal.ir}{www.codal.ir}}
and TSETMC's website\footnote{\href{http://www.tsetmc.com}{www.tsetmc.com}}. We exclude ETFs. The final sample used in our empirical analysis spans from 2015 to 2020 (1393/01-1398/12 Persian calendar). 



Panel \subref{t2-1} table \ref{st1} reports summary statistics for ownership and business groups data. An average firm in our sample has six blockholders holding 75 percent of shares in aggregate. More than two-thirds of our sample firms are part of a business group. There are around 40 business groups, each consisting of, on average, seven firms.






\subsection{{Pair composition} }

	If any two firms have at least one common blockholder in a month, we consider them a commonly held firm pair. By this definition, there are 17522  unique pairs in our entire sample period, which is 20\%of all possible pairs ($\frac{554*553}{2} = 153181$). Firm pairs in our sample have on average 1.2common owners. 
	
	An important feature of common ownership in the Iranian public sector is that business groups are the main driver of common ownership, in contrast to US data in which the rise in common ownership is generally attributed to passively managed funds such as index funds. The asset management industry, especially in the form of mutual funds or ETFs, is generally a young but growing industry in Iran and does not have a large presence in firms' ownership structures. We, therefore, also identify pairs based on whether the two firms belong to the same business groups. 
		
		If both firms in a pair belong to the same ultimate owner, we identify them as being in the same business group. The two firms in a pair could also belong to different business groups or not be part of any business groups. Figure \ref{g2-1} illustrates all possibilities based on whether firm pairs belong to the same, different, or any business groups. In about one-third of our pairs, neither of the two firms belong to any business group, while in about 11 percent of our pairs, firms are part of the same business group. Panel \subref{t2-2} of table \ref{st1} reports summary statistics for firm pairs.
	
	\captionsetup[subfigure]{labelformat=parens,font=footnotesize}
			\renewcommand{\thesubfigure}{\Alph{subfigure}}	
	\begin{figure}[htbp]
		\centering
		\caption{ Firm pairs and business groups}
		\label{g2-1}
		\normalcolor
	\bigskip
		\begin{subfigure}[t]{.6\linewidth}
			\centering
			\tiny
			\resizebox{1\textwidth}{!}{
				
\begin{tikzpicture}[node distance=2cm]


\node (start) [startstop] {\normalsize Ultimate Owner};



\node (end) [startstop1,below of = start , yshift=0cm , xshift=-3.5cm ] {\normalsize $ \text{Firm A} $};
\node (end2) [startstop1,below of = start , yshift=0cm , xshift=3.5cm ] {\normalsize $ \text{Firm B} $};



\node (sur) [startstop2 ,below of = end ,yshift=0cm,xshift=0cm] {\normalsize $ \text{Firm X} $};


\node (sur2) [startstop2 ,below of = end2 ,yshift=0cm,xshift=0cm] {\normalsize $ \text{Firm Y} $};


\node (CH) [process, below of = start ,xshift=0] {\normalsize Common Owner};


\draw [arrow] (start) --(end);
\draw [arrow] (end) --(sur);

\draw [arrow] (start) --(end2);

\draw [arrow] (end) --(sur);
\draw [arrow] (end2) -- (sur2);


\draw [arrow] (CH) -- (sur);
\draw [arrow] (CH) -- (sur2);
\draw [dashed ,->] (start) --(CH);

\draw [latex'-latex'] (sur) to [bend right =0]  node[sloped, anchor=center, below] {} (sur2);


\end{tikzpicture}

			}
			\caption{ Pair in the same business group}
		\end{subfigure}

		\begin{subfigure}[t]{.65\linewidth}
			\centering
			\tiny
			\resizebox{1\textwidth}{!}{
						\begin{tikzpicture}[node distance=2cm]
					
					
					\node (start) [startstop] { \normalsize$ \text{Ultimate Owner A} $};
					\node (start2) [startstop,right of = start,xshift=5cm] {\normalsize$ \text{Ultimate Owner B} $};
					
					
					\node (CH) [process, below of = start2,xshift=-3.5cm] {\normalsize Common Owner};
					
					\node (end) [startstop1,below of = start ] {\normalsize $ \text{Firm A} $};
					
					\node (end2) [startstop1,below of = start2 ,yshift=0cm,xshift=0cm] {\normalsize $ \text{Firm B} $};
					
					\node (sur) [startstop2 ,below of = end ,yshift=0cm,xshift=0cm] {\normalsize $ \text{Firm X} $};
					
					\node (sur2) [startstop2,below of = end2 ,yshift=0cm,xshift=0cm] {\normalsize $ \text{Firm Y} $};
					
					
					
					\draw [arrow] (start) --(end);
					\draw [arrow] (start2) -- (end2);
					
					\draw [arrow] (end) --(sur);
					\draw [arrow] (end2) -- (sur2);
					
					\draw [dash dot,->] (start) -- (CH);
					\draw [dash dot,->] (start2) -- (CH);
					
					\draw [arrow] (CH) -- (sur);
					\draw [arrow] (CH) -- (sur2);
					
					\draw [latex'-latex'] (sur) to [bend right =0]  node[sloped, anchor=center, below] {} (sur2);
					
					
				\end{tikzpicture}
		
			}   
			\caption{ Pair in two distinct business group}
		\end{subfigure}
			\bigskip
		\begin{subfigure}[t]{1\linewidth}
			
			\resizebox{0.49\textwidth}{!}{
			\begin{tikzpicture}[node distance=2cm]
					
					
					
					\node (CH) [process,yshift = -2cm ,xshift=3.5cm] {Common Owner};
					
					\node (end) [startstop1,left of = CH ,xshift=-1.5cm ] {$ \text{Firm A} $};
					
					\node (end2) [startstop1,right of = CH ,yshift=0cm,xshift=1.5cm] {$ \text{Firm B} $};
					
					\node (sur) [startstop2 ,below of = end ,yshift=0cm,xshift=0cm] {$ \text{Firm X} $};
					
					\node (sur2) [startstop2,below of = end2 ,yshift=0cm,xshift=0cm] {$ \text{Firm Y} $};
					
					
					\draw [arrow] (end) --(sur);
					\draw [arrow] (end2) -- (sur2);
					
					
					\draw [arrow] (CH) -- (sur);
					\draw [arrow] (CH) -- (sur2);
					
					\draw [latex'-latex'] (sur) to [bend right =0]  node[sloped, anchor=center, below] {} (sur2);
					
					
				\end{tikzpicture}
		
			}   
			\hfill
			\resizebox{0.49\textwidth}{!}{
						\begin{tikzpicture}[node distance=2cm]
					
					
					\node (start) [startstop] { $ \text{Ultimate Owner} $};
					
					
					\node (CH) [process, below of = start,xshift=3.5cm] {Common Owner};
					
					\node (end) [startstop1,below of = start ] {$ \text{Firm A} $};
					
					\node (end2) [startstop1,right of = CH ,yshift=0cm,xshift=1.5cm] {$ \text{Firm B} $};
					
					\node (sur) [startstop2 ,below of = end ,yshift=0cm,xshift=0cm] {$ \text{Firm X} $};
					
					\node (sur2) [startstop2,below of = end2 ,yshift=0cm,xshift=0cm] {$ \text{Firm Y} $};
					
					
					
					\draw [arrow] (start) --(end);
					
					\draw [arrow] (end) --(sur);
					\draw [arrow] (end2) -- (sur2);
					
					\draw [dash dot,->] (start) -- (CH);
					
					\draw [arrow] (CH) -- (sur);
					\draw [arrow] (CH) -- (sur2);
					
					\draw [latex'-latex'] (sur) to [bend right =0]  node[sloped, anchor=center, below] {} (sur2);
					
					
				\end{tikzpicture}
		
				}
			\caption{ Pair not in the business group}
		\end{subfigure}
			
		
		
	\end{figure}  
	
	
	\captionsetup[subfigure]{labelformat=empty}
	
\DeclareRobustCommand{\myname}{17522}


	
				
		\captionsetup[subtable]{labelformat=parens}
			\renewcommand{\thesubtable}{\Alph{subtable}}
			 \begin{table}[htbp]
			 \caption{ Summary Statistics \\ \small
			 This table reports summary statistics of ownership features for all TSE stocks from 2015 to 2020. Panel \subref{t2-1} lists the total number of firms and Business groups and other features as of the year end for each of the years in our sample. Panel \subref{t2-2} reports summary statistics for firm pairs. The number of unique stock pairs is $ n(n-1)/2 $, where n is the number of stocks. In total, we have \myname unique firm pairs in our sample.  }
			\label{st1}
			\centering
			\subcaption{ Ownership Characteristics for listed firms}
			\label{t2-1}
			\resizebox{1\textwidth}{!}
			{
				\begin{tabular}{lrrrrrr}
\toprule
Year &  2014 &  2015 &  2016 &  2017 &  2018 &  2019 \\
\midrule
No. of Firms                        &   365 &   376 &   446 &   552 &   587 &   618 \\
No. of Blockholders                 &  1606 &  1676 &  2099 &  2978 &  3374 &  3416 \\
No. of Groups                       &    38 &    41 &    43 &    44 &    40 &    43 \\
No. of Firms in Groups              &   249 &   268 &   300 &   336 &   346 &   375 \\
Ave. Number of group Members        &     7 &     7 &     7 &     8 &     9 &     9 \\
Ave. ownership of each Blockholders &    18 &    19 &    18 &    17 &    18 &    19 \\
Med. ownership of each Blockholders &     5 &     4 &     4 &     4 &     4 &     4 \\
Ave. Number of Owners               &     7 &     6 &     6 &     7 &     7 &     7 \\
Ave. Block. Ownership               &    77 &    77 &    75 &    76 &    75 &    72 \\
\bottomrule
\end{tabular}

			}
			
		\centering
	\bigskip
			\subcaption{Number of Pairs, in, and outside the Business Groups }
		\label{t2-2}
		\resizebox{1\textwidth}{!}
		{
			\begin{tabular}{lrrrrrr}
\toprule
year &   1393 &   1394 &   1395 &   1396 &   1397 &   1398 \\
\midrule
No. of Pairs                          &  20876 &  21187 &  27784 &  41449 &  47234 &  67232 \\
No. of Groups                         &     37 &     40 &     42 &     43 &     39 &     43 \\
No. of Pairs not in Groups            &  11452 &  11192 &  15351 &  26530 &  29182 &  43433 \\
Number of Pairs not in the same Group &   7962 &   8731 &  10971 &  12916 &  15366 &  20745 \\
Number of Pairs in the same Group     &    923 &    955 &   1099 &   1260 &   1536 &   1774 \\
Average Number of Common owner        &      1 &      1 &      1 &      1 &      1 &      1 \\
Med. Number of Common owner           &      1 &      1 &      1 &      1 &      1 &      1 \\
Average Percent of each blockholder   &     19 &     19 &     19 &     19 &     19 &     20 \\
Med. Percent of each blockholder      &     13 &     12 &     12 &     12 &     12 &     14 \\
Average Number of Pairs in one Group  &     31 &     30 &     30 &     34 &     39 &     44 \\
Med. Number of Pairs in one Group     &      8 &     10 &      8 &     10 &      9 &     10 \\
Average Number of Owners              &      5 &      5 &      5 &      5 &      4 &      5 \\
Med. Number of Owners                 &      5 &      5 &      5 &      5 &      4 &      5 \\
Average Block. Ownership              &     73 &     73 &     72 &     70 &     70 &     70 \\
Med. Block. Ownership                 &     73 &     73 &     73 &     71 &     71 &     71 \\
\bottomrule
\end{tabular}

		}
	\end{table}
	
	\captionsetup[subtable]{labelformat=empty}

	
	
	


\FloatBarrier


\subsection{{Measurement of common-ownership}}


There are a number of different measures for common ownrship used in the literature. Table \ref{maasurmentsSummary} summarizes all the major common ownership measures, which can be categorized into two groups; model-based (e.g, \cite{harford2011institutional}; \cite{azar2018anticompetitive}; \cite{gilje2020s}) as well as ad hoc measures (e.g, \cite{AntonPolk}; \cite{azar2011new}; \cite{freeman2019effects}; \cite{hansen1996externalities};  \cite{he2017product}; \cite{lewellen2021does}; \cite{newham2018common}). 

%In addition to model-based measures, some ad-hoc common ownership measures are used in the empirical literature. There is significant doubt on how these measures capture common ownership's impact on the management, and many of them have unappealing properties.

	{\begin{table}[htbp]
			\centering
			\scriptsize
			\caption{ Common ownership measurements in the literature.}
			\label{maasurmentsSummary}
			\resizebox{\textwidth}{!}{
				\begin{tabular}{cllc}
	\hline\hline
	\multicolumn{1}{c}{Group}      & \multicolumn{1}{c}{Paper} & \multicolumn{1}{c}{measurment} & \multicolumn{1}{c}{Flaws} \\
	\hline\hline
	\addlinespace
	\multicolumn{1}{c}{\multirow{5}[2]{*}{Model Based}} &  \cite{harford2011institutional}     &  \scriptsize  $
	\sum_{i\in I^{A,B}}\frac{\alpha_{i,B}}{\alpha_{i,A} + \alpha_{i,B}}     $     & Bi-directional \\
	\addlinespace 
	&  \cite{azar2018anticompetitive}     &  $   \sum_{j} \sum_k s_j s_k \frac{\sum_i \mu_{ij} \nu_{ik}}{\sum_i \mu_{ij} \nu_{ij}}   $     & Industry level \\
	\addlinespace
	&  \cite{gilje2020s}     &    $ \sum_{i = 1}^{I} \alpha_{i,A}g(\beta_{i,A})\alpha_{i,B}    $   & Bi-directional  \\
	\midrule
	\addlinespace 
	\multicolumn{1}{c}{\multirow{7}[5]{*}{Ad hoc}} & \cite{he2017product};      &  \multirow{2}{*}{$ \sum_{i\in I^{A,B}} 1 $}     & invariant to the level   \\
	& \cite{he2019internalizing} & & of ‌common ownership \\
	\addlinespace
	&  \cite{newham2018common}     &   $ \sum_{i\in I^{A,B}} min\{\alpha_{i,A},\alpha_{i,B}\} $    & ? \\
	\addlinespace
	& \multirow{2}{*}{   \cite{AntonPolk} }  &  \multirow{2}{*}{ $ \sum_{i\in I^{A,B}} \alpha_{i,A}\frac{\bar{\nu}_A}{\bar{\nu}_A +\bar{\nu}_B } + \alpha_{i,B}\frac{\bar{\nu}_B}{\bar{\nu}_A +\bar{\nu}_B }  $ }   &  Invariant to the  \\
	& & & decomposition of ownership \\
	\addlinespace
	& \cite{freeman2019effects}; & \multirow{2}{*}{ $ \sum_{i\in I^{A,B}} \alpha_{i,A} \times \sum_{i\in I^{A,B}} \alpha_{i,B} $ }&?\\
	&  \cite{hansen1996externalities} & & ?\\
	\hline\hline
\end{tabular}
			}
		\end{table}
	}
\unsure[inline]{Check this table. Is is percise?} 
	
Since we want to estimate the impact of common ownership on stock return comovement in our primary analysis, we need a pair-level measure of common ownership. \cite{AntonPolk} study the impact of common ownership on US stock return comovements using a measure that captures the total value held by the common owners of the two stocks, scaled by the total market capitalization, hereafter \textit{FCAP}. This measure is straightforward to construct, is not bi-directional, and provides a meaningful economic interpretation, which are all features we would like our measure of common ownership to have. One shortcoming of this measure, however, is that it does not capture the distributional impact of ownership by each of the common owners (e.g., \textit{FCAP} yields the same values if common owners each hold 5 percent of a firm's stocks; versus if one holds 1 percent and the other 9 percent of the firm's stocks).\unsure[inline]{Is it important? Our modification is true? We have a grater measures at a firm with 3 common owners than a pair with 2 common owner. Which one is more connected?!} As a result, we propose a modification to \textit{FCAP} that allows us to capture the extent of ownership by each of the common owners, , \textit{MFCAP}, although we replicate our entire analysis with the measure introduced in \cite{AntonPolk}, \textit{FCAP}. Our proposed measure is
\begin{equation}
	MFCAP(i, j) =  [\frac{\sum_{f =1}^{F}(\sqrt{S^f_{i,t}P_{i,t}}+\sqrt{S^f_{j,t}P_{j,t}})}{\sqrt{S_{i,t}P_{i,t}} + \sqrt{S_{j,t}P_{j,t}}}]^2 
	\label{sqrt}
\end{equation}
where $ S^f_{i,t}$ is the number of shares held by owner \textit{f} in firm \textit{i} at time \textit{t} trading at a price $ P{i,t} $ with total shares outstanding of $ S_{i,t} $. Taking the square root of the dollar value of each common owenrs's holding allows us to capture the ownership differences among common owners ({See appendix \ref{ModifiedMeasure}} for further discussion). 

%Modified measure represents the number of equal percents held block-holder. In other words, If for a pair of stocks with n mutual owners, all owners have even shares of each firm's market cap, then the proposed index will be equal to the number of holders.

To construct a monthly measure of common ownership for each firm pair, we calculate the \textit{MFCAP} and \textit{FCAP} every trading day and take the average of the daily values over a month. Panel \subref{measureResults} table \ref{st2} compares the distribution of common ownership measures for both methods. As expected, the modified measure generates a wider distribution of values between the two common ownership measures. The average common ownership measure is five times larger for firms in the same business group. This is consistent with our prior understanding of the ownership structure in the Iranian public sector, in which business groups are one of the main drivers of common ownership. In addition, the average common ownership measure is three times larger for firms that are in the same industry. It is worth noting that firms that belong to the same business group tend to be in the same industry. \info[inline]{Business groups might consist of firms which are similar in fundamentals. So, our result comes from this fundumental connection and it is endogenous!} Hence, as part of our analyses, we also study the impact of business groups on stock return comovement, which is to the best of our knowledge a novel contribution of our paper to the literature.

	
\FloatBarrier
\subsection{{Stock Return comovement}}
\label{comovement}

Since we are interested in studying the impact of common ownership and business groups on stock return comovement and that firms in the same business groups in Iran tend to be from the same industry, we would ideally want to subtract the impact of industry return in calculating abnormal returns. In addition, we know from prior literature that stocks from the same industry tend to comove together {(\cite{king1966market},\cite{meyers1973re})}. Therefore, we use the Fama-French four-factor model plus industry return to calculate abnormal returns, as shown in equation \ref{e5Factor}. To measure monthly stock return comovement for each firm pair at the end of each month, we first estimate our benchmark model using the past three months. Using the estimated coefficients (betas), we then measure daily residuals and calculate the correlation of daily residuals from equation \ref{e5Factor} during each month. 
	\begin{equation}
		\begin{split}
			R_{i,t} =\alpha _{i}&+\beta _{mkt,i}{\mathit {R}}_{M,t} + \beta_{Ind,i}{\mathit {R}}_{Ind,t} + \\
			&+\beta _{HML,i}{\mathit {HML}}_{t}+\beta _{SMB,i}{\mathit {SMB}}_{t}+\beta _{UMD,i}{\mathit {UMD}}_{t}+ \varepsilon_{i,t}
		\end{split}
		\label{e5Factor}
	\end{equation}
	where $ R_{i,t} $, $ R_{M,t} $ and $ R_{Ind,t} $ are firm, market and firm's industry excess daily, respectivly. Our proxy for risk free rate is bank deposit's daily rate. Other variabales difinition is based on Carhart four-factor model [\cite{Carhart4Factor}]. Using other benchmark models (e.g. CAPM and Fama French four factor model) in calculating monthly correlations generate similar results and are reported in panel \subref{tCorr} table \ref{st2}. 
	
	
	
	
%{	\begin{table}[htbp]
%		\centering
%		\caption{\footnotesize This table reports distribution of calculated correlation base on different models.}
%		\label{tCorr}
%%		\resizebox{0.7\textwidth}{!}
%		{
%			\begin{tabular}{lrrrrr}
\toprule
{} &   mean &    std &  min &  median &  max \\
\midrule
 CAPM + Industry    &  0.018 &  0.205 & -1.0 &   0.018 &  1.0 \\
4 Factor            &  0.031 &  0.206 & -1.0 &   0.027 &  1.0 \\
4 Factor + Industry &  0.014 &  0.204 & -1.0 &   0.012 &  1.0 \\
\bottomrule
\end{tabular}

%		}
%	\end{table}}



\FloatBarrier


\subsection{Controls}
\label{control}
 Stocks' intrinsic similarities may well drive return comovement. We follow the literature (e.g., \cite{AntonPolk}) to control for potential drivers of stock return comovements, which can be firm-specific as well as pair characteristics. We separately control for both firms' size and book to market in a pair. Following \cite{AntonPolk}, we use the normalized rank-transform of the percentile market capitalization of the two stocks, \textbf{Size1} and \textbf{Size2}, where we label the larger stock in a pair as the first stock. Similarly, we control for the normalized rank-transform of the percentile book to market of the two stocks, \textbf{BM1} and \textbf{BM2}. 
 We also control for whether firms in a pair are similar in size and book to market: \textbf{SameSize}, and \textbf{SameBM} are the negative of the absolute difference in percentile ranking of size and book to market of the two stocks in a pais, respectively. 
 
 In addition, we control for whether the two stocks are in the same industry and business group, \textbf{SameIndustry}, \textbf{SameGroup}, respectively. We also control for cross-ownership between two stocks and define  \textbf{CrossOwnership} as the maximum percent of cross-ownership between the two firms.


%	{\begin{table}[htbp]
%			\caption{\scriptsize This table reports the number of pairs in the same industry and business group.}
%			\label{SameGroupIndustry}
%			\centering 
%			{
%				
    \begin{tabular}{lcc}\hline\hline
    {Type of Pairs} & {Yes} &{No} \\
    \hline
    \addlinespace
    {SameIndustry} & 1760  & 16739 \\
          & \tiny(10\%) & \tiny (90\%) \\
          \addlinespace
{SameGroup} & 1118  & 17381 \\
          & \tiny(6\%) & \tiny (94\%) \\
          \addlinespace
{SameGroup \& SameIndustry} & 492  & 18007 \\
          & \tiny(3\%) & \tiny (97\%) \\    
                
          \hline\hline
    \end{tabular}%
%			}
%	\end{table}}



We construct our control variables daily and take the monthly averages as the value of each variable at the end of each month. Panel \subref{ControlsSummary} of table \ref{st2} reports the summary statistics of our control variables.
\info[inline]{We can use the calculated correlation in the regression as a latend variable to control for omitted variable biase. It can capture our fundumental correlation from definition of BGs in the similar frims. }

	\captionsetup[subtable]{labelformat=parens}
			\begin{table}[htbp]
			\caption{Summary Statistics of Pairs' Features\\ \small
			This table reports summary statistics for all the founded pairs from 2014 to 2019. Panel \subref{measureResults} reports snapshots from the calculation of common ownership for our measurement of common ownership (MFCAP) and \cite{AntonPolk} measure (FCAP). Panel \subref{tCorr} shows the distribution of calculated correlation of residuals for different models. Panel \subref{ControlsSummary} depicts Control variables' distribution.
			}
			\label{st2}
				%	\centering
%\toprule
%\multirow{2}{*}{Subset}& \multicolumn{5}{c}{MFCAP} & \multicolumn{5}{c}{FCAP} \\
%\cmidrule(lr){2-6} \cmidrule(lr){7-11}
%&       mean &    std &    min & median &    max &         mean &    std &    min & median &    max \\
%\midrule
				\subcaption{Common Ownership with for two measures}
				\label{measureResults}
				\resizebox{1\textwidth}{!}
				{
						{\begin{tabular}{lrrrrrrrrrr}
\toprule
\multirow{2}{*}{Subset}& \multicolumn{5}{c}{MFCAP} & \multicolumn{5}{c}{FCAP} \\
\cmidrule(lr){2-6} \cmidrule(lr){7-11}
&       mean &    std &    min & median &    max &         mean &    std &    min & median &    max \\
\midrule
All               &  0.15 &  0.24 &  0.00 &   0.06 &  4.62 &  0.12 &  0.16 &  0.0 &   0.05 &  0.97 \\
Same Group        &  0.47 &  0.41 &  0.00 &   0.41 &  4.04 &  0.38 &  0.25 &  0.0 &   0.37 &  0.97 \\
Not Same Group    &  0.10 &  0.16 &  0.00 &   0.04 &  2.90 &  0.08 &  0.11 &  0.0 &   0.04 &  0.97 \\
Same Industry     &  0.34 &  0.41 &  0.01 &   0.18 &  4.04 &  0.25 &  0.24 &  0.0 &   0.16 &  0.96 \\
Not Same Industry &  0.12 &  0.19 &  0.00 &   0.05 &  4.62 &  0.10 &  0.14 &  0.0 &   0.05 &  0.97 \\
\bottomrule
\end{tabular}
}
				}
				\bigskip
		\centering
		\subcaption{ Distribution of Correlation base on Different models}
		\label{tCorr}
%		\resizebox{1\textwidth}{!}
		{
			\begin{tabular}{lrrrrr}
\toprule
{} &   mean &    std &  min &  median &  max \\
\midrule
 CAPM + Industry    &  0.018 &  0.205 & -1.0 &   0.018 &  1.0 \\
4 Factor            &  0.031 &  0.206 & -1.0 &   0.027 &  1.0 \\
4 Factor + Industry &  0.014 &  0.204 & -1.0 &   0.012 &  1.0 \\
\bottomrule
\end{tabular}

		}
		
						\bigskip
					
 \subcaption{Distribution of specified Controls}
 \label{ControlsSummary}
               \centering 
%               	\resizebox{1\textwidth}{!}
                 {
    \begin{tabular}{lrrrrrrr}\hline\hline
          & \multicolumn{1}{l}{mean} & \multicolumn{1}{l}{std} & \multicolumn{1}{l}{min} & 25\%  & 50\%  & 75\%  & \multicolumn{1}{l}{max} \\
          \hline
          
          SameIndustry & 0.10  & 0.29  & 0.00  & 0.00  & 0.00  & 0.00  & 1.00 \\
          SameGroup & 0.06  & 0.23  & 0.00  & 0.00  & 0.00  & 0.00  & 1.00 \\
          Size1 & 0.72  & 0.21  & 0.01  & 0.58  & 0.78  & 0.91  & 1.00 \\
          Size2 & 0.43  & 0.25  & 0.00  & 0.23  & 0.42  & 0.62  & 0.99 \\
          SameSize & -0.29 & 0.21  & -0.97 & -0.42 & -0.24 & -0.12 & 0.00 \\
          BookToMarket1 & 0.53  & 0.26  & 0.00  & 0.34  & 0.54  & 0.73  & 1.00 \\
          BookToMarket2 & 0.52  & 0.24  & 0.00  & 0.34  & 0.52  & 0.71  & 1.00 \\
          SameBookToMarket & -0.30 & 0.19  & -0.99 & -0.42 & -0.26 & -0.15 & 0.00 \\
          MonthlyCrossOwnership & 0.01  & 0.05  & 0.00  & 0.00  & 0.00  & 0.00  & 0.96 \\
          
    
    \hline\hline
            \end{tabular}
                 }
             \end{table}


\begin{table}[htbp]
\centerfloat
\caption{Summary Statistics of Sub-samples\\ \small
This table reports the mean of control variables for the three subsamples, for the pairs in the same business group, same industry, and high level of common ownership, which is in the fourth quarter of each period.}
\label{QarterSummary}
\resizebox{\textwidth}{!}{	

\footnotesize
\newcolumntype{Y}{>{\centering\arraybackslash}X}

\begin{tabularx} {1.3\textwidth} {@{} l Y Y Y Y Y Y Y Y Y Y Y Y Y Y Y Y@{}} 
\toprule
 & \multicolumn{6}{c}{Mean} \\
\cmidrule(l{.75em}){2-7} 
&Comovement&SameGroup&SameInd.&SameBM&SameSize&CrossOwner. \\
\hline
SameGroup&&&&&& \\
No (88\%)&1.02\%&0.00&0.09&-0.30&-0.28&0.18\% \\
Yes (11\%)&4.24\%&1.00&0.45&-0.25&-0.26&4.53\% \\
\hline
SameIndustry&&&&&& \\
No (86\%)&1.00\%&0.07&0.00&-0.30&-0.29&0.33\% \\
Yes (13\%)&4.05\%&0.40&1.00&-0.23&-0.21&2.96\% \\
\hline
Pairs&&&&&& \\
Others (74\%)&1.08\%&0.04&0.08&-0.29&-0.29&0.50\% \\
ForthQuarter (25\%)&2.35\%&0.35&0.29&-0.28&-0.24&1.21\% \\
\hline
Total (100\%)&1.40\%&0.11&0.13&-0.29&-0.28&0.68\% \\
\bottomrule
\addlinespace[.75ex]
\end{tabularx}
\par
%\scriptsize{\emph{Source: }auto.dta}
\normalsize

}
\end{table}
	\captionsetup[subtable]{labelformat=empty}
