% !TeX spellcheck = <none>

\section{{Data and Methodology}}



\subsection{{Data and Sample}}



Our data has several unique features which distiguishes our paper from the existing literature. Stock ownership data for Iranian public firms is available with daily frequncy reprted at the end of each trading day. This data includes all blockholders, defined as having at least one percent of the outstanding shares in a firm for all investor types- institutional as well as individuals- and is automatically reported by a central authority named Tehran Securities Exchange Technology Managament Corporation (TSETMC), which is a subsidiary of Tehran Stcok Exchange. This eliminates potential issues present in self-reported data (such as those exisiting in US institutional holding data, known as 13F filings). In order to put together our dataset, we compile daily stock ownerhip tables (available in a separate table for each firm) which are publicaly available starting from 2010.


We use data on business groups, defined as groups of listed firms with interconnected ownership structures controlled by an ultimate owner. Business groups are common organizational structure in corporate ownership in Iran as well as many other parts of the world. Two-third of Iranian public firms are part of complex interlinked ownership networks each governed by an ultimate owner, which sits at the top of a multi-layer pyramid ownership structure ({\cite{Aliabadi2022}}). Unlike countries like South Korea, Japan, and India that formally announce business groups, we do not have officially defined business groups in Iran. We use data provided by {\cite{Aliabadi2022}} which builds a comprehensive data of all Iraninan business groups.  Using two different methodologies, one introduced by \cite{almeida2011structure} with a 40\% threshold for control rights, and the other by \cite{aminadav2011rebuilding} which is based on Shapley-Shubik index (\cite{shapley1954method}) generates the same business group definitions in Iranian public sector({\cite{Aliabadi2022}}).\footnote{For further discussion see appendix \ref{BGDef}} Our business group data covers 2015 to 2020. 



	We also gather stock returns, trading volume, firm-level trading data and accounting information from Codal (equvalent to SEC's EDGAR) \footnote{\href{http://www.codal.ir}{www.codal.ir}}
and TSETMC's website\footnote{\href{http://www.tsetmc.com}{www.tsetmc.com}}. We exclude ETFs. The final sample used in our empirical analysis covers 2015-2020 (1393/01-1398/12 Persian calendar). 



Panel \subref{t2-1} table \ref{st1} reports summary statistics for ownership and business groups data. An average firm in our sample has 6 blockholders holding 75 percent of shares in aggregate. More than two-third of our sample firms are part of a business group. There are around 40 business groups each consist of on average 7 firms.






\subsection{{Pair composition} }

	If any two firms have at least one common blockholder, we consider them a commonly held firm pair. By this definition, there are \input{Elements/pairsnumber}  unique pairs in our entire sample period, which is \input{Elements/PairPercent.tex}of all possible pairs ($\frac{560*559}{2} = 156520$). Firm pairs in our sample have on average \input{Elements/numberofcommonowner.tex}common owners. 
	
	
	\normalcolor
	
	An important feature of common ownership in Iranian public sector is that its main driver seems to be ownership by business groups. This is in contrast to US data in which the rise in common ownership is generally attributed to passively managed funds such as index funds \textit{(add a sentence about the prevalnece of index funds in Iran}). We, therefore, also identify pairs based on whether the two firms belong to any and/or the same business groups. 
	
	If both firms in a pair belong to the same ultimate owner, we identify that firm pair as being in the same business group. The two firms in a pair could also belong to different business groups, or not be part of any business groups. Figure \ref{g2-1} illustrates all possibilities based on whether firm pairs belong to the same, different, or any business groups. In about one third of our pairs, neither of the two firms belong to any business groups, while in about 10 percent of our pairs, the two firms are part of the same business group. Panel \subref{t2-2} of table \ref{st1} reports summary statistics for firm pairs.
	
	\captionsetup[subfigure]{labelformat=parens,font=footnotesize}
			\renewcommand{\thesubfigure}{\Alph{subfigure}}	
	\begin{figure}[htbp]
		\centering
		\caption{ Firm pairs and business groups}
		\label{g2-1}
		\normalcolor
	\bigskip
		\begin{subfigure}[t]{.6\linewidth}
			\centering
			\tiny
			\resizebox{1\textwidth}{!}{
				
\begin{tikzpicture}[node distance=2cm]


\node (start) [startstop] {\normalsize Ultimate Owner};



\node (end) [startstop1,below of = start , yshift=0cm , xshift=-3.5cm ] {\normalsize $ \text{Firm A} $};
\node (end2) [startstop1,below of = start , yshift=0cm , xshift=3.5cm ] {\normalsize $ \text{Firm B} $};



\node (sur) [startstop2 ,below of = end ,yshift=0cm,xshift=0cm] {\normalsize $ \text{Firm X} $};


\node (sur2) [startstop2 ,below of = end2 ,yshift=0cm,xshift=0cm] {\normalsize $ \text{Firm Y} $};


\node (CH) [process, below of = start ,xshift=0] {\normalsize Common Owner};


\draw [arrow] (start) --(end);
\draw [arrow] (end) --(sur);

\draw [arrow] (start) --(end2);

\draw [arrow] (end) --(sur);
\draw [arrow] (end2) -- (sur2);


\draw [arrow] (CH) -- (sur);
\draw [arrow] (CH) -- (sur2);
\draw [dashed ,->] (start) --(CH);

\draw [latex'-latex'] (sur) to [bend right =0]  node[sloped, anchor=center, below] {} (sur2);


\end{tikzpicture}

			}
			\caption{ Pair in the same business group}
		\end{subfigure}

		\begin{subfigure}[t]{.65\linewidth}
			\centering
			\tiny
			\resizebox{1\textwidth}{!}{
						\begin{tikzpicture}[node distance=2cm]
					
					
					\node (start) [startstop] { \normalsize$ \text{Ultimate Owner A} $};
					\node (start2) [startstop,right of = start,xshift=5cm] {\normalsize$ \text{Ultimate Owner B} $};
					
					
					\node (CH) [process, below of = start2,xshift=-3.5cm] {\normalsize Common Owner};
					
					\node (end) [startstop1,below of = start ] {\normalsize $ \text{Firm A} $};
					
					\node (end2) [startstop1,below of = start2 ,yshift=0cm,xshift=0cm] {\normalsize $ \text{Firm B} $};
					
					\node (sur) [startstop2 ,below of = end ,yshift=0cm,xshift=0cm] {\normalsize $ \text{Firm X} $};
					
					\node (sur2) [startstop2,below of = end2 ,yshift=0cm,xshift=0cm] {\normalsize $ \text{Firm Y} $};
					
					
					
					\draw [arrow] (start) --(end);
					\draw [arrow] (start2) -- (end2);
					
					\draw [arrow] (end) --(sur);
					\draw [arrow] (end2) -- (sur2);
					
					\draw [dash dot,->] (start) -- (CH);
					\draw [dash dot,->] (start2) -- (CH);
					
					\draw [arrow] (CH) -- (sur);
					\draw [arrow] (CH) -- (sur2);
					
					\draw [latex'-latex'] (sur) to [bend right =0]  node[sloped, anchor=center, below] {} (sur2);
					
					
				\end{tikzpicture}
		
			}   
			\caption{ Pair in two distinct business group}
		\end{subfigure}
			\bigskip
		\begin{subfigure}[t]{1\linewidth}
			
			\resizebox{0.49\textwidth}{!}{
		\input{Elements/No1BGShape.tex}
			}   
			\hfill
			\resizebox{0.49\textwidth}{!}{
						\begin{tikzpicture}[node distance=2cm]
					
					
					\node (start) [startstop] { $ \text{Ultimate Owner} $};
					
					
					\node (CH) [process, below of = start,xshift=3.5cm] {Common Owner};
					
					\node (end) [startstop1,below of = start ] {$ \text{Firm A} $};
					
					\node (end2) [startstop1,right of = CH ,yshift=0cm,xshift=1.5cm] {$ \text{Firm B} $};
					
					\node (sur) [startstop2 ,below of = end ,yshift=0cm,xshift=0cm] {$ \text{Firm X} $};
					
					\node (sur2) [startstop2,below of = end2 ,yshift=0cm,xshift=0cm] {$ \text{Firm Y} $};
					
					
					
					\draw [arrow] (start) --(end);
					
					\draw [arrow] (end) --(sur);
					\draw [arrow] (end2) -- (sur2);
					
					\draw [dash dot,->] (start) -- (CH);
					
					\draw [arrow] (CH) -- (sur);
					\draw [arrow] (CH) -- (sur2);
					
					\draw [latex'-latex'] (sur) to [bend right =0]  node[sloped, anchor=center, below] {} (sur2);
					
					
				\end{tikzpicture}
		
				}
			\caption{ Pair not in the business group}
		\end{subfigure}
			
		
		
	\end{figure}  
	
	
	\captionsetup[subfigure]{labelformat=empty}
	
\DeclareRobustCommand{\myname}{\input{Elements/pairsnumber.tex}}


	
				
		\captionsetup[subtable]{labelformat=parens}
			\renewcommand{\thesubtable}{\Alph{subtable}}
			 \begin{table}[htbp]
			 \caption{ Summary Statistics \\ \small
			 This table reports summary statistics of ownership features for all TSE stocks from 2015 to 2020. Panel \subref{t2-1} lists the total number of firms and Business groups and other features as of the year end for each of the years in our sample. Panel \subref{t2-2} reports summary statistics for firm pairs. The number of unique stock pairs is $ n(n-1)/2 $, where n is the number of stocks. In total, we have \myname unique firm pairs in our sample.  }
			\label{st1}
			\centering
			\subcaption{ Ownership Characteristics for listed firms}
			\label{t2-1}
			\resizebox{1\textwidth}{!}
			{
				  \begin{tabular}{lccccccc}
          \hline\hline
          Year  & \multicolumn{1}{c}{2015} & \multicolumn{1}{c}{2016} & \multicolumn{1}{c}{2017} & \multicolumn{1}{c}{2018} & \multicolumn{1}{c}{2019} & \multicolumn{1}{c}{2020} & Meann \\\hline
     No. of Firms & 355   & 383   & 520   & 551   & 579   & 602   & 498 \\
     No. of Blockholders & 724   & 887   & 1274  & 1383  & 1409  & 1390  & 1178 \\
     No. of Groups & 41    & 42    & 46    & 45    & 40    & 40    & 42 \\
     No. of Firms not in Groups & 113   & 128   & 207   & 224   & 247   & 270   & 198 \\
     No. of Firms in Groups & 242   & 265   & 332   & 339   & 332   & 332   & 307 \\
     Mean Number of Members & 6     & 6     & 7     & 8     & 8     & 8     & 7 \\
%     Max. Number of Members & 24    & 25    & 27    & 28    & 28    & 28    & 27 \\
     Med. of Number of Members & 4     & 4     & 6     & 5     & 6     & 6     & 5 \\
     Mean Of each Blockholder's ownership & 21.30 & 22.00 & 20.80 & 20.50 & 21.90 & 23.00 & 21.58 \\
     Med. of Owners' Percent & 7.94  & 7.55  & 6.95  & 6.34  & 8.31  & 9     & 8 \\
     Mean Number of Blockholders & 5     & 5     & 5     & 5     & 5     & 4     & 5 \\
     Med. Number of Owners & 4     & 4     & 4     & 4     & 4     & 3     & 4 \\
%     Max. Number of Owners & 21    & 22    & 28    & 27    & 25    & 24.00 & 25 \\
     Mean Block. Ownership & 71.6  & 71.2  & 68    & 67.7  & 65.4  & 62.00 & 67.65 \\
     Med. Block. Ownership & 79.9  & 80.1  & 77    & 77.1  & 72.9  & 69.70 & 76.12 \\
%     Max. Block. Ownership & 100   & 100   & 100   & 100   & 100   & 100   & 100 \\
          \hline\hline 
%                  \multicolumn{8}{l}{\footnotesize \textit{}}}\\
    
          \end{tabular}
         
			}
			
		\centering
	\bigskip
			\subcaption{Number of Pairs, in, and outside the Business Groups }
		\label{t2-2}
		\resizebox{1\textwidth}{!}
		{
			\begin{tabular}{lrrrrrr}
\toprule
Year &   2014 &   2015 &   2016 &   2017 &   2018 &   2019 \\
\midrule
No. of Pairs                       &  18368 &  19285 &  22817 &  37204 &  44722 &  60631 \\
No. of Pairs not in Groups         &   9555 &   9927 &  11831 &  23630 &  27425 &  38474 \\
No. of Pairs not in the same Group &   7564 &   8194 &   9604 &  11748 &  14776 &  19351 \\
No. of Pairs in the same Group     &    893 &    885 &   1038 &   1174 &   1511 &   1698 \\
Ave. Number of Common owner        &      1 &      1 &      1 &      1 &      1 &      1 \\
\bottomrule
\end{tabular}

		}
	\end{table}
	
	\captionsetup[subtable]{labelformat=empty}

	
	
	


\FloatBarrier


\subsection{{Measurement of common-ownership}}


There are a number of different measures for common ownrship used in the literature. Table \ref{maasurmentsSummary} summarizes all the major common ownership measures, which can be categorized into two groups; model-based (e.g, \cite{harford2011institutional}; \cite{azar2018anticompetitive}; \cite{gilje2020s}) as well as ad hoc measures (e.g, \cite{AntonPolk}; \cite{azar2011new}; \cite{freeman2019effects}; \cite{hansen1996externalities};  \cite{he2017product}; \cite{lewellen2021does}; \cite{newham2018common}). 

%In addition to model-based measures, some ad-hoc common ownership measures are used in the empirical literature. There is significant doubt on how these measures capture common ownership's impact on the management, and many of them have unappealing properties.
	{\begin{table}[htbp]
			\centering
			\scriptsize
			\caption{ Common ownership measurements in the literature.}
			\label{maasurmentsSummary}
			\resizebox{\textwidth}{!}{
				\begin{tabular}{cllc}
	\hline\hline
	\multicolumn{1}{c}{Group}      & \multicolumn{1}{c}{Paper} & \multicolumn{1}{c}{measurment} & \multicolumn{1}{c}{Flaws} \\
	\hline\hline
	\addlinespace
	\multicolumn{1}{c}{\multirow{5}[2]{*}{Model Based}} &  \cite{harford2011institutional}     &  \scriptsize  $
	\sum_{i\in I^{A,B}}\frac{\alpha_{i,B}}{\alpha_{i,A} + \alpha_{i,B}}     $     & Bi-directional \\
	\addlinespace 
	&  \cite{azar2018anticompetitive}     &  $   \sum_{j} \sum_k s_j s_k \frac{\sum_i \mu_{ij} \nu_{ik}}{\sum_i \mu_{ij} \nu_{ij}}   $     & Industry level \\
	\addlinespace
	&  \cite{gilje2020s}     &    $ \sum_{i = 1}^{I} \alpha_{i,A}g(\beta_{i,A})\alpha_{i,B}    $   & Bi-directional  \\
	\midrule
	\addlinespace 
	\multicolumn{1}{c}{\multirow{7}[5]{*}{Ad hoc}} & \cite{he2017product};      &  \multirow{2}{*}{$ \sum_{i\in I^{A,B}} 1 $}     & invariant to the level   \\
	& \cite{he2019internalizing} & & of ‌common ownership \\
	\addlinespace
	&  \cite{newham2018common}     &   $ \sum_{i\in I^{A,B}} min\{\alpha_{i,A},\alpha_{i,B}\} $    & ? \\
	\addlinespace
	& \multirow{2}{*}{   \cite{AntonPolk} }  &  \multirow{2}{*}{ $ \sum_{i\in I^{A,B}} \alpha_{i,A}\frac{\bar{\nu}_A}{\bar{\nu}_A +\bar{\nu}_B } + \alpha_{i,B}\frac{\bar{\nu}_B}{\bar{\nu}_A +\bar{\nu}_B }  $ }   &  Invariant to the  \\
	& & & decomposition of ownership \\
	\addlinespace
	& \cite{freeman2019effects}; & \multirow{2}{*}{ $ \sum_{i\in I^{A,B}} \alpha_{i,A} \times \sum_{i\in I^{A,B}} \alpha_{i,B} $ }&?\\
	&  \cite{hansen1996externalities} & & ?\\
	\hline\hline
\end{tabular}
			}
		\end{table}
	}
	
Since in our primary analysis we want to estimate the impact of common ownership on stock return comovement, we need a pair-level measure of common ownership. (\cite{AntonPolk}) study the impact of common ownership on US stock return comovements using a measure that captures the total value held by the common owners of the two stocks, scaled by the total market capitalization of the two stocks. This measure is straight forward to construct, is not bi directional and provides a meaningful economic interpretation, which all are features we would like our measure of common ownership to have. One short coming of this measure, however, is it does not capture the distributional impact of owenrship by each of the common owners (e.g. the measure yields the same values if common owners each hold 5 percent of a firm's stocks; versus if one holds 1 percent and the other 9 percent of the firm's stocks). As a result, we propose a modification to the measure used in (\cite{AntonPolk}) that allows us to capture the extent of ownership by each of the common owners, although we replicate our entire analysis with the measure introduced in (\cite{AntonPolk}). Our proposed measure is
\begin{equation}
	\text{Overlap}_{Sqrt}(i, j) =  [\frac{\sum_{f =1}^{F}(\sqrt{S^f_{i,t}P_{i,t}}+\sqrt{S^f_{j,t}P_{j,t}})}{\sqrt{S_{i,t}P_{i,t}} + \sqrt{S_{j,t}P_{j,t}}}]^2 
	\label{sqrt}
\end{equation}
where $ S^f_{i,t}$ is the number of shares held by owner \textit{f} in firm \textit{i} at time \textit{t} trading at a price $ P{i,t} $ with total shares outstanding of $ S_{i,t} $. Taking the square root of the dollar value of each common owenrs's holding, allows us to capture the ownership differences among common owners ({See appendix \ref{ModifiedMeasure}} for further discussion). 

%Modified measure represent the number of equal percents held block-holder. In other words, If for a pair of stocks with n mutual owners, all owners have even shares of each firm's market cap, then the proposed index will be equal to number of holders.

To construct a monthly measure of common ownership for each firm pair, we calculate the measure introduced in equation \ref{sqrt} every trading day and take the average of the daily values over a month. We repeat the same process for constructing the measure in  (\cite{AntonPolk}. Panel \subref{measureResults} table \ref{st2} compares the distribution of common ownership measures for both methods. As we expected, the modified measure generates a wider distribution of values between the two common ownerhsip measures. The average common ownership measure is five times larger for firms that are in the same business group. This is consistent with our prior understadning of the ownerhisp structure in Iranian public sector in which business groups are one of the main drivers of common ownership. In addition, the average common ownership measure is three times larger for firms that are in the same industry. It is worth noting that firms that belong to the same business group tend to be in the same industry.
%		\begin{table}[htbp]
%			%	\centering
%%			\toprule
%%			 & \multicolumn{5}{c}{MonthlyFCA} & \multicolumn{5}{c}{MonthlyFCAPf} \\
%%			 \cmidrule(lr){2-6} \cmidrule(lr){7-11}
%%			 &       mean &    std &    min & median &    max &         mean &    std &    min & median &    max \\
%%			\midrule
%			\caption{Calculation of common ownership with two measure}
%			\label{measureResults}
%			\resizebox{1\textwidth}{!}
%			{
%					{\begin{tabular}{lllllllllll}
\toprule
 & \multicolumn{5}{c}{MFCAP} & \multicolumn{5}{c}{FCAP} \\
\cmidrule(lr){2-6}  \cmidrule(lr){7-11} 
 &       mean &    std &    min & median &    max &         mean &    std &    min & median &    max \\
\midrule
All               &      0.158 &  0.234 &  0.002 &  0.079 &  12.65 &        0.144 &  0.166 &  0.002 &  0.077 &    1.0 \\
Same Group        &      0.474 &  0.478 &  0.005 &  0.367 &  6.174 &        0.346 &  0.265 &  0.004 &  0.321 &    1.0 \\
Not Same Group    &      0.147 &  0.212 &  0.002 &  0.077 &  12.65 &        0.137 &  0.157 &  0.002 &  0.074 &    1.0 \\
Same Industry     &      0.274 &  0.383 &  0.003 &  0.126 &  6.262 &        0.207 &  0.215 &  0.003 &   0.12 &  0.999 \\
Not Same Industry &       0.15 &  0.217 &  0.002 &  0.077 &  12.65 &         0.14 &  0.161 &  0.002 &  0.074 &    1.0 \\
\bottomrule
\end{tabular}
}
%			}
%		\end{table}
%		
%		\multirow{2}[3]{*}{variable} & \multicolumn{5}{c}{MFCAP} & \multicolumn{5}{c}{FCAP} \\
%		 \cmidrule(lr){2-6} \cmidrule(lr){7-11}
%		 &       mean &   std &   min & median &   max &         mean &   std &  min & median &   max \\
		
\FloatBarrier
\subsection{{Stock Return comovement}}
\label{comovement}

	We calculate the monthly correlation of each pair from stocks' daily abnormal returns. Benchmark for calculating abnormal return is the following equation which is a four-factor model plus industry return due to the importance of industries on stocks' return in the Tehran stock exchange (TSE) :
	\begin{equation}
		\begin{split}
			R_{i,t} =\alpha _{i}&+\beta _{mkt,i}{\mathit {R}}_{M,t} + \beta_{Ind,i}{\mathit {R}}_{Ind,t} + \\
			&+\beta _{HML,i}{\mathit {HML}}_{t}+\beta _{SMB,i}{\mathit {SMB}}_{t}+\beta _{UMD,i}{\mathit {UMD}}_{t}+ \varepsilon_{i,t}
		\end{split}
		\label{e5Factor}
	\end{equation}
	where $ R_{i,t} $, $ R_{M,t} $ and $ R_{Ind,t} $ are excess daily return of respectivly  firm, market and firm's industry from bank deposit's daily rate(risk free). Other variabales difinition is base on Carhart four-factor model [\cite{Carhart4Factor}].
	
	At the end of each month, we estimate our benchmark model base on the past three-month period (from two months before the end of the preceding month) and measure daily residuals.  After that, we calculate the monthly correlation of daily residuals during that month for the pair.
	
	We use other benchmarks (CAPM, 4 Factor, and Benchmark\footnote{we follow \cite{daniel1997measuring} to control risk characteristics: abnormal returns are calculated using a stock’s daily return minus the average return of
	the stock’s benchmark group, which is formed at every month’s end based on stocks’ capitalization and market-to-book ratio using the sample of all stocks}) for calculating a monthly correlation and report its summary in panel \subref{tCorr} table \ref{st2}. 
	As we expected,  models that include industry returns remove pairs' correlation. According to the results, it seems that our selected benchmark (4 Factor + Industry) almost captures all the pairs' comovement because it is nearly a zero mean variable. We use this correlation for our analysis but our results are robust for other models.
	
	
	
	
%{	\begin{table}[htbp]
%		\centering
%		\caption{\footnotesize This table reports distribution of calculated correlation base on different models.}
%		\label{tCorr}
%%		\resizebox{0.7\textwidth}{!}
%		{
%			\begin{tabular}{lrrrrr}
\toprule
{} &   mean &    std &    min &  median &    max \\
\midrule
 CAPM + Industry    &  0.019 &  0.127 & -0.925 &   0.015 &  0.902 \\
4 Factor            &  0.032 &  0.136 & -0.877 &   0.023 &  0.837 \\
4 Factor + Industry &  0.015 &  0.125 & -0.903 &   0.012 &  0.755 \\
\bottomrule
\end{tabular}

%		}
%	\end{table}}



\FloatBarrier


\subsection{Controls}

We are interested in the effects of common ownership and business group on pair's comovement.
Our prediction of a higher correlation for a connection dominates by stocks' intrinsic similarity, and these similarities motivate block-holders to hold these stocks simultaneously. These related stocks will comove regardless of who owns them.

The first group of controls is pair controls. These controls include
a dummy variable for whether two stocks are in the same industry, \textbf{SameIndustry}; a dummy variable for whether two stocks are in the same business group, \textbf{SameGroup}.
10\% and 14\%of pairs are in the same industry and business group. Furthermore, we control for cross-ownership between two stocks and define  \textbf{CrossOwnership} as the maximum percent of cross-ownership between two firms in the following month.


%	{\begin{table}[htbp]
%			\caption{\scriptsize This table reports the number of pairs in the same industry and business group.}
%			\label{SameGroupIndustry}
%			\centering 
%			{
%				\begin{tabular}{lll}
\toprule
{} &     Yes &        No \\
\midrule
SameIndustry             &  749265 &  12348105 \\
                         &  (5.7\%) &   (94.3\%) \\
SameGroup                &  302610 &   4480905 \\
                         &  (6.3\%) &   (93.7\%) \\
SameGroup \& SameIndustry &  114840 &  13097370 \\
                         &  (0.9\%) &   (99.1\%) \\
\bottomrule
\end{tabular}

%			}
%	\end{table}}


Another group of controls are firm-specific controls.  We define these variables base on  \cite{AntonPolk} methodology. One of these is size control based on the normalized rank-transform of the percentile market capitalization of the two stocks, \textbf{Size1} and \textbf{Size2} (where we label the
larger stock in the pair as the first stock). The other one is a book to market ratio based on the normalized rank-transform of the percentile book to market of the two stocks, \textbf{BM1} and \textbf{BM2}.
We also control these characteristics on a pair level. Our measures of similarity, \textbf{SameSize}, and \textbf{SameBM}, are the negative of the absolute difference in percentile ranking for a particular characteristic across a pair.


We calculate our controls daily and then report the average of these variables for the entire period at the end of each month. Panel \subref{ControlsSummary} table \ref{st2} shows the summary statistics of specified controls in this section. In addition


	\captionsetup[subtable]{labelformat=parens}
			\begin{table}[htbp]
			\caption{Summary Statistics of Pairs' Features\\ \small
			This table reports summary statistics for all the founded pairs from 2014 to 2019. Panel \subref{measureResults} reports snapshots from the calculation of common ownership for our measurement of common ownership (MFCAP) and \cite{AntonPolk} measure (FCAP). Panel \subref{tCorr} shows the distribution of calculated correlation of residuals for different models. Panel \subref{ControlsSummary} depicts Control variables' distribution.
			}
			\label{st2}
				%	\centering
%\toprule
%\multirow{2}{*}{Subset}& \multicolumn{5}{c}{MFCAP} & \multicolumn{5}{c}{FCAP} \\
%\cmidrule(lr){2-6} \cmidrule(lr){7-11}
%&       mean &    std &    min & median &    max &         mean &    std &    min & median &    max \\
%\midrule
				\subcaption{Common Ownership with for two measures}
				\label{measureResults}
				\resizebox{1\textwidth}{!}
				{
						{\begin{tabular}{lllllllllll}
\toprule
 & \multicolumn{5}{c}{MFCAP} & \multicolumn{5}{c}{FCAP} \\
\cmidrule(lr){2-6}  \cmidrule(lr){7-11} 
 &       mean &    std &    min & median &    max &         mean &    std &    min & median &    max \\
\midrule
All               &      0.158 &  0.234 &  0.002 &  0.079 &  12.65 &        0.144 &  0.166 &  0.002 &  0.077 &    1.0 \\
Same Group        &      0.474 &  0.478 &  0.005 &  0.367 &  6.174 &        0.346 &  0.265 &  0.004 &  0.321 &    1.0 \\
Not Same Group    &      0.147 &  0.212 &  0.002 &  0.077 &  12.65 &        0.137 &  0.157 &  0.002 &  0.074 &    1.0 \\
Same Industry     &      0.274 &  0.383 &  0.003 &  0.126 &  6.262 &        0.207 &  0.215 &  0.003 &   0.12 &  0.999 \\
Not Same Industry &       0.15 &  0.217 &  0.002 &  0.077 &  12.65 &         0.14 &  0.161 &  0.002 &  0.074 &    1.0 \\
\bottomrule
\end{tabular}
}
				}
				\bigskip
		\centering
		\subcaption{ Distribution of Correlation base on Different models}
		\label{tCorr}
%		\resizebox{1\textwidth}{!}
		{
			\begin{tabular}{lrrrrr}
\toprule
{} &   mean &    std &    min &  median &    max \\
\midrule
 CAPM + Industry    &  0.019 &  0.127 & -0.925 &   0.015 &  0.902 \\
4 Factor            &  0.032 &  0.136 & -0.877 &   0.023 &  0.837 \\
4 Factor + Industry &  0.015 &  0.125 & -0.903 &   0.012 &  0.755 \\
\bottomrule
\end{tabular}

		}
		
						\bigskip
					
 \subcaption{Distribution of specified Controls}
 \label{ControlsSummary}
               \centering 
%               	\resizebox{1\textwidth}{!}
                 {
    \begin{tabular}{lrrrrr}
\toprule
{} &  mean &   std &   min &  median &    max \\
\midrule
Size1          &  0.72 &  0.22 &  0.01 &    0.77 &   1.00 \\
Size2          &  0.45 &  0.24 &  0.00 &    0.43 &   0.99 \\
SameSize       & -0.28 &  0.20 & -0.97 &   -0.23 &  -0.00 \\
BM1            &  0.51 &  0.25 &  0.00 &    0.52 &   1.00 \\
BM2            &  0.50 &  0.23 &  0.01 &    0.50 &   1.00 \\
SameBM         & -0.30 &  0.19 & -0.96 &   -0.26 &  -0.00 \\
CrossOwnership &  0.56 &  5.14 &  0.00 &    0.00 &  95.56 \\
\bottomrule
\end{tabular}

                 }
             \end{table}


\begin{table}[htbp]
\centerfloat
\caption{Summary Statistics of Sub-samples\\ \small
This table reports the mean of control variables for the three subsamples, for the pairs in the same business group, same industry, and high level of common ownership, which is in the fourth quarter of each period.}
\label{QarterSummary}
\resizebox{\textwidth}{!}{	
\begin{center}
\footnotesize
\newcolumntype{Y}{>{\centering\arraybackslash}X}

\begin{tabularx} {\textwidth} {@{} l Y Y Y Y Y Y Y Y Y Y Y Y Y Y Y Y@{}} 
\toprule
 & \multicolumn{4}{c}{Mean} \\
\cmidrule(l{.75em}){2-5} 
&SameB/M&SameInd.&SameSize&CrossOwner. \\
\hline
Pairs&&&& \\
Others (n=323,046)&-0.30&0.08&-0.28&0.52 \\
Forth Quarter (n=107,741)&-0.28&0.29&-0.24&1.21 \\
Total (n=430,787)&-0.29&0.13&-0.27&0.69 \\
\hline
Same Group&&&& \\
No (n=380,511)&-0.30&0.09&-0.28&0.18 \\
Yes (n=50,276)&-0.25&0.46&-0.26&4.61 \\
Total (n=430,787)&-0.29&0.13&-0.27&0.69 \\
\bottomrule
\addlinespace[.75ex]
\end{tabularx}
\par
%\scriptsize{\emph{Source: }auto.dta}
\normalsize
\end{center}

}
\end{table}
	\captionsetup[subtable]{labelformat=empty}
