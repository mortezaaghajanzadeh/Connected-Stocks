\section{Conclusion}

 A unique feature of the Iranian stock ownership data is that it is published with daily frequency by the central authority in charge of trading clearing house. This allows us to revisit an important question on the impact of common ownership on stock return comovements. Two firms are defined as commonly owned if they share a direct common owner. Even short of direct common ownership, however, firms can be part of the same business groups, defined as a group of listed firms with interconnected ownership structures controlled by an ultimate owner. Given the prevalent presence of business groups in Iran's public sector, we focus on both direct and indirect common ownership. Direct common ownership is proxied by a modified version of the measure introduced in \cite{AntonPolk}. We use affiliation with the same busness group as a proxy for indirect common ownership. The average common ownership measure is five times larger for firms in the same business group. 
 
We find that business group affiliation is positively associated with higher stock return comovement. Moreover, among firm pairs that belong to the same business groups, those with higher direct common ownership experience higher levels of return comovement. Additional analyses suggest simultaneous trades in the same direction among firms affiliated with the same business groups explains higher return comovements among those stocks, and that direct common ownership likely facilitate simultaneous trading.
\improvement{Is there any implications?}

	


\newpage
	{
	\footnotesize
		\bibliographystyle{apalike}
		\bibliography{../Report/Ref}
	}





