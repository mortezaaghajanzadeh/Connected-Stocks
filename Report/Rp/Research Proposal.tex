% !TeX spellcheck = en_US
\documentclass[12pt, a4paper]{article}
\usepackage{comment}
\usepackage{ragged2e}
\usepackage{amsmath}
\usepackage{xcolor}
\usepackage{multirow}
\usepackage{caption}
\usepackage{tikz}
\usepackage{booktabs}
\usepackage{tabu}
\usepackage{placeins}
\usepackage{pdflscape}
\usetikzlibrary{arrows}
\usepackage{hyperref}
\usepackage{multirow}
\usepackage{subcaption}
\usepackage{pdflscape}

\usepackage{color,soul}

\title{\textbf{Research Proposal}}
\author{}
\date{}

\newcommand{\namelistlabel}[1]{\mbox{#1}\hfil}
\newenvironment{namelist}[1]{%1
	\begin{list}{}
		{
			\let\makelabel\namelistlabel
			\settowidth{\labelwidth}{#1}
			\setlength{\leftmargin}{1.1\labelwidth}
		}
	}{%1
\end{list}}

\captionsetup{font=footnotesize,labelfont=footnotesize}

\hypersetup{
    colorlinks=true,
    linkcolor=blue,
    filecolor=blue,      
    urlcolor=blue,
    citecolor=blue
}

\usepackage{natbib}
\usepackage[title]{appendix}


\def\sym#1{\ifmmode^{#1}\else\(^{#1}\)\fi}


\renewcommand{\today}{\ifcase \month \or January\or February\or March\or %
April\or May \or June\or July\or August\or September\or October\or November\or %
December\fi, \number \year} 



\def\boxit#1{%
  \smash{\color{red}\fboxrule=1pt\relax\fboxsep=2pt\relax%
  \llap{\rlap{\fbox{\vphantom{0}\makebox[#1]{}}}~}}\ignorespaces
}


\usepackage{lipsum}

%\linespread{1.2}


%\title{{Connected Stocks via Business Groups: Evidence from an Emerging Market}}
\author{{S.M. Aghajanzadeh\sym{*} \qquad M. Heidari\sym{*} \qquad M. Mohseni\sym{*} }\\
{\sym{*} \footnotesize  Tehran Institute for Advanced Studies, Khatam University, Tehran, Iran}
}

\begin{document}
\maketitle
\begin{namelist}{xxxxxxxxxxxx}
	\item[{\bf Title:}]
	\textit{Connected Stocks via Business Groups: Evidence from an Emerging Market }
	\item[{\bf Author:}]
	\textit{ Seyyed Morteza Aghajanzadeh}
	\item[{\bf Supervisors:}]
	\textit{	Dr. Mahdi Heidari, Dr. Mahdi Mohseni}
	\item[{\bf Institution:}]
	\textit{	Tehran Institute for Advanced Studies}
\end{namelist}


\section*{Research Objective}
Related literature points out that common ownership and business groups are non-fundamental factors that lead to co-movement in stock returns. \hl{Using unique Iran's financial market context, we shed light on factors that affect co-movement.}
\section*{Motivation}
	{
		
		\hl{The phenomenon of "co-movement" has been observed by researchers and analysts. There is an increase in interest in risk models, notably after the financial crisis of 2008. According to these models, price correlation plays a significant role in risk measurement.}
		 While first coming  investigations   attributed the companies' return co-movement to their fundamentals, (e.g. {\cite{shiller1989comovements}})}, recent findings have focused on the role of non-fundamental characteristics. {\cite{barberis2003style} and \cite{barberis2005comovement}} provided theoretical models for predicting the co-movement between fundamentally unrelated companies.	%	Among various factors, the common ownership has been subject to lots of investigations. of on firms' behavior. There has been a surge in the popularity of index investing in the United States, that \hl{led to an increase in common ownership}. 
%			For instance, \cite{azar2018anticompetitive} claims that an increase in common ownership in airline companies leads to less competitive ticket pricing. However, this subject is controversial and many papers discuss whether common ownership affects companies' behavior.
%			 \cite{lewellen2021does} stated that in previous investigations, other effective factors have wrongly been replaced by common ownership effect.
		{Trying to explain factors affecting co-movement, \cite{AntonPolk}} suggest that common ownership positively affects co-movement\footnote{There are some factors like, Index inclusion ({\cite{barberis2005comovement}}), investors' attention to the companies ({\cite{wu2014investor}}), Investment banks' underwriting ({\cite{grullon2014comovement}}), correlated beliefs ({\cite{david2016correlated}}), shareholders' coordination ({\cite{pantzalis2017shareholder}}), and preference for companies' dividends ({\cite{HAMEED2019103}}) that have been identified by researchers.}.
			Subsequently, {\cite{Liquidity2016}} provides evidence that even owners' liquidity needs' correlation can result in co-movement independent of direct common ownership. 	
		
		
		
		
		Another strand of the literature tries to investigate effect of the common ownership on firms' behavior
		
			
Besides, As a common phenomena in both developed and developing countries, there is a debate about the pros and cons of business groups .[\cite{khanna2000group}].
While the co-movement in business groups is accepted, the co-movement channels remained undiscovered.
Both {\cite{cho2015stock} and \cite{kim2015stock}} studied the South Korean market and suggested two different sources for the co-movement in business groups. The first paper attributed co-movement to the companies' fundamentals. However, the second paper presents that the investors' category/habitat behavior is responsible for co-movement.
		
Measures of common ownership have been categorized into two types in the literature.
First of all, model-based measures that capture common ownership base on a proper  model. These measures have a better economic interpretation, but most of them are bi-directional or industry-level measures.(e.g, \cite{harford2011institutional}; \cite{azar2018anticompetitive}; \cite{gilje2020s}) In addition to model-based measures, some ad hoc common ownership measures are used in the empirical literature. There is significant doubt on how these measures capture common ownership's impact on the management, and many of them have unappealing properties.(e.g, \cite{AntonPolk}; \cite{azar2011new}; \cite{freeman2019effects}; \cite{hansen1996externalities};  \cite{he2017product}; \cite{he2019internalizing}; \cite{lewellen2021does}; \cite{newham2018common})		

\section*{Data}
We  use our unique data set, including the daily ownership table that reports all end-of-the-days block-holders of listed firms with their changes in that day.  Block-holder is a shareholder who owns at least 1\% of the total shares outstanding. 
	We also gathered industries index and stock returns, trading volume, and other relevant market and accounting data from the Codal website \footnote{\href{http://www.codal.ir}{www.codal.ir}}
and the  Tehran Securities Exchange Technology Management Co (TSETMC)\footnote{\href{http://www.tsetmc.com}{www.tsetmc.com}} database.
\section*{Methodology}
We use the same methodoloy as  \cite{AntonPolk} to compose pairs, define control variables, and calculating co-movement. 
A method wildly used in the Emperical asset pricing is the two-step approach of \cite{FamaMacBeth}. In the first step, for each time period, cross-sectional regressions
are used to obtain estimates of the parameters of interest. Then, in the second step, the time series
of these estimates are used to obtain final estimates for the parameters and standard errors so that
t-statistics can be computed. [\cite{skoulakis2008panel}] 


\section*{Contribution}
According to the restrictions of data in the US that quarterly fund ownership data is available, investigations in this area are limited to the fund ownership impact on co-movement. This type of owners perform particular types of behavior due to their needs and the fact that they are intermediates.
	Nevertheless, in Iran, the block holders' daily ownership data, including mutual fund ownership, is publicly accessible. So research through this data can show whether common ownership other than mutual fund ownership can lead to co-movement or not.

	
		In this paper, we consider the co-movement of the companies in business groups. Best of our knowledge, it is the first study that compares direct and indirect common ownership. Also, a modified measurement is introduced in this paper to calculate the common ownership of the companies. 











%%%%%%%%%%%%%%%%%%%%%%%%%%%%%%%%%%
%%%%%%%%%%%%%%%%%%%%%%%%%%%%%

	


	
\newpage
	{
	\footnotesize
	\bibliographystyle{apalike}
	\bibliography{../Ref}
}

\end{document}