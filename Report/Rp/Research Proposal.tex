% !TeX spellcheck = en_US
\documentclass[12pt, a4paper]{article}
\usepackage{comment}
\usepackage{ragged2e}
\usepackage{amsmath}
\usepackage{xcolor}
\usepackage{multirow}
\usepackage{caption}
\usepackage{tikz}
\usepackage{booktabs}
\usepackage{tabu}
\usepackage{placeins}
\usepackage{pdflscape}
\usetikzlibrary{arrows}
\usepackage{hyperref}
\usepackage{multirow}
\usepackage{subcaption}
\usepackage{pdflscape}

\usepackage{color,soul}

\title{\textbf{Research Proposal}}
\author{}
\date{}

\newcommand{\namelistlabel}[1]{\mbox{#1}\hfil}
\newenvironment{namelist}[1]{%1
	\begin{list}{}
		{
			\let\makelabel\namelistlabel
			\settowidth{\labelwidth}{#1}
			\setlength{\leftmargin}{1.1\labelwidth}
		}
	}{%1
\end{list}}

\captionsetup{font=footnotesize,labelfont=footnotesize}

\hypersetup{
    colorlinks=true,
    linkcolor=blue,
    filecolor=blue,      
    urlcolor=blue,
    citecolor=blue
}

\usepackage{natbib}
\usepackage[title]{appendix}


\def\sym#1{\ifmmode^{#1}\else\(^{#1}\)\fi}


\renewcommand{\today}{\ifcase \month \or January\or February\or March\or %
April\or May \or June\or July\or August\or September\or October\or November\or %
December\fi, \number \year} 



\def\boxit#1{%
  \smash{\color{red}\fboxrule=1pt\relax\fboxsep=2pt\relax%
  \llap{\rlap{\fbox{\vphantom{0}\makebox[#1]{}}}~}}\ignorespaces
}


\usepackage{lipsum}

%\linespread{1.2}


%\title{{Connected Stocks via Business Groups: Evidence from an Emerging Market}}
\author{{S.M. Aghajanzadeh\sym{*} \qquad M. Heidari\sym{*} \qquad M. Mohseni\sym{*} }\\
{\sym{*} \footnotesize  Tehran Institute for Advanced Studies, Khatam University, Tehran, Iran}
}

\begin{document}
\maketitle
\begin{namelist}{xxxxxxxxxxxx}
	\item[{\bf Title:}]
	\textit{Connected Stocks via Business Groups: Evidence from an Emerging Market }
	\item[{\bf Author:}]
	\textit{ Seyyed Morteza Aghajanzadeh}
	\item[{\bf Supervisors:}]
	\textit{	Dr. Mahdi Heidari, Dr. Mahdi Mohseni}
	\item[{\bf Institution:}]
	\textit{	Tehran Institute for Advanced Studies}
\end{namelist}


\section*{Research Objective}
Related literature points out that common ownership and business groups are non-fundamental factors that lead to co-movement in stock returns. Using unique Iran's financial market context, this paper attempts to find which factors intensively and extensively affect co-movement. 
\section*{Motivation}
	{
	It is well established in the literature that socks comove in many dimensions. While first coming  investigations   attributed the companies co-movement to their fundamentals, (e.g. {\cite{shiller1989comovements}})}, recent findings have focused on the role of non-fundamental characteristics. {\cite{barberis2003style} and \cite{barberis2005comovement}} provided theoretical models for predicting the co-movement between fundamentally unrelated companies.	Trying to explain factors affecting co-movement, \cite{AntonPolk} suggests that common ownership positively affects co-movement\footnote{There are some factors like, Index inclusion ({\cite{barberis2005comovement}}), investors' attention to the companies ({\cite{wu2014investor}}), Investment banks' underwriting ({\cite{grullon2014comovement}}), correlated beliefs ({\cite{david2016correlated}}), shareholders' coordination ({\cite{pantzalis2017shareholder}}), and preference for companies' dividends ({\cite{HAMEED2019103}}) that have been identified by researchers.}.	Subsequently, {\cite{Liquidity2016}} provides evidence that even owners' liquidity needs' correlation can result in co-movement independent of direct common ownership.	
	
%	Measures of common ownership have been categorized into two types in the literature. First of all, model-based measures that capture common ownership base on a proper  model. These measures have a better economic interpretation, but most of them are bi-directional or industry-level measures.(e.g, \cite{harford2011institutional}; \cite{azar2018anticompetitive}; \cite{gilje2020s}) In addition to model-based measures, some ad hoc common ownership measures are used in the empirical literature. There is significant doubt on how these measures capture common ownership's impact on the management, and many of them have unappealing properties.(e.g, \cite{AntonPolk}; \cite{azar2011new}; \cite{freeman2019effects}; \cite{hansen1996externalities};  \cite{he2017product}; \cite{he2019internalizing}; \cite{lewellen2021does}; \cite{newham2018common})		
	
	
	
	
Despite the findings in recent financial literature regarding common ownership, business groups have not been considered as matters of common ownership. Business groups are everywhere in emerging markets (e.g., Brazil, Chile, China, India, Indonesia, South Korea, and many more) and even in some developed economies (e.g., Italy, Sweden), and there are debates about the pros and cons of them (\cite{khanna2000group}, \cite{khanna2007business}, \cite{Johnson2000}, \cite{Bertrand2002}). Studies have found co-movement among stocks of business groups, but the explanation for co-movement is controversial.
	
Even though there have been investigations on the effects of common ownership, they have been primarily focused on fund ownership. This type of owner performs particular behavior due to their needs, and little is known about other ownership types. An extensive empirical literature considers the role of block-holders in firm governance. A long literature surveyed by \cite{holderness2003survey}, \cite{edmans2014blockholders}, and \cite{edmans2017blockholders} considers the potential role of block-holders in firm governance. Following \cite{AntonPolk}, we are the first study that uses block-holder ownership to investigate the relationship between common ownership and co-movement.
			
			
			
%As an emerging literature, the pros and cons of business groups have been the subject of the debates [\cite{khanna2000group}]. 
% While the co-movement in business groups is accepted, the co-movement channels remain undiscovered.
%Both {\cite{cho2015stock} and \cite{kim2015stock}} studied the South Korean market and suggested two different sources for the co-movement in business groups. The first paper attributed co-movement to the companies' fundamentals. However, the second paper presents that the investors' category/habitat behavior is responsible for co-movement.
		


\section*{Data}
We use a unique data set that includes the daily report of the block-holder's ownership, defined as a shareholder who owns at least 1\% of the total outstanding shares. The set of variables contains firms' characteristics like market cap and book value, detailed information on daily trade like volume and return, and members of business groups. The time period of the study is from 2015 to 2020.

\section*{Methodology}

A method wildly used in Empirical asset pricing is the two-step approach of \cite{FamaMacBeth}. In the first step, cross-sectional regressions
are used to obtain estimates of the parameters of interest for each period. Then, in the second step, the time series of these estimates are used to get final estimates for the parameters and standard errors so that t-statistics can be computed (\cite{skoulakis2008panel}). Furthermore, We use the same methodology as  \cite{AntonPolk} to compose pairs, define control variables, and calculate co-movement. 


\section*{Contribution}

Recent studies shed light on the role of direct and indirect common ownership of mutual funds on co-movement. Using Iran's unique setting, we are trying to clarify the role of block-holder ownership on co-movement and compare its effect with the business groups. Additionally, we propose a modification for the measurement of common ownership in \cite{AntonPolk}.



	











%%%%%%%%%%%%%%%%%%%%%%%%%%%%%%%%%%
%%%%%%%%%%%%%%%%%%%%%%%%%%%%%

	


	
\newpage
	{
	\footnotesize
	\bibliographystyle{apalike}
	\bibliography{../Ref}
}

\end{document}