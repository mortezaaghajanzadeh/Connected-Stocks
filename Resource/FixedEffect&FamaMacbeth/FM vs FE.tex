\documentclass{beamer}
\usetheme{Madrid}
\usepackage{comment}
\usepackage{ragged2e}
\usepackage{amsmath}
\setbeamertemplate{caption}[numbered]

\title[FamaMacbeth regression vs. Fixed effect]{FamaMacbeth regression vs. Fixed effect}
%\subtitle{}
\author[S.M.Aghajanzadeh]{Presentation:  \\ Seyyed Morteza Aghajanzadeh}
\institute[TeIAS]{Tehran Institute for Advanced Studies \\ TeIAS}
\centering
\AtBeginSection[]
{
    \begin{frame}
        \frametitle{Table of Contents}
        \tableofcontents[currentsection]
    \end{frame}
}	
\begin{document}
\maketitle

\begin{frame}{Intro}
\begin{itemize}
\item  Two general forms of dependence
\begin{itemize}
\item {Firm Effect} 
\begin{itemize}
\item The residuals of a regression can be cross sectionally
correlated\\
 (e.g. the observations of a firm in different years are correlated)
\end{itemize}

\item {Time Effect}
\begin{itemize}
\item  The residuals of a given year may be correlated across firms.
\\
 (e.g. the observations of a year in different firms are correlated)
\end{itemize}


\end{itemize}
\item  An important statistical issue is that firm returns are cross-sectionall correlated: $ \text{Cov}(r_{it},r_{jt}) $ is far from zero.
\end{itemize}
\end{frame}


\begin{frame}{Fama Macbeth (1973)}
\begin{itemize}

\item Two Step Regression
\begin{itemize}
 \item First Step
 \begin{equation*}
\begin{array}{c}
Y_{i1} = \delta_{0,1} + \delta_{1,1}^1 X^1_{i,1} + \dots  + \delta_{k,1}^k X^k_{i,1}  + \varepsilon_{i,1}\\
\vdots\\
Y_{iT} = \delta_{0,1} + \delta_{1,T}^1 X^1_{i,T} + \dots  + \delta_{k,T}^k X^k_{i,T}  + \varepsilon_{i,T}
\end{array}
\end{equation*}



\item Second Step
\begin{equation*}
\left[\begin{matrix}
\bar{Y_1}\\
\vdots\\
\bar{Y_T}
\end{matrix}\right]_{T\times 1} =
 \left[\begin{matrix}
1 &  \delta_1^0 &  \delta_1^1 & \dots  &  \delta_1^k\\
\vdots&  \vdots &  \vdots &  \dots &  \vdots\\
1 &  \delta_T^0 &  \delta_T^1 & \dots &  \delta_T^k
\end{matrix}\right]_{T\times (k+2)}
\times \left[\begin{matrix}
\lambda\\
\lambda_0\\
\lambda_1\\
\vdots\\
\lambda_k
\end{matrix}\right] _{(k+2)\times 1}
\end{equation*}


\end{itemize}
\item Fama-MacBeth technique was
developed to account for correlation between observations on different firms in the same year
\end{itemize}
\end{frame}

\begin{frame}{Fama Macbeth (1973)}
\begin{itemize}
\item Fama-MacBeth approach was designed to deal with time effects in a panel data set,
not firm effects.

\item The firm effect may be less important in regressions where the dependent variable is returns
(and excess returns are serially uncorrelated) than in corporate finance applications where
unobserved firm effects can be very important

\item FM method is used for the estimation of factor risk premia in the analysis of linear factor models
\end{itemize}
\end{frame}

\begin{frame}{Newey and West (1987)}
\begin{itemize}
\item Newey and West (1987) adjustment to the results of the
regression, however, produces a new standard error for the estimated mean that is
adjusted for autocorrelation and heteroscedasticity.
\begin{itemize}
\item Only input is the number of lags to use when performing the adjustment
\begin{equation*}
Lag = 4(T/100)^{\frac{2}{9}}
\end{equation*}
where T is the number of periods in the time series
\end{itemize}
\end{itemize}
\end{frame}

\begin{frame}{Fixed Effect}
\begin{equation*}
y_{i,t} = \alpha + \beta x_{it} + \varepsilon_{it} \rightarrow \left\{
\begin{array}{lr}
y_{i,t} = \alpha + \beta x_{it} + (a_i + u_{it}) & \text{ Firm Fixed effect}\\\\
y_{i,t} = \alpha + \beta x_{it} + (a_t + \nu_{it}) & \text{ Time Fixed effect}
\end{array}
\right.
\end{equation*}

\begin{itemize}
\item Assumptions about unobserved terms: \textbf{strict exogeneity}
\begin{enumerate}
\item Firm Fixed effect:
$ E(x_{it}u_{is}) = 0 \text{ for } s = 1, 2, \dots , T $
\item Time Fixed effect:
$ E(x_{it}\nu_{st}) = 0 \text{ for } s = 1, 2, \dots , n $
\end{enumerate}

\end{itemize}
\end{frame}

\begin{frame}{Conclusion}
\begin{itemize}
\item Both methods rely on zero correlation between the error terms of non-contemporaneous periods. A difference is weighting: 
\begin{itemize}
\item The Fama-Macbeth procedure weights each time period equally.
\item A panel regression will effectively give greater weight to periods with more observations or greater variation in right hand side variables
\end{itemize}
\item The econometric analysis of panel data depends in a crucial way on the cross-sectional and timeseries correlation of the regression residuals
\end{itemize}
\end{frame}
\begin{frame}{Estimation Results }{By different methods}
\begin{table}[htbp]
\centering
    \resizebox{\textheight}{!}{
{
\def\sym#1{\ifmmode^{#1}\else\(^{#1}\)\fi}
\begin{tabular}{l*{5}{c}}
\hline\hline
                    &\multicolumn{1}{c}{(1)}&\multicolumn{1}{c}{(2)}&\multicolumn{1}{c}{(3)}&\multicolumn{1}{c}{(4)}&\multicolumn{1}{c}{(5)}\\
                    &\multicolumn{1}{c}{Lag(5)}&\multicolumn{1}{c}{Lag(4)}&\multicolumn{1}{c}{TimeFE}&\multicolumn{1}{c}{PairFE}&\multicolumn{1}{c}{Time Cluster}\\
\hline
$ \text{FCA}^* $    &     0.00296\sym{***}&     0.00296\sym{***}&     0.00351\sym{***}&    -0.00330\sym{**} &     0.00347\sym{***}\\
                    &      (6.35)         &      (6.48)         &      (8.93)         &     (-2.71)         &      (7.42)         \\
[1em]
 $ { \text{FCA} ^ * } ^2 $&     0.00480\sym{***}&     0.00480\sym{***}&     0.00542\sym{***}&    0.000901         &     0.00538\sym{***}\\
                    &     (11.15)         &     (10.64)         &     (12.49)         &      (0.93)         &     (10.27)         \\
[1em]
$ \rho\_t $          &       0.303\sym{***}&       0.303\sym{***}&       0.275\sym{***}&       0.253\sym{***}&       0.275\sym{***}\\
                    &     (18.05)         &     (16.20)         &    (337.99)         &    (306.57)         &     (10.95)         \\
[1em]
ActiveHolder        &     0.00356\sym{***}&     0.00356\sym{***}&     0.00438\sym{***}&     0.00724\sym{*}  &     0.00466\sym{***}\\
                    &      (3.81)         &      (3.76)         &      (4.56)         &      (2.40)         &      (3.99)         \\
[1em]
SameHolderType      &    -0.00569         &    -0.00569         &    -0.00227         &           0         &    -0.00170         \\
                    &     (-0.94)         &     (-0.93)         &     (-0.42)         &         (.)         &     (-0.31)         \\
[1em]
SameGroup           &      0.0167\sym{***}&      0.0167\sym{***}&      0.0171\sym{***}&           0         &      0.0173\sym{***}\\
                    &      (7.24)         &      (7.32)         &     (14.41)         &         (.)         &      (8.45)         \\
[1em]
Constant            &      0.0372\sym{**} &      0.0372\sym{**} &      0.0306\sym{***}&      0.0197\sym{*}  &      0.0297\sym{***}\\
                    &      (3.06)         &      (3.18)         &      (5.05)         &      (2.56)         &      (4.03)         \\
\hline
Observations        &     1381437         &     1381437         &     1381437         &     1381437         &     1381437         \\
\hline\hline
\multicolumn{6}{l}{\footnotesize \textit{t} statistics in parentheses}\\
\multicolumn{6}{l}{\footnotesize \sym{*} \(p<0.05\), \sym{**} \(p<0.01\), \sym{***} \(p<0.001\)}\\
\end{tabular}
}

}
\end{table}
\end{frame}

\end{document}