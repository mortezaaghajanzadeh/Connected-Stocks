\documentclass[12pt, a4paper]{article}
\usepackage{comment}
\usepackage{ragged2e}
\usepackage{amsmath}
\usepackage{xcolor}
\usepackage{multirow}
\usepackage{caption}
\usepackage{tikz}
\usepackage{booktabs}
\usepackage{tabu}
\usepackage{placeins}
\usepackage{pdflscape}
\usetikzlibrary{arrows}
\usepackage{hyperref}
\usepackage{multirow}
\usepackage{subcaption}


\captionsetup{font=footnotesize,labelfont=footnotesize}

\hypersetup{
	colorlinks=true,
	linkcolor=blue,
	filecolor=blue,      
	urlcolor=blue,
	citecolor=blue
}

\usepackage{natbib}
\usepackage[title]{appendix}



\def\sym#1{\ifmmode^{#1}\else\(^{#1}\)\fi}


\renewcommand{\today}{\ifcase \month \or January\or February\or March\or %
	April\or May \or June\or July\or August\or September\or October\or November\or %
	December\fi, \number \year} 

\title{Connected Stocks: Evidence from Tehran Stock Exchange}
%\subtitle{}
\author{S.M. Aghajanzadeh\sym{*} \qquad M. Heidari\sym{*} \qquad M. Mohseni\sym{*} \\
	\sym{*} \footnotesize  Tehran Institute for Advanced Studies, Khatam University, Tehran, Iran
}

\def\boxit#1{%
	\smash{\color{red}\fboxrule=1pt\relax\fboxsep=2pt\relax%
		\llap{\rlap{\fbox{\vphantom{0}\makebox[#1]{}}}~}}\ignorespaces
}
\begin{document}
	\maketitle

\textbf{Hypothesis 1:} Simple measures of institutional connnectedness statistically and economically improve forecasts of cross-sectional variation in the correlation. The effect is stronger when pairs are in the same business groups

		\begin{table}[htbp]
	\centering
	\resizebox{\textwidth}{!}{
		{
\def\sym#1{\ifmmode^{#1}\else\(^{#1}\)\fi}
\begin{tabular}{l*{7}{c}}
\hline\hline
                &\multicolumn{7}{c}{Dependent Variable: Future Monthly Correlation of 4F+Industry Residuals}                                         \\\cmidrule(lr){2-8}
                &\multicolumn{1}{c}{(1)}         &\multicolumn{1}{c}{(2)}         &\multicolumn{1}{c}{(3)}         &\multicolumn{1}{c}{(4)}         &\multicolumn{1}{c}{(5)}         &\multicolumn{1}{c}{(6)}         &\multicolumn{1}{c}{(7)}         \\
\hline
Same Group      &   0.0138\sym{***}&   0.0128\sym{***}&                  &                  &  0.00978\sym{***}&  0.00458         &  0.00356         \\
                &   (5.76)         &   (6.29)         &                  &                  &   (4.29)         &   (1.43)         &   (1.11)         \\
[1em]
$ \text{FCA*} $ &                  &                  &  0.00405\sym{***}&  0.00375\sym{***}&  0.00296\sym{***}&  0.00258\sym{***}&  0.00273\sym{***}\\
                &                  &                  &   (4.94)         &   (5.12)         &   (3.77)         &   (3.53)         &   (3.51)         \\
[1em]
 $ (\text{FCA}^*) \times {\text{SameGroup} }  $ &                  &                  &                  &                  &                  &  0.00524\sym{**} &  0.00517\sym{**} \\
                &                  &                  &                  &                  &                  &   (3.21)         &   (3.18)         \\
\hline
Observations    &   388492         &   388492         &   388492         &   388492         &   388492         &   388492         &   388492         \\
Group Effect    &       No         &       No         &       No         &       No         &       No         &       No         &      Yes         \\
Controls        &       No         &      Yes         &       No         &      Yes         &      Yes         &      Yes         &      Yes         \\
$ R^2 $         & 0.000404         &  0.00200         & 0.000423         &  0.00201         &  0.00229         &  0.00245         &  0.00875         \\
\hline\hline
\multicolumn{8}{l}{\footnotesize \textit{t} statistics in parentheses}\\
\multicolumn{8}{l}{\footnotesize \sym{*} \(p<0.05\), \sym{**} \(p<0.01\), \sym{***} \(p<0.001\)}\\
\end{tabular}
}

	}
\end{table}

	\FloatBarrier
	\newpage
	
\textbf{Hypothesis 2:} Pairs of companies belonging to the same business group have a higher correlation than pairs not in the same group. In addition, Pairs that belong to the same group and have a common ownership co-move more than pairs that don't have common ownership. 
		\begin{table}[htbp]
	\centering
	\caption{one of these tables}
	\resizebox{\textwidth}{!}{
		{
\def\sym#1{\ifmmode^{#1}\else\(^{#1}\)\fi}
\begin{tabular}{l*{6}{c}}
\hline\hline
                &\multicolumn{6}{c}{Future Monthly Correlation of 4F+Industry Residuals}                                          \\\cmidrule(lr){2-7}
                &\multicolumn{1}{c}{(1)}         &\multicolumn{1}{c}{(2)}         &\multicolumn{1}{c}{(3)}         &\multicolumn{1}{c}{(4)}         &\multicolumn{1}{c}{(5)}         &\multicolumn{1}{c}{(6)}         \\
\hline
 $ (\text{FCA} > Median[\text{FCA}]) $ &                  & -0.00168         & -0.00337\sym{**} &  0.00855\sym{**} &                  & -0.00513\sym{***}\\
                &                  &  (-1.45)         &  (-2.89)         &   (2.76)         &                  &  (-4.32)         \\
[1em]
SameGroup       &   0.0122\sym{***}&                  &   0.0135\sym{***}&                  &                  &  0.00574\sym{*}  \\
                &   (5.81)         &                  &   (6.48)         &                  &                  &   (2.02)         \\
[1em]
 $ (\text{FCA} > Median[\text{FCA}]) \times  {\text{SameGroup} }  $ &                  &                  &                  &                  &                  &   0.0181\sym{***}\\
                &                  &                  &                  &                  &                  &   (5.91)         \\
[1em]
$ \text{FCA*} $ &                  &                  &                  &                  &  0.00174\sym{*}  &                  \\
                &                  &                  &                  &                  &   (2.43)         &                  \\
\hline
Observations    &  5148109         &  5148109         &  5148109         &    76240         &    76240         &  5148109         \\
Sub Sample      &    Total         &    Total         &    Total         &SameGroups         &SameGroups         &    Total         \\
Controls        &      Yes         &      Yes         &      Yes         &      Yes         &      Yes         &      Yes         \\
$ R^2 $         & 0.000455         & 0.000439         & 0.000485         &   0.0136         &   0.0135         & 0.000513         \\
\hline\hline
\multicolumn{7}{l}{\footnotesize \textit{t} statistics in parentheses}\\
\multicolumn{7}{l}{\footnotesize \sym{*} \(p<0.05\), \sym{**} \(p<0.01\), \sym{***} \(p<0.001\)}\\
\end{tabular}
}

	}
\newline
\resizebox{\textwidth}{!}{
	{
\def\sym#1{\ifmmode^{#1}\else\(^{#1}\)\fi}
\begin{tabular}{l*{6}{c}}
\hline\hline
                &\multicolumn{6}{c}{Future Monthly Correlation of 4F+Industry Residuals}                                          \\\cmidrule(lr){2-7}
                &\multicolumn{1}{c}{(1)}         &\multicolumn{1}{c}{(2)}         &\multicolumn{1}{c}{(3)}         &\multicolumn{1}{c}{(4)}         &\multicolumn{1}{c}{(5)}         &\multicolumn{1}{c}{(6)}         \\
\hline
 Common Ownership &                  & -0.00350\sym{**} & -0.00445\sym{***}&  0.00651\sym{*}  &                  & -0.00527\sym{***}\\
                &                  &  (-3.30)         &  (-4.22)         &   (2.48)         &                  &  (-4.72)         \\
[1em]
SameGroup       &   0.0122\sym{***}&                  &   0.0140\sym{***}&                  &                  &  0.00607\sym{*}  \\
                &   (5.81)         &                  &   (7.01)         &                  &                  &   (2.09)         \\
[1em]
 $ \text{Common Ownership} \times  {\text{SameGroup} }  $ &                  &                  &                  &                  &                  &   0.0157\sym{***}\\
                &                  &                  &                  &                  &                  &   (5.51)         \\
[1em]
$ \text{FCA*} $ &                  &                  &                  &                  &  0.00174\sym{*}  &                  \\
                &                  &                  &                  &                  &   (2.43)         &                  \\
\hline
Observations    &  5148109         &  5148109         &  5148109         &    76240         &    76240         &  5148109         \\
Sub Sample      &    Total         &    Total         &    Total         &SameGroups         &SameGroups         &    Total         \\
Controls        &      Yes         &      Yes         &      Yes         &      Yes         &      Yes         &      Yes         \\
$ R^2 $         & 0.000455         & 0.000456         & 0.000504         &   0.0135         &   0.0135         & 0.000528         \\
\hline\hline
\multicolumn{7}{l}{\footnotesize \textit{t} statistics in parentheses}\\
\multicolumn{7}{l}{\footnotesize \sym{*} \(p<0.05\), \sym{**} \(p<0.01\), \sym{***} \(p<0.001\)}\\
\end{tabular}
}

}
\end{table}

\newpage
\textbf{Hypothesis 3:} Return of business group improve forecasts of cross-sectional variation in stocks' return.

			\begin{table}[htbp]
		\centering
		\resizebox{\textwidth}{!}{
			{
\def\sym#1{\ifmmode^{#1}\else\(^{#1}\)\fi}
\begin{tabular}{l*{5}{c}}
\hline\hline
                &\multicolumn{5}{c}{ $ \text{Return}\_i - r\_f = R\_i$ }                                          \\\cmidrule(lr){2-6}
                &\multicolumn{1}{c}{(1)}         &\multicolumn{1}{c}{(2)}         &\multicolumn{1}{c}{(3)}         &\multicolumn{1}{c}{(4)}         &\multicolumn{1}{c}{(5)}         \\
\hline
 $ R\_M $        &    0.898\sym{***}&    0.665\sym{***}&    0.728\sym{***}&    0.363\sym{***}&    0.416\sym{***}\\
                &  (39.40)         &  (19.93)         &  (20.74)         &  (12.76)         &  (11.15)         \\
[1em]
 $ R\_{Industry} $ &                  &    0.302\sym{***}&    0.290\sym{***}&    0.184\sym{***}&    0.168\sym{***}\\
                &                  &   (5.88)         &   (5.43)         &   (6.43)         &   (5.64)         \\
[1em]
 $ R\_{Business group} $ &                  &                  &                  &    0.447\sym{***}&    0.453\sym{***}\\
                &                  &                  &                  &  (13.88)         &  (13.91)         \\
[1em]
 $ SMB $        &                  &                  &    0.227\sym{***}&                  &    0.157\sym{***}\\
                &                  &                  &   (8.57)         &                  &   (5.93)         \\
[1em]
 $ UMD $        &                  &                  &   0.0315\sym{*}  &                  & -0.00268         \\
                &                  &                  &   (2.38)         &                  &  (-0.16)         \\
[1em]
 $ HML $        &                  &                  &   0.0106         &                  & 0.000198         \\
                &                  &                  &   (0.53)         &                  &   (0.01)         \\
[1em]
Constant        &    0.204\sym{***}&    0.132\sym{***}&   0.0758\sym{***}&   0.0576\sym{***}&  0.00885         \\
                &   (7.71)         &   (7.27)         &   (4.16)         &   (6.64)         &   (0.67)         \\
\hline
Observations    &   351728         &   351728         &   351728         &   351728         &   351728         \\
\(R^{2}\)       &    0.140         &    0.224         &    0.251         &    0.684         &    0.695         \\
\hline\hline
\multicolumn{6}{l}{\footnotesize \textit{t} statistics in parentheses}\\
\multicolumn{6}{l}{\footnotesize \sym{*} \(p<0.05\), \sym{**} \(p<0.01\), \sym{***} \(p<0.001\)}\\
\end{tabular}
}

		}
	\end{table}


\newpage
\textbf{Channels:} 

\begin{table}[htbp]
	\centering
	\resizebox{\textwidth}{!}{
		{
\def\sym#1{\ifmmode^{#1}\else\(^{#1}\)\fi}
\begin{tabular}{l*{6}{c}}
\hline\hline
                    &\multicolumn{6}{c}{Dependent Variable:  Future Pairs's Comovement}                                                                 \\\cmidrule(lr){2-7}
                    &\multicolumn{1}{c}{(1)}         &\multicolumn{1}{c}{(2)}         &\multicolumn{1}{c}{(3)}         &\multicolumn{1}{c}{(4)}         &\multicolumn{1}{c}{(5)}         &\multicolumn{1}{c}{(6)}         \\
\hline
SameGroup           &      0.0208\sym{***}&      0.0206\sym{***}&                     &                     &     0.00619         &     0.00630\sym{*}  \\
                    &      (7.91)         &      (7.94)         &                     &                     &      (1.95)         &      (2.04)         \\
[1em]
LowImbalanceStd     &                     &    -0.00144         &      0.0282\sym{***}&    -0.00724\sym{***}&    -0.00610\sym{***}&    -0.00267         \\
                    &                     &     (-1.15)         &      (6.06)         &     (-5.74)         &     (-4.87)         &     (-1.85)         \\
[1em]
 $ \text{LowImbalanceStd} \times {\text{SameGroup} } $ &                     &                     &                     &                     &      0.0358\sym{***}&      0.0325\sym{***}\\
                    &                     &                     &                     &                     &      (8.57)         &      (7.48)         \\
\hline
Sub-sample          &       Total         &       Total         &   SameGroup         &      Others         &       Total         &       Total         \\
Business Group FE   &          No         &          No         &          No         &          No         &          No         &         Yes         \\
Observations        &      354209         &      354209         &       43274         &      310935         &      354209         &      354209         \\
\hline\hline
\multicolumn{7}{l}{\footnotesize \textit{t} statistics in parentheses}\\
\multicolumn{7}{l}{\footnotesize \sym{*} \(p<0.05\), \sym{**} \(p<0.01\), \sym{***} \(p<0.001\)}\\
\end{tabular}
}

	}
\end{table}



	\begin{table}[htbp]
	\centering
	\caption{Estimate regression for
		each stock across trading days  }
	\resizebox{0.6\textheight}{!}{
		{
\def\sym#1{\ifmmode^{#1}\else\(^{#1}\)\fi}
\begin{tabular}{l*{4}{c}}
\hline\hline
                    &\multicolumn{4}{c}{Dependent Variable: $\Delta \text{TurnOver}\_{i} $ }                 \\\cmidrule(lr){2-5}
                    &\multicolumn{1}{c}{(1)}         &\multicolumn{1}{c}{(2)}         &\multicolumn{1}{c}{(3)}         &\multicolumn{1}{c}{(4)}         \\
\hline
 $ \Delta \text{TurnOver}_{\text{Market}} $ &       0.416\sym{***}&       0.326\sym{***}&       0.252\sym{***}&       0.228\sym{***}\\
                    &     (12.25)         &      (5.35)         &      (6.41)         &      (4.24)         \\
[1em]
 $ \Delta \text{TurnOver}_{\text{Industry-i}} $ &       0.142\sym{***}&       0.213\sym{***}&      0.0335         &       0.167\sym{**} \\
                    &      (3.79)         &      (6.29)         &      (1.34)         &      (2.87)         \\
[1em]
 $ \Delta \text{TurnOver}_{\text{Group,-i}} $ &                     &                     &       0.330\sym{***}&       0.218\sym{***}\\
                    &                     &                     &     (12.74)         &      (3.80)         \\
\hline
Control             &          No         &         Yes         &          No         &         Yes         \\
Observations        &      854662         &      851772         &      333789         &      331263         \\
$ R^2 $             &       0.285         &       0.543         &       0.433         &       0.712         \\
\hline\hline
\multicolumn{5}{l}{\footnotesize \textit{t} statistics in parentheses}\\
\multicolumn{5}{l}{\footnotesize \sym{*} \(p<0.05\), \sym{**} \(p<0.01\), \sym{***} \(p<0.001\)}\\
\end{tabular}
}

	}
\end{table}


	\begin{table}[htbp]
	\centering
	\caption{Estimate regression for
		each stock across trading days  }
	\resizebox{0.7\textheight}{!}{
		\centering
		{
\def\sym#1{\ifmmode^{#1}\else\(^{#1}\)\fi}
\begin{tabular}{l*{6}{c}}
\hline\hline
                    &\multicolumn{6}{c}{Dependent Variable: $\Delta \text{Amihud}_{i} $ }                                                               \\\cmidrule(lr){2-7}
                    &\multicolumn{1}{c}{(1)}         &\multicolumn{1}{c}{(2)}         &\multicolumn{1}{c}{(3)}         &\multicolumn{1}{c}{(4)}         &\multicolumn{1}{c}{(5)}         &\multicolumn{1}{c}{(6)}         \\
\hline
 $ \Delta \text{Amihud}_{\text{Market}} $ &       0.324\sym{***}&       0.598\sym{*}  &       0.373\sym{***}&       0.327\sym{***}&       0.391\sym{***}&       0.346\sym{***}\\
                    &      (6.46)         &      (2.17)         &     (13.09)         &     (12.07)         &     (13.09)         &     (12.27)         \\
[1em]
 $ \Delta \text{Amihud}_{\text{Group}} $ &                     &                     &       0.165\sym{**} &       0.150\sym{*}  &       0.143\sym{*}  &       0.126\sym{*}  \\
                    &                     &                     &      (2.60)         &      (2.58)         &      (2.07)         &      (1.98)         \\
[1em]
 $ \Delta \text{Amihud}_{\text{Industry}} $ &      0.0567         &       0.118         &    -0.00390         &    -0.00278         &    -0.00322         &   0.0000345         \\
                    &      (1.21)         &      (1.58)         &     (-0.06)         &     (-0.04)         &     (-0.04)         &      (0.00)         \\
\hline
Observations        &      293264         &      291933         &      184699         &      183301         &      184699         &      183301         \\
Weight              &           -         &           -         & $ \text{MC} \times \text{CR} $          & $ \text{MC} \times \text{CR} $          & $ \text{MC} $          & $ \text{MC} $          \\
Control             &          No         &         Yes         &          No         &         Yes         &          No         &         Yes         \\
$ R^2 $             &      0.0976         &       0.149         &       0.194         &       0.235         &       0.199         &       0.239         \\
\hline\hline
\multicolumn{7}{l}{\footnotesize \textit{t} statistics in parentheses}\\
\multicolumn{7}{l}{\footnotesize \sym{*} \(p<0.05\), \sym{**} \(p<0.01\), \sym{***} \(p<0.001\)}\\
\end{tabular}
}

	}
\end{table}

\newpage
		\begin{figure}
	\centering  
	\includegraphics[width=\linewidth]{"FCAtimeSeries.eps"}
	
\end{figure} 

			\begin{figure}
	\centering  
	\includegraphics[width=\linewidth]{"FCAtimeSeriesBG.eps"}
\end{figure}    

\begin{figure}
	\centering  
	\includegraphics[width=\linewidth]{"FCAtimeSeriesPairType.eps"}
\end{figure}




	\begin{figure}
	\centering  
	\includegraphics[width=\linewidth]{"BGtimeSeries.eps"}
	
\end{figure}  

\begin{figure}
	\centering  
	\includegraphics[width=\linewidth]{"BGMarketCaptimeSeries.eps"}
	
\end{figure}
 
\begin{figure}
	\centering  
	\includegraphics[width=\linewidth]{"CorrtimeSeries.eps"}
\end{figure}    

\begin{figure}
	\centering  
	\includegraphics[width=\linewidth]{"BGCorrtimeSeries.eps"}
\end{figure}
\end{document}