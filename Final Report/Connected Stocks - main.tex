\documentclass[12pt, a4paper]{article}
\usepackage{comment}
\usepackage{ragged2e}
\usepackage{amsmath}
\usepackage{xcolor}
\usepackage{multirow}
\usepackage{caption}
\usepackage{tikz}
\usepackage{booktabs}
\usepackage{tabu}
\usepackage{placeins}
\usepackage{pdflscape}
\usetikzlibrary{arrows}
\usepackage{hyperref}
\usepackage{multirow}
\usepackage{subcaption}


\captionsetup{font=footnotesize,labelfont=footnotesize}

\hypersetup{
	colorlinks=true,
	linkcolor=blue,
	filecolor=blue,      
	urlcolor=blue,
	citecolor=blue
}

\usepackage{natbib}
\usepackage[title]{appendix}



\def\sym#1{\ifmmode^{#1}\else\(^{#1}\)\fi}


\renewcommand{\today}{\ifcase \month \or January\or February\or March\or %
	April\or May \or June\or July\or August\or September\or October\or November\or %
	December\fi, \number \year} 

\title{Connected Stocks: Evidence from Tehran Stock Exchange}
%\subtitle{}
\author{S.M. Aghajanzadeh\sym{*} \qquad M. Heidari\sym{*} \qquad M. Mohseni\sym{*} \\
	\sym{*} \footnotesize  Tehran Institute for Advanced Studies, Khatam University, Tehran, Iran
}

\def\boxit#1{%
	\smash{\color{red}\fboxrule=1pt\relax\fboxsep=2pt\relax%
		\llap{\rlap{\fbox{\vphantom{0}\makebox[#1]{}}}~}}\ignorespaces
}
\begin{document}
	\maketitle
\section*{Effects}
\subsection*{\textbf{Hypothesis 1}}
 Simple measures of institutional connnectedness statistically and economically improve forecasts of cross-sectional variation in the correlation. The effect is stronger when pairs are in the same business groups

		\begin{table}[htbp]
	\centering
	\resizebox{\textwidth}{!}{
		{
\def\sym#1{\ifmmode^{#1}\else\(^{#1}\)\fi}
\begin{tabular}{l*{9}{c}}
\hline\hline
                &\multicolumn{9}{c}{Dependent Variable: Future Monthly Correlation of 4F+Industry Residuals}                                                                               \\\cmidrule(lr){2-10}
                &\multicolumn{1}{c}{(1)}         &\multicolumn{1}{c}{(2)}         &\multicolumn{1}{c}{(3)}         &\multicolumn{1}{c}{(4)}         &\multicolumn{1}{c}{(5)}         &\multicolumn{1}{c}{(6)}         &\multicolumn{1}{c}{(7)}         &\multicolumn{1}{c}{(8)}         &\multicolumn{1}{c}{(9)}         \\
\hline
Same Group      &   0.0166\sym{***}&   0.0153\sym{***}&                  &                  &   0.0147\sym{***}&                  &                  &  0.00624\sym{***}&  0.00549\sym{**} \\
                &   (8.54)         &   (7.90)         &                  &                  &   (6.97)         &                  &                  &   (2.81)         &   (2.27)         \\
[1em]
$ \text{FCA*} $ &                  &                  &  0.00150\sym{***}&  0.00112\sym{**} & 0.000736         &  0.00944\sym{***}& 0.000397         & 0.000377         &-0.0000113         \\
                &                  &                  &   (2.90)         &   (2.11)         &   (1.33)         &   (7.24)         &   (0.68)         &   (0.65)         &  (-0.02)         \\
[1em]
 $ (\text{FCA}^*) \times {\text{SameGroup} }  $ &                  &                  &                  &                  &                  &                  &                  &  0.00992\sym{***}&   0.0107\sym{***}\\
                &                  &                  &                  &                  &                  &                  &                  &   (6.49)         &   (6.97)         \\
\hline
Observations    &  1665996         &  1665996         &  1665996         &  1665996         &  1665996         &    58337         &  1607659         &  1665996         &  1665996         \\
Sub-sample      &      All         &      All         &      All         &      All         &      All         &SameGroup         &   Others         &      All         &      All         \\
Group Effect    &       No         &       No         &       No         &       No         &       No         &       No         &       No         &       No         &      Yes         \\
Controls        &       No         &      Yes         &       No         &      Yes         &      Yes         &      Yes         &      Yes         &      Yes         &      Yes         \\
$ R^2 $         & 0.000180         & 0.000637         & 0.000170         & 0.000652         & 0.000804         &   0.0112         & 0.000577         & 0.000898         &  0.00575         \\
\hline\hline
\multicolumn{10}{l}{\footnotesize \textit{t} statistics in parentheses}\\
\multicolumn{10}{l}{\footnotesize \sym{*} \(p<0.10\), \sym{**} \(p<0.05\), \sym{***} \(p<0.01\)}\\
\end{tabular}
}

	}
\end{table}

	\FloatBarrier
	\newpage
	
\subsection*{\textbf{Hypothesis 2}}
 Pairs of companies belonging to the same business group have a higher correlation than pairs not in the same group. In addition, Pairs that belong to the same group and have a common ownership co-move more than pairs that don't have common ownership. 
		\begin{table}[htbp]
	\centering
	\caption{one of these tables}
	\resizebox{\textwidth}{!}{
		{
\def\sym#1{\ifmmode^{#1}\else\(^{#1}\)\fi}
\begin{tabular}{l*{6}{c}}
\hline\hline
                &\multicolumn{6}{c}{Future Monthly Correlation of 4F+Industry Residuals}                                          \\\cmidrule(lr){2-7}
                &\multicolumn{1}{c}{(1)}         &\multicolumn{1}{c}{(2)}         &\multicolumn{1}{c}{(3)}         &\multicolumn{1}{c}{(4)}         &\multicolumn{1}{c}{(5)}         &\multicolumn{1}{c}{(6)}         \\
\hline
 $ (\text{FCA} > Median[\text{FCA}]) $ &                  & -0.00168         & -0.00337\sym{**} &  0.00855\sym{**} &                  & -0.00513\sym{***}\\
                &                  &  (-1.45)         &  (-2.89)         &   (2.76)         &                  &  (-4.32)         \\
[1em]
SameGroup       &   0.0122\sym{***}&                  &   0.0135\sym{***}&                  &                  &  0.00574\sym{*}  \\
                &   (5.81)         &                  &   (6.48)         &                  &                  &   (2.02)         \\
[1em]
 $ (\text{FCA} > Median[\text{FCA}]) \times  {\text{SameGroup} }  $ &                  &                  &                  &                  &                  &   0.0181\sym{***}\\
                &                  &                  &                  &                  &                  &   (5.91)         \\
[1em]
$ \text{FCA*} $ &                  &                  &                  &                  &  0.00174\sym{*}  &                  \\
                &                  &                  &                  &                  &   (2.43)         &                  \\
\hline
Observations    &  5148109         &  5148109         &  5148109         &    76240         &    76240         &  5148109         \\
Sub Sample      &    Total         &    Total         &    Total         &SameGroups         &SameGroups         &    Total         \\
Controls        &      Yes         &      Yes         &      Yes         &      Yes         &      Yes         &      Yes         \\
$ R^2 $         & 0.000455         & 0.000439         & 0.000485         &   0.0136         &   0.0135         & 0.000513         \\
\hline\hline
\multicolumn{7}{l}{\footnotesize \textit{t} statistics in parentheses}\\
\multicolumn{7}{l}{\footnotesize \sym{*} \(p<0.05\), \sym{**} \(p<0.01\), \sym{***} \(p<0.001\)}\\
\end{tabular}
}

	}
\newline
\resizebox{\textwidth}{!}{
	{
\def\sym#1{\ifmmode^{#1}\else\(^{#1}\)\fi}
\begin{tabular}{l*{6}{c}}
\hline\hline
                &\multicolumn{6}{c}{Future Monthly Correlation of 4F+Industry Residuals}                                          \\\cmidrule(lr){2-7}
                &\multicolumn{1}{c}{(1)}         &\multicolumn{1}{c}{(2)}         &\multicolumn{1}{c}{(3)}         &\multicolumn{1}{c}{(4)}         &\multicolumn{1}{c}{(5)}         &\multicolumn{1}{c}{(6)}         \\
\hline
 Common Ownership &                  &  0.00108\sym{*}  & 0.000775         &  0.00293         &                  & 0.000662         \\
                &                  &   (2.54)         &   (1.85)         &   (1.04)         &                  &   (1.64)         \\
[1em]
SameGroup       &   0.0152\sym{***}&                  &   0.0151\sym{***}&                  &                  &   0.0129\sym{***}\\
                &   (9.39)         &                  &   (9.38)         &                  &                  &   (6.07)         \\
[1em]
 $ \text{Common Ownership} \times  {\text{SameGroup} }  $ &                  &                  &                  &                  &                  &  0.00413         \\
                &                  &                  &                  &                  &                  &   (1.46)         \\
[1em]
$ \text{FCA*} $ &                  &                  &                  &                  &  0.00211         &                  \\
                &                  &                  &                  &                  &   (1.85)         &                  \\
\hline
Observations    &  5851137         &  5851137         &  5851137         &   112696         &   112696         &  5851137         \\
Sub Sample      &    Total         &    Total         &    Total         &SameGroups         &SameGroups         &    Total         \\
Controls        &      Yes         &      Yes         &      Yes         &      Yes         &      Yes         &      Yes         \\
$ R^2 $         &  0.00125         &  0.00119         &  0.00129         &   0.0166         &   0.0167         &  0.00132         \\
\hline\hline
\multicolumn{7}{l}{\footnotesize \textit{t} statistics in parentheses}\\
\multicolumn{7}{l}{\footnotesize \sym{*} \(p<0.05\), \sym{**} \(p<0.01\), \sym{***} \(p<0.001\)}\\
\end{tabular}
}

}
\end{table}

\newpage
\subsection*{\textbf{Hypothesis 3}} 
Return of business group improve forecasts of cross-sectional variation in stocks' return.

			\begin{table}[htbp]
		\centering
		\resizebox{\textwidth}{!}{
			{
\def\sym#1{\ifmmode^{#1}\else\(^{#1}\)\fi}
\begin{tabular}{l*{5}{c}}
\hline\hline
                &\multicolumn{5}{c}{ $ \text{Return}_i - r_f = R_i$ }                                          \\\cmidrule(lr){2-6}
                &\multicolumn{1}{c}{(1)}         &\multicolumn{1}{c}{(2)}         &\multicolumn{1}{c}{(3)}         &\multicolumn{1}{c}{(4)}         &\multicolumn{1}{c}{(5)}         \\
\hline
 $ R_M $        &    0.801\sym{***}&    0.643\sym{***}&    0.701\sym{***}&    0.257\sym{***}&    0.280\sym{***}\\
                &  (29.99)         &  (10.68)         &  (11.05)         &   (8.84)         &   (9.02)         \\
[1em]
 $ R_{Industry} $ &                  &   -2.085         &   -1.878         &   -0.150         &   -0.148         \\
                &                  &  (-0.92)         &  (-0.93)         &  (-0.48)         &  (-0.50)         \\
[1em]
 $ R_{Business group} $ &                  &                  &                  &    0.493\sym{***}&    0.493\sym{***}\\
                &                  &                  &                  &  (11.36)         &  (11.34)         \\
[1em]
 $ SMB $        &                  &                  &    0.104\sym{***}&                  &   0.0770\sym{***}\\
                &                  &                  &   (3.52)         &                  &   (5.24)         \\
[1em]
 $ UMD $        &                  &                  &   0.0282         &                  &   0.0218         \\
                &                  &                  &   (1.23)         &                  &   (1.94)         \\
[1em]
 $ HML $        &                  &                  &    0.102\sym{***}&                  &   0.0395\sym{***}\\
                &                  &                  &   (6.05)         &                  &   (6.39)         \\
[1em]
Constant        &   0.0442         &   0.0145         &  -0.0297         &   0.0499\sym{***}&   0.0198         \\
                &   (1.92)         &   (0.53)         &  (-0.83)         &   (3.87)         &   (1.25)         \\
\hline
Observations    &   207552         &   207552         &   207552         &   207552         &   207552         \\
\(R^{2}\)       &    0.123         &    0.196         &    0.213         &    0.672         &    0.679         \\
\hline\hline
\multicolumn{6}{l}{\footnotesize \textit{t} statistics in parentheses}\\
\multicolumn{6}{l}{\footnotesize \sym{*} \(p<0.05\), \sym{**} \(p<0.01\), \sym{***} \(p<0.001\)}\\
\end{tabular}
}

		}
	\end{table}


\newpage

\section*{Channels} 
\subsection*{Trading}
 
For each firm, we calculate daily institutional imbalances, which is the net buying value of institutional investors relative to total trade on that day ($ \text{InsImb} = \frac{\text{Buy}_{\text{value}} - \text{Sell}_{\text{value}}}{\text{Buy}_{\text{value}} + \text{Sell}_{\text{value}}} $). 
We expect that institutional imbalances have a lower variation in groups due to the correlated tradings that the ultimate owner ordered to do. So, we calculate the monthly standard deviation of the group's imbalances and compare them. As we expected grouped standard error is  13.1\% lower with t-stat of 12.57 than ungrouped firms. 

\begin{table}[htbp]
	\centering
	\resizebox{0.75\textwidth}{!}{
\begin{tabular}{lcccccc}
	\hline\hline
	& \multicolumn{1}{l}{count} & \multicolumn{1}{l}{mean} &
	
	 \multicolumn{1}{l}{std} & \multicolumn{1}{l}{min} &
	  \multicolumn{1}{l}{median} & \multicolumn{1}{l}{max} \\
	  \midrule
	Ungrouped & 60    & 0.645 & 0.063 & 0.492 & 0.653 & 0.784 \\
	\addlinespace
	Grouped & 60    & 0.514 & 0.050 & 0.406 & 0.514 & 0.625 \\
	\hline\hline
\end{tabular}%	
}
	\label{tab:addlabel}%
\end{table}%



\begin{figure}[htbp]
	\centering
	\includegraphics[width=0.8\linewidth]{GroupedSTD}
	\label{fig:groupedstd}
\end{figure}

According to the main hypothesis, we need to compare comovement between pairs in groups with low standard error and other pairs.
 For this purpose, we define \textbf{Low Imbalance std} dummy for groups whose average standard errors are lower than half of the sample. 
So, this dummy is equal to one if at least one pair's firms belong to the low imbalance std business group.
\begin{table}[htbp]
	\centering
	\resizebox{\textwidth}{!}{
		{
\def\sym#1{\ifmmode^{#1}\else\(^{#1}\)\fi}
\begin{tabular}{l*{7}{c}}
\hline\hline
                &\multicolumn{7}{c}{Future Monthly Corr. of 4F+Ind. Residuals}                                                                       \\\cmidrule(lr){2-8}
                &\multicolumn{1}{c}{(1)}         &\multicolumn{1}{c}{(2)}         &\multicolumn{1}{c}{(3)}         &\multicolumn{1}{c}{(4)}         &\multicolumn{1}{c}{(5)}         &\multicolumn{1}{c}{(6)}         &\multicolumn{1}{c}{(7)}         \\
\hline
$ \text{FCA*} $ &  0.00296\sym{***}&  0.00277\sym{***}&  0.00275\sym{***}&                  &  0.00611\sym{**} &  0.00244\sym{**} &  0.00284\sym{**} \\
                &   (3.77)         &   (3.57)         &   (3.55)         &                  &   (3.21)         &   (3.14)         &   (3.40)         \\
[1em]
Same Group      &  0.00978\sym{***}&  0.00981\sym{***}&  0.00858\sym{**} &   0.0110\sym{***}&                  &  0.00861\sym{**} &  0.00826\sym{**} \\
                &   (4.29)         &   (4.35)         &   (3.37)         &   (4.73)         &                  &   (3.38)         &   (3.05)         \\
[1em]
Low Imbalance std&                  & -0.00364\sym{**} & -0.00388\sym{**} & -0.00446\sym{**} & -0.00725\sym{*}  & -0.00393\sym{**} & 0.000437         \\
                &                  &  (-2.81)         &  (-2.83)         &  (-3.24)         &  (-2.47)         &  (-2.87)         &   (0.21)         \\
[1em]
 $ \text{Low Imbalance std} \times {\text{SameGroup} } $ &                  &                  &  0.00301         &  0.00365         &                  & -0.00904         & -0.00990\sym{*}  \\
                &                  &                  &   (0.81)         &   (0.98)         &                  &  (-1.84)         &  (-2.02)         \\
[1em]
 $ \text{Low Imbalance std} \times {\text{SameGroup} } \times \text{FCA}^*  $ &                  &                  &                  &                  &                  &   0.0104\sym{***}&  0.00941\sym{***}\\
                &                  &                  &                  &                  &                  &   (3.87)         &   (3.53)         \\
\hline
Observations    &   388492         &   388492         &   388492         &   388492         &    37114         &   388492         &   388492         \\
Group Effect    &       No         &       No         &       No         &       No         &       No         &       No         &      Yes         \\
Sub-sample      &    Total         &    Total         &    Total         &    Total         &Same Groups         &    Total         &    Total         \\
Controls        &      Yes         &      Yes         &      Yes         &      Yes         &      Yes         &      Yes         &      Yes         \\
$ R^2 $         &  0.00229         &  0.00255         &  0.00274         &  0.00246         &   0.0199         &  0.00290         &  0.00906         \\
\hline\hline
\multicolumn{8}{l}{\footnotesize \textit{t} statistics in parentheses}\\
\multicolumn{8}{l}{\footnotesize \sym{*} \(p<0.05\), \sym{**} \(p<0.01\), \sym{***} \(p<0.001\)}\\
\end{tabular}
}

	}
\end{table}
\FloatBarrier

Furthermore, we should show that stocks in groups have a similar daily trading behavior. Accordingly, We estimate changes in firm's turnover on changes of turnover in market, industry, or business groups. we aggregate turnover by using weighted average of firms turnover in that day.
\begin{equation*}
	\Delta \text{TurnOver}_{i,t} = \ln(\frac{\text{TurnOver}_{i,t}}{\text{TurnOver}_{i,t-1}}) = 
	\ln({\frac{\text{volume}_{i,t}}{\text{MarketCap}_{i,t}}}) - \ln({\frac{\text{volume}_{i,t-1}}{\text{MarketCap}_{i,t-1}}})
\end{equation*}


	\begin{table}[htbp]
	\centering
	\caption{Estimate regression for
		each stock across trading days  }
	\resizebox{0.6\textheight}{!}{
		{
\def\sym#1{\ifmmode^{#1}\else\(^{#1}\)\fi}
\begin{tabular}{l*{4}{c}}
\hline\hline
                    &\multicolumn{4}{c}{Dependent Variable: $\Delta \text{TurnOver}_{i} $ }                 \\\cmidrule(lr){2-5}
                    &\multicolumn{1}{c}{(1)}         &\multicolumn{1}{c}{(2)}         &\multicolumn{1}{c}{(3)}         &\multicolumn{1}{c}{(4)}         \\
\hline
 $ \Delta \text{TurnOver}_{\text{Market}} $ &       0.416\sym{***}&       0.326\sym{***}&       0.252\sym{***}&       0.228\sym{***}\\
                    &     (12.25)         &      (5.35)         &      (6.41)         &      (4.24)         \\
[1em]
 $ \Delta \text{TurnOver}_{\text{Industry-i}} $ &       0.142\sym{***}&       0.213\sym{***}&      0.0335         &       0.167\sym{**} \\
                    &      (3.79)         &      (6.29)         &      (1.34)         &      (2.87)         \\
[1em]
 $ \Delta \text{TurnOver}_{\text{Group,-i}} $ &                     &                     &       0.330\sym{***}&       0.218\sym{***}\\
                    &                     &                     &     (12.74)         &      (3.80)         \\
\hline
Control             &          No         &         Yes         &          No         &         Yes         \\
Observations        &      854662         &      851772         &      333789         &      331263         \\
$ R^2 $             &       0.285         &       0.543         &       0.433         &       0.712         \\
\hline\hline
\multicolumn{5}{l}{\footnotesize \textit{t} statistics in parentheses}\\
\multicolumn{5}{l}{\footnotesize \sym{*} \(p<0.05\), \sym{**} \(p<0.01\), \sym{***} \(p<0.001\)}\\
\end{tabular}
}

	}
\end{table}


	\begin{table}[htbp]
	\centering
	\caption{Estimate regression for
		each stock across trading days  }
	\resizebox{0.7\textheight}{!}{
		\centering
		{
\def\sym#1{\ifmmode^{#1}\else\(^{#1}\)\fi}
\begin{tabular}{l*{6}{c}}
\hline\hline
                    &\multicolumn{6}{c}{Dependent Variable: $\Delta \text{Amihud}_{i} $ }                                                               \\\cmidrule(lr){2-7}
                    &\multicolumn{1}{c}{(1)}         &\multicolumn{1}{c}{(2)}         &\multicolumn{1}{c}{(3)}         &\multicolumn{1}{c}{(4)}         &\multicolumn{1}{c}{(5)}         &\multicolumn{1}{c}{(6)}         \\
\hline
 $ \Delta \text{Amihud}_{\text{Market}} $ &       0.324\sym{***}&       0.549\sym{*}  &       0.373\sym{***}&       0.343\sym{***}&       0.391\sym{***}&       0.361\sym{***}\\
                    &      (6.46)         &      (2.23)         &     (13.09)         &     (12.01)         &     (13.09)         &     (12.14)         \\
[1em]
 $ \Delta \text{Amihud}_{\text{Group}} $ &                     &                     &       0.165\sym{**} &       0.153\sym{*}  &       0.143\sym{*}  &       0.129\sym{*}  \\
                    &                     &                     &      (2.60)         &      (2.57)         &      (2.07)         &      (1.98)         \\
[1em]
 $ \Delta \text{Amihud}_{\text{Industry}} $ &      0.0567         &       0.121         &    -0.00390         &    -0.00670         &    -0.00322         &    -0.00430         \\
                    &      (1.21)         &      (1.36)         &     (-0.06)         &     (-0.10)         &     (-0.04)         &     (-0.06)         \\
\hline
Observations        &      293264         &      291933         &      184699         &      183301         &      184699         &      183301         \\
Weight              &           -         &           -         & $ \text{MC} \times \text{CR} $          & $ \text{MC} \times \text{CR} $          & $ \text{MC} $          & $ \text{MC} $          \\
Control             &          No         &         Yes         &          No         &         Yes         &          No         &         Yes         \\
$ R^2 $             &      0.0976         &       0.132         &       0.194         &       0.220         &       0.199         &       0.224         \\
\hline\hline
\multicolumn{7}{l}{\footnotesize \textit{t} statistics in parentheses}\\
\multicolumn{7}{l}{\footnotesize \sym{*} \(p<0.05\), \sym{**} \(p<0.01\), \sym{***} \(p<0.001\)}\\
\end{tabular}
}

	}
\end{table}

\newpage
		\begin{figure}
	\centering  
	\includegraphics[width=\linewidth]{"FCAtimeSeries.eps"}
	
\end{figure} 

			\begin{figure}
	\centering  
	\includegraphics[width=\linewidth]{"FCAtimeSeriesBG.eps"}
\end{figure}    

\begin{figure}
	\centering  
	\includegraphics[width=\linewidth]{"FCAtimeSeriesPairType.eps"}
\end{figure}




	\begin{figure}
	\centering  
	\includegraphics[width=\linewidth]{"BGtimeSeries.eps"}
	
\end{figure}  

\begin{figure}
	\centering  
	\includegraphics[width=\linewidth]{"BGMarketCaptimeSeries.eps"}
	
\end{figure}
 
\begin{figure}
	\centering  
	\includegraphics[width=\linewidth]{"CorrtimeSeries.eps"}
\end{figure}    

\begin{figure}
	\centering  
	\includegraphics[width=\linewidth]{"BGCorrtimeSeries.eps"}
\end{figure}
\end{document}