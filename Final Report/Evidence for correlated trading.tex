\section{\lr{Evidence for correlated trading} }

\begin{LTR}
	In the previous sections, we have provided evidence consistent with the hypothesis that the presence of firms in the business groups can raise firms' co-movement. Although we don't have definitive insight into the specific channel that business groups can promote commonality, our analysis provides a useful overview.
	We claim that this relationship exists because the business group is an important proxy for the likelihood that trading in these stocks will be correlated. To gain a clearer understanding of how the business group is able to generate co-movement in firms' return, we now refine our basic analysis to consider other proxy measures for business group trading.
	We employ two proxies for business group trading that are designed to capture different trading motivations: turnover and institutional imbalance. While the first could be due to buying or selling of business groups, the latter reflects buying.
	
	
\end{LTR}


%\begin{itemize}
%	\item 
%	به نظر می آید در شرکت های عضو گروه های کسب و کار به همراه یکدیگر معامله می شوند
%	\item
%	از ملاک های اندازه گیری معاملات برای این هدف استفاده کرده یم
%\end{itemize}


\subsection{\lr{Turnover}}

\begin{LTR}
	First of all, we should show that stocks in groups have a similar daily trading behavior. Accordingly, We use the turnover measure as a daily trading measures. For each firm we run time-series regressions of the firm's daily change in turnover, $ \Delta \text{TurnOver}_{i,t} $, on changes in market turnover,$ \Delta\text{TurnOver}_{Market,t}   $ , changes in the industry and business group portfolio's turnover,$ \Delta\text{TurnOver}_{Ind,t} $ and  $\Delta \text{TurnOver}_{Group,t} $ and  ,as well as control variables.
We compute the daily change of turnover by this definition $ \Delta \text{TurnOver}_{i,t} = \ln(\frac{\text{TurnOver}_{i,t}}{\text{TurnOver}_{i,t-1}}) $. 
We estimate the following regression for each stock across trading days in given year separately and cross-sectional averages of the estimated coefficients are reported, with t-statistics in parentheses :

	\begin{equation*}
	\begin{split}
		\Delta \text{Turnover}_{i,t} =  & \text{	}\alpha + \beta_{Market,t} \Delta \text{Turnover}_{Market,t}  
		+ \beta_{Ind,t} \Delta \text{Turnover}_{Ind,t} \\ & + \beta_{Group,t} \Delta \text{Turnover}_{Group,t} + \delta\text{Controls} + \varepsilon_{i,t}
	\end{split}
\end{equation*}
 We control for lead and lag changes in the two portfolio and market's measures and size of the firm. We estimate that model with \cite{FamaMacBeth} method and adjust its standard errors with \cite{newey1987hypothesis} for seven periods.  As shown in Table \ref{turnover}, firms' change in turnover comes from market reaction and group's change. (This result is robust to the different method of weighting for portfolios) This observation shows that firms in one group trade together in each day. 
\lr{\begin{table}[htbp]
	\centering
	\caption{cross-sectional average of the time-series coefficients for daily changes in turnover }
	\resizebox{0.7\textheight}{!}{
		{
\def\sym#1{\ifmmode^{#1}\else\(^{#1}\)\fi}
\begin{tabular}{l*{4}{c}}
\hline\hline
                    &\multicolumn{4}{c}{Dependent Variable: $\Delta \text{TurnOver}_{i} $ }                 \\\cmidrule(lr){2-5}
                    &\multicolumn{1}{c}{(1)}         &\multicolumn{1}{c}{(2)}         &\multicolumn{1}{c}{(3)}         &\multicolumn{1}{c}{(4)}         \\
\hline
 $ \Delta \text{TurnOver}_{\text{Market}} $ &       0.416\sym{***}&       0.326\sym{***}&       0.252\sym{***}&       0.228\sym{***}\\
                    &     (12.25)         &      (5.35)         &      (6.41)         &      (4.24)         \\
[1em]
 $ \Delta \text{TurnOver}_{\text{Industry-i}} $ &       0.142\sym{***}&       0.213\sym{***}&      0.0335         &       0.167\sym{**} \\
                    &      (3.79)         &      (6.29)         &      (1.34)         &      (2.87)         \\
[1em]
 $ \Delta \text{TurnOver}_{\text{Group,-i}} $ &                     &                     &       0.330\sym{***}&       0.218\sym{***}\\
                    &                     &                     &     (12.74)         &      (3.80)         \\
\hline
Control             &          No         &         Yes         &          No         &         Yes         \\
Observations        &      854662         &      851772         &      333789         &      331263         \\
$ R^2 $             &       0.285         &       0.543         &       0.433         &       0.712         \\
\hline\hline
\multicolumn{5}{l}{\footnotesize \textit{t} statistics in parentheses}\\
\multicolumn{5}{l}{\footnotesize \sym{*} \(p<0.05\), \sym{**} \(p<0.01\), \sym{***} \(p<0.001\)}\\
\end{tabular}
}

	} \label{turnover}
\end{table}}

Furthermore, we have to show that firms with higher level of group turnover, have a higher level of co-movement. So, For each month, we extract annual average level of firms' turnover and monthly turnover. We assume that the residual of the model belongs to the business groups. We expect that firms in the groups have a lower dispersion in their residuals than firms out of the groups. We calculate firms' residuals. Its summary stats is in table \ref{tab:ResidualTrunSummary}. As we expected, residuals in business groups have a lower dispersion than others.
We calculate standard error of monthly turn over residuals of firms in the business groups for each business group. Groups' standard errors description is shown in table \ref{tab:ResidualTrunStdSummary}. On average group's standard error is lower than ungrouped  ones. For finding relation between standard error of monthly turnover residuals, we define a dummy variable for groups in the low level of standard error which is lower than median. As shown in table \ref{Turnovercrosssection}, pairs in the business groups of low dispersion have a higher level of co-movement than other firms.  


\lr{			\begin{table}[htbp]
	\centering
	\caption{Frims' Monthly residuals' summary statistics}
	\label{tab:ResidualTrunSummary}
	\resizebox{0.8\textwidth}{!}{
		\begin{tabular}{lcccccccc}
\toprule
{} &  Firm $\times$ Month &   mean &    std &    min &    25\% &    50\% &    75\% &    max \\
\midrule
Ungrouped &                   8423 & -0.003 &  0.857 & -4.647 & -0.514 & -0.015 &  0.521 &  4.918 \\
Grouped   &                  18428 &  0.002 &  0.808 & -5.856 & -0.498 & -0.030 &  0.490 &  5.467 \\
\bottomrule
\end{tabular}

	}
\end{table}	
		\begin{table}[htbp]
			\centering
			\caption{Frims' Monthly residuals' standard erros' summary statistics}
			\label{tab:ResidualTrunStdSummary}
			\resizebox{0.8\textwidth}{!}{
				\begin{tabular}{lcccccccc}
\toprule
{} &  Group \$ \textbackslash times \$ Month &   mean &    std &    min &    25\% &    50\% &    75\% &    max \\
\midrule
Ungrouped &                      72 &  0.776 &  0.113 &  0.504 &  0.685 &  0.781 &  0.867 &  1.030 \\
Grouped   &                    2441 &  0.601 &  0.313 &  0.001 &  0.403 &  0.567 &  0.763 &  3.274 \\
\bottomrule
\end{tabular}

			}
		\end{table}
\begin{figure}[htbp]
	\centering
	\includegraphics[width=0.85\linewidth]{Output/GroupedResSTD.eps}
	\label{fig:GroupedResSTD}
\end{figure}
		\begin{table}[htbp]
			\centering
			\caption{text}
			\label{Turnovercrosssection}
			\resizebox{\textwidth}{!}{
				\centering
				{
\def\sym#1{\ifmmode^{#1}\else\(^{#1}\)\fi}
\begin{tabular}{l*{6}{c}}
\hline\hline
                &\multicolumn{6}{c}{Dependent Variable:  Future Pairs's Comovement}                                               \\\cmidrule(lr){2-7}
                &\multicolumn{1}{c}{(1)}         &\multicolumn{1}{c}{(2)}         &\multicolumn{1}{c}{(3)}         &\multicolumn{1}{c}{(4)}         &\multicolumn{1}{c}{(5)}         &\multicolumn{1}{c}{(6)}         \\
\hline
SameGroup       &   0.0229\sym{***}&   0.0241\sym{***}&                  &                  &   0.0141\sym{***}&   0.0114\sym{**} \\
                &   (7.20)         &   (8.00)         &                  &                  &   (3.60)         &   (2.93)         \\
[1em]
LowTurnoverStd  &                  &  0.00233\sym{**} &   0.0296\sym{***}&-0.000636         &-0.000473         &  0.00284         \\
                &                  &   (2.65)         &   (5.72)         &  (-0.60)         &  (-0.45)         &   (1.88)         \\
[1em]
$ {\text{LowTurnoverStd} } \times {\text{SameGroup} }  $ &                  &                  &                  &                  &   0.0279\sym{***}&   0.0260\sym{***}\\
                &                  &                  &                  &                  &   (4.78)         &   (4.77)         \\
\hline
Sub-sample      &    Total         &    Total         &SameGroup         &   Others         &    Total         &    Total         \\
Business Group FE&       No         &       No         &       No         &       No         &       No         &      Yes         \\
Observations    &   389591         &   389591         &    47076         &   342515         &   389591         &   389591         \\
\hline\hline  \end{tabular}}

			}			
		\end{table}}
\end{LTR}
%\begin{itemize}
%	\item 
%	از تغییرات 
%	\lr{turnover}
%	برای بررسی معامله هم زمان استفاده می کنیم
%	\item 
%	تعریف تغییرات انحراف معیار
%	\begin{equation}
%		\Delta \text{TurnOver} = \ln(\frac{\text{TurnOver}_{i,t}}{\text{TurnOver}_{i,t-1}}) = 
%		\ln({\frac{\text{volume}_{i,t}}{\text{MarketCap}_{i,t}}}) - \ln({\frac{\text{volume}_{i,t-1}}{\text{MarketCap}_{i,t-1}}})
%	\end{equation}
%	\item 
%	از شیوه مقاله 
%	\lr{\cite{Liquidity2016}}
%	برای تعریف استفاده کرده ایم
%\end{itemize}
%
%
%\begin{itemize}
%	\item
%	به منظور بررسی معامله هم زمان شرکت ها در گروه نیاز است تا رابطه تغییرات 
%	\lr{turnover}
%	را با میانگین تغییرات 
%	\lr{turnover}
%	در گروه بدست بیاوریم
%	\item 
%	مدل زیر را برآورد می کنیم
%	\begin{equation*}
	\begin{split}
		\Delta \text{Turnover}_{i,t} =  & \text{	}\alpha + \beta_{Market,t} \Delta \text{Turnover}_{Market,t}  
		+ \beta_{Ind,t} \Delta \text{Turnover}_{Ind,t} \\ & + \beta_{Group,t} \Delta \text{Turnover}_{Group,t} + \delta\text{Controls} + \varepsilon_{i,t}
	\end{split}
\end{equation*}
%	\item
%	انتظار داریم متوسط ضرایب برای تغییرات
%	\lr{turnover}
%	گروه معنا دار و مثبت باشد
%	\item
%	جدول 
%	\ref{turnover}
%	نتایج برآورد را نشان می دهد
%	\begin{LTR}
%		\lr{\begin{table}[htbp]
%				\centering
%				\caption{cross-sectional average of the time-series coefficients for daily changes in turnover }
%				\resizebox{1\textwidth}{!}{
%					{
\def\sym#1{\ifmmode^{#1}\else\(^{#1}\)\fi}
\begin{tabular}{l*{4}{c}}
\hline\hline
                    &\multicolumn{4}{c}{Dependent Variable: $\Delta \text{TurnOver}_{i} $ }                 \\\cmidrule(lr){2-5}
                    &\multicolumn{1}{c}{(1)}         &\multicolumn{1}{c}{(2)}         &\multicolumn{1}{c}{(3)}         &\multicolumn{1}{c}{(4)}         \\
\hline
 $ \Delta \text{TurnOver}_{\text{Market}} $ &       0.416\sym{***}&       0.326\sym{***}&       0.252\sym{***}&       0.228\sym{***}\\
                    &     (12.25)         &      (5.35)         &      (6.41)         &      (4.24)         \\
[1em]
 $ \Delta \text{TurnOver}_{\text{Industry-i}} $ &       0.142\sym{***}&       0.213\sym{***}&      0.0335         &       0.167\sym{**} \\
                    &      (3.79)         &      (6.29)         &      (1.34)         &      (2.87)         \\
[1em]
 $ \Delta \text{TurnOver}_{\text{Group,-i}} $ &                     &                     &       0.330\sym{***}&       0.218\sym{***}\\
                    &                     &                     &     (12.74)         &      (3.80)         \\
\hline
Control             &          No         &         Yes         &          No         &         Yes         \\
Observations        &      854662         &      851772         &      333789         &      331263         \\
$ R^2 $             &       0.285         &       0.543         &       0.433         &       0.712         \\
\hline\hline
\multicolumn{5}{l}{\footnotesize \textit{t} statistics in parentheses}\\
\multicolumn{5}{l}{\footnotesize \sym{*} \(p<0.05\), \sym{**} \(p<0.01\), \sym{***} \(p<0.001\)}\\
\end{tabular}
}

%				} \label{turnover}
%		\end{table}}
%	\end{LTR}
%	\item
%	از شیوه فاما مکبث برای برآورد این معادله استفاده شده است
%	\lr{\cite{FamaMacBeth}}
%	\item
%	علاوه بر شرایط بازار، گروه کسب و کار بیشترین تاثیر را بر روی تغییرات معاملات در گروه دارد
%	%	\item
%	%	در قدم بعدی سعی می کنیم نشان دهیم با افزایش همبستگی تغییرات 
%	%	\lr{turnover}
%	%	هم بستگی شرکت های درون جفت افزایش پیدا می کند
%	%	\item
%	%	در این راستا برای هر جفت پیدا شده هم بستگی تغییرات روزانه 
%	%	\lr{turnover}
%	%را محاسبه می کنیم
%	%
%	%\item
%	%در قدم اول رابطه هم بستگی تغییرات 
%	%	\lr{turnover}
%	%را  بر روی مالکیت مشترک و عضویت در یک گروه کسب و کار بررسی میکنیم
%	%\begin{itemize}
%	%	\item 
%	%	در مدل اصلی به جای پیش بینی هم حرکتی بازده آبنده از همبستگی تغییرات 
%	%		\lr{turnover}
%	%		استفاده می کنیم:
%	%		\input{Model14.tex}
%	%		
%	%
%	%	را بر روی متغیر های مورد نظر خودمان بررسی می کنیم
%	%	\item 
%	%	جدول
%	%	\ref{mresult2-turnover}
%	%	نتایج را نشان می دهد
%	%	\begin{LTR}
%		%		\lr{\begin{table}[htbp]
%				%				\centering
%				%				\caption{Pairwise correlation in turnover  }
%				%				\label{mresult2-turnover}
%				%				\resizebox{\textwidth}{!}{
%					%					\centering
%					%					{
\def\sym#1{\ifmmode^{#1}\else\(^{#1}\)\fi}
\begin{tabular}{l*{7}{c}}
\hline\hline
                &\multicolumn{7}{c}{Dependent Variable: Future Monthly Correlation of Delta turnover}                                                \\\cmidrule(lr){2-8}
                &\multicolumn{1}{c}{(1)}         &\multicolumn{1}{c}{(2)}         &\multicolumn{1}{c}{(3)}         &\multicolumn{1}{c}{(4)}         &\multicolumn{1}{c}{(5)}         &\multicolumn{1}{c}{(6)}         &\multicolumn{1}{c}{(7)}         \\
\hline
Same Group      &   0.0334\sym{***}&   0.0178\sym{**} &                  &                  &   0.0216\sym{***}&   0.0161\sym{***}&   0.0167\sym{***}\\
                &   (7.65)         &   (2.97)         &                  &                  &   (5.09)         &   (3.74)         &   (3.89)         \\
[1em]
$ \text{MFCAP*} $&                  &                  &-0.000261         & -0.00284         & -0.00356         & -0.00389\sym{*}  & -0.00391\sym{*}  \\
                &                  &                  &  (-0.30)         &  (-1.50)         &  (-1.91)         &  (-2.09)         &  (-2.33)         \\
[1em]
 $ (\text{MFCAP}^*) \times {\text{SameGroup} }  $ &                  &                  &                  &                  &                  &  0.00567         &  0.00555         \\
                &                  &                  &                  &                  &                  &   (1.92)         &   (1.69)         \\
\hline
Observations    &  1447955         &  1341445         &  1447955         &  1341445         &  1341445         &  1341445         &  1341445         \\
Group Effect    &       No         &       No         &       No         &       No         &       No         &       No         &      Yes         \\
Controls        &       No         &      Yes         &       No         &      Yes         &      Yes         &      Yes         &      Yes         \\
$ R^2 $         & 0.000573         &  0.00303         & 0.000317         &  0.00307         &  0.00337         &  0.00349         &   0.0147         \\
\hline\hline
\multicolumn{8}{l}{\footnotesize \textit{t} statistics in parentheses}\\
\multicolumn{8}{l}{\footnotesize \sym{*} \(p<0.05\), \sym{**} \(p<0.01\), \sym{***} \(p<0.001\)}\\
\end{tabular}
}

%					%				}
%				%		\end{table}}
%		%	\end{LTR}
%	%	\item 
%	%	نتایج نشان می دهد شرکت های درون گروه های کسب و کار هم بستگی بیشتری در تغییرات
%	%	\lr{turnover}
%	%	دارند و مالکیت مشترک از مسیر معاملات هم زمان تاثیری بر روی هم حرکتی از این کانال ندارد.
%	%
%	%	
%	%\end{itemize}
%	
%	
%	
%	
%	
%	%\item
%	%با توجه به بررسی ها  داریم عضویت در گروه کسب و کار سبب همبستگی تغییرات 
%	%\lr{turnover}
%	%می شود
%	\item
%	حال باید نشان دهیم 
%	\lr{turnover}
%	مرتبط در گروه های کسب و کار سبب افزایش هم حرکتی می شود
%	\item
%	\lr{turnover}
%	در سطح ماه به صورت متوسط 
%	\lr{turnover}
%	را محاسبه کردیم
%	\item
%	از 
%	\lr{turnover}
%	میانگین سالانه
%	\lr{turnover}
%	و 
%	\lr{turnover}
%	ماهانه بازار  در آن ماه را کم کردیم
%	\item
%	بررسی کردیم پراکندگی باقی مانده در گروه های کسب و کار چگونه است
%	\item
%	انتظار داریم برای شرکت های در  گروه های کسب و کار پراکندگی باقی مانده کمتر از دیگر شرکت ها باشد
%	\lr{\begin{LTR}
%			\begin{table}[htbp]
%				\centering
%				\resizebox{0.8\textwidth}{!}{
%					\begin{tabular}{lcccccccc}
\toprule
{} &  Firm $\times$ Month &   mean &    std &    min &    25\% &    50\% &    75\% &    max \\
\midrule
Ungrouped &                   8423 & -0.003 &  0.857 & -4.647 & -0.514 & -0.015 &  0.521 &  4.918 \\
Grouped   &                  18428 &  0.002 &  0.808 & -5.856 & -0.498 & -0.030 &  0.490 &  5.467 \\
\bottomrule
\end{tabular}

%				}
%				\label{tab:ResidualTrunSummary}
%			\end{table}	
%	\end{LTR}}
%	\item
%	به صورت متوسط میانگین باقی مانده صفر است
%	\item
%	پراکندگی در بیرون گروه های کسب و کار بیشتر است
%	\item
%	در سطح گروه های کسب و کار انحراف معیار را محاسبه کردیم
%	
%	\lr{\begin{LTR}
%			\begin{table}[htbp]
%				\centering
%				\resizebox{0.8\textwidth}{!}{
%					\begin{tabular}{lcccccccc}
\toprule
{} &  Group \$ \textbackslash times \$ Month &   mean &    std &    min &    25\% &    50\% &    75\% &    max \\
\midrule
Ungrouped &                      72 &  0.776 &  0.113 &  0.504 &  0.685 &  0.781 &  0.867 &  1.030 \\
Grouped   &                    2441 &  0.601 &  0.313 &  0.001 &  0.403 &  0.567 &  0.763 &  3.274 \\
\bottomrule
\end{tabular}

%				}
%				\label{tab:ResidualTrunStdSummary}
%			\end{table}
%	\end{LTR}}
%	\begin{figure}[htbp]
%		\centering
%		\includegraphics[width=0.85\linewidth]{Output/GroupedResSTD.eps}
%		\label{fig:GroupedResSTD}
%	\end{figure}
%	\item
%	برای گروه های کسب و کار متغیر دامی تعریف می کنیم
%	\item
%	که برای گروه های با انحراف معیار کم در باقی مانده ها برابر یک است
%	\item
%	بررسی می کنیم که سبب افزایش هم حرکتی می شود
%	\item
%	برای کنترل و حذف اثر اندازه گروه های کسب و کار، تعداد اعضای هر گروه را هم به صورت کن
%	\lr{\begin{LTR}
%			\begin{table}[htbp]
%				\centering
%				\resizebox{0.8\textwidth}{!}{
%					\centering
%					{
\def\sym#1{\ifmmode^{#1}\else\(^{#1}\)\fi}
\begin{tabular}{l*{6}{c}}
\hline\hline
                &\multicolumn{6}{c}{Dependent Variable:  Future Pairs's Comovement}                                               \\\cmidrule(lr){2-7}
                &\multicolumn{1}{c}{(1)}         &\multicolumn{1}{c}{(2)}         &\multicolumn{1}{c}{(3)}         &\multicolumn{1}{c}{(4)}         &\multicolumn{1}{c}{(5)}         &\multicolumn{1}{c}{(6)}         \\
\hline
SameGroup       &   0.0229\sym{***}&   0.0241\sym{***}&                  &                  &   0.0141\sym{***}&   0.0114\sym{**} \\
                &   (7.20)         &   (8.00)         &                  &                  &   (3.60)         &   (2.93)         \\
[1em]
LowTurnoverStd  &                  &  0.00233\sym{**} &   0.0296\sym{***}&-0.000636         &-0.000473         &  0.00284         \\
                &                  &   (2.65)         &   (5.72)         &  (-0.60)         &  (-0.45)         &   (1.88)         \\
[1em]
$ {\text{LowTurnoverStd} } \times {\text{SameGroup} }  $ &                  &                  &                  &                  &   0.0279\sym{***}&   0.0260\sym{***}\\
                &                  &                  &                  &                  &   (4.78)         &   (4.77)         \\
\hline
Sub-sample      &    Total         &    Total         &SameGroup         &   Others         &    Total         &    Total         \\
Business Group FE&       No         &       No         &       No         &       No         &       No         &      Yes         \\
Observations    &   389591         &   389591         &    47076         &   342515         &   389591         &   389591         \\
\hline\hline  \end{tabular}}

%				}
%				\label{Turnovercrosssection}
%			\end{table}
%	\end{LTR}}
%	%\item
%	%عضویت در گروه کسب و کار سبب هم حرکتی قیمت شرکت ها می شود
%	%\item
%	%کانال تاثیر کدام است؟
%	%\begin{itemize}
%	%\item
%	%از متغیر یکسان بودن صنعت به عنوان متغیر ابزاری برای هم بستگی تغییرات
%	%\lr{turnover}
%	%استفاده می کنیم تا هم حرکتی قیمت شرکت ها را بررسی کنیم
%	%\item
%	%شرایط استفاده از متغیر ابزاری
%	%\begin{itemize}
%	%\item
%	%ارتباط: شرکت های در یک صنعت به همراه یکدیگر معامله می شوند 
%	%نتایج برآورد 
%	%\lr{Reduced Form}
%	%این سوال را پاسخ می دهد
%	%\item
%	%برونزایی:
%	%هم حرکتی شرکت ها نمی تواند صنعت شرکت را تعیین کند
%	%\item
%	%\lr{Exclusion restriction}:
%	%با توجه به نحوه محاسبه هم حرکتی بازده صنعت از آن حذف شده است
%	%در نتیجه یکسان بودن صنعت دو شرکت نمی تواند سبب هم حرکتی قیمت آن ها شود.
%	%
%	%\item \lr{{First Stage} : 
%		%\begin{equation*}
\begin{split}
\rho(\Delta \text{TurnOver})_{ij,t+1} = & \text{ 	}\beta_0 + \beta_1* \text{SameIndustry} +\beta_3* \rho(\Delta \text{TurnOver}) \\
&+ \sum_{k=1} ^{n} \alpha_k*\text{Control}_{ij,t} + \varepsilon_{ij,t+1}
\end{split}
\end{equation*}}
%	%\item \lr{{Reduced Form} : 
%		%\input{ReducedForm.tex}}
%	%\item \lr{{Second Stage} : 
%		%\begin{equation*}
\begin{split}
\rho_{ij,t+1} = & \text{ 	}\beta_0 + \beta_1* IV(\text{SameIndustry}) +\beta_3* \rho_{ij,t} \\
&+ \sum_{k=1} ^{n} \alpha_k*\text{Control}_{ij,t} + \varepsilon_{ij,t+1}
\end{split}
\end{equation*}
%		%}
%	%\end{itemize}
%	%	\lr{
%		%	\begin{LTR}
%			%	\begin{table}[htbp]
%				%			\centering
%				%			\resizebox{0.75\textwidth}{!}{
%					%				\centering
%					%				{
\def\sym#1{\ifmmode^{#1}\else\(^{#1}\)\fi}
\begin{tabular}{l*{3}{c}}
\hline\hline
                    & First Stage         &Reduced form         &Second Stage         \\
                    &\multicolumn{1}{c}{(1)}         &\multicolumn{1}{c}{(2)}         &\multicolumn{1}{c}{(3)}         \\
\hline
SameIndustry        &      0.0285\sym{***}&     0.00133         &                     \\
                    &     (14.59)         &      (1.09)         &                     \\
[1em]
 $ {\rho(\Delta \text{TurnOver})_{t+1}} $ &                     &                     &      0.0805\sym{*}  \\
                    &                     &                     &      (2.45)         \\
[1em]
Same Group          &      0.0242\sym{***}&      0.0167\sym{***}&      0.0174\sym{***}\\
                    &     (10.73)         &      (9.72)         &     (10.40)         \\
[1em]
SameSize            &      0.0332\sym{***}&      0.0158\sym{***}&      0.0160\sym{***}\\
                    &      (3.47)         &      (5.67)         &      (9.18)         \\
[1em]
SameBookToMarket    &      0.0183\sym{***}&     0.00711\sym{***}&     0.00554\sym{***}\\
                    &      (4.38)         &      (4.46)         &      (4.48)         \\
[1em]
CrossOwnership      &      0.0393\sym{***}&      0.0172         &      0.0165\sym{*}  \\
                    &      (3.52)         &      (1.59)         &      (2.13)         \\
\hline
Observations        &     1341445         &     1665996         &     1447736         \\
Method              &          FE         &          FE         &        2sls         \\
Group FE            &         Yes         &         Yes         &         Yes         \\
Pair Size Control   &         Yes         &         Yes         &         Yes         \\
Lag of Dep. Var.    &         Yes         &         Yes         &         Yes         \\
$ R^2$              &     0.00231         &     0.00111         &                     \\
\hline\hline
\multicolumn{4}{l}{\footnotesize \textit{t} statistics in parentheses}\\
\multicolumn{4}{l}{\footnotesize \sym{*} \(p<0.05\), \sym{**} \(p<0.01\), \sym{***} \(p<0.001\)}\\
\end{tabular}
}

%					%			}
%				%			\label{TurnIv}
%				%		\end{table}
%			%		\end{LTR}
%		%		}
%	%\end{itemize}
%	%حال برای بررسی نهایی به عنوان متغییر کنترلی همبستگی تغییرات 
%	%\lr{turnover}
%	%جفت ها را به مدل اصلی اضافه می کنیم
%	%
%	%		
%	%	\begin{itemize}
%		%		\item 
%		%	برای رفع مشکل 
%		%	\lr{reverse casulity}
%		%	از لگ هم حرکتی قیمت شرکت ها استفاده می کنیم
%		%	\item
%		%	نتایج در جدول 
%		%			\ref{BigBusinessGroup}
%		%			نشان داده شده است
%		%			
%		%				\begin{itemize}
%			%				\item
%			%				در دو ستون اول هم بستگی تغییرات 
%			%				\lr{turnover}
%			%				را به مدل اضافه کرده ایم
%			%				\item
%			%				افزایش هم بستگی تغییرات
%			%				\lr{turnover}
%			%				در شرکت ها سبب افزایش هم حرکتی قیمت شرکت ها می شود
%			%				\item
%			%				در جفت های عضو یک گروه کسب و کار نیز تاثیر هم حرکتی تغییرات  
%			%				\lr{turnover}
%			%				بیشتر است
%			%				\end{itemize}
%		%					\item 
%		%								در ادامه انتظار داریم گروه های بزرگ کسب و کار به دلیل آگاهی فعالان بازار در رابطه با وجود این گروه ها، به همراه یکدیگر معامله شوند	
%		%							\begin{itemize}
%			%								
%			%								\item 
%			%								برای این منظور گروه های کسب و کار را براساس تعداد اعضا دسته بندی کردیم و  گروه های بالاتر از میانه را گروه های بزرگ در نظر گرفتیم.
%			%								\item 
%			%								انتظار داریم هم حرکتی جفت های در گروه های کسب و کار بزرگ بیشتر از جفت های دیگر باشد
%			%								\item
%			%								از طرفی تاثیر معاملات هم زمان نیز در این گروه ها باید بیشتر از گروه های کوچک باشد
%			%								\item
%			%								در گروه های کسب و کار بزرگ هم حرکتی در تغییرات 
%			%								\lr{turnover}
%			%								توضیح دهندگی بیشتری نسبت به دیگر گروه ها داشته باشد
%			%								\item
%			%								نتایج برآورد در جدول
%			%								\ref{BigBusinessGroup}
%			%								نشان داده شده است
%			%							
%			%								\item
%			%								در گروه های بزرگ هم بستگی تغییرات 
%			%								\lr{turnover}
%			%								تاثیر بیشتری بر روی هم بستگی دارد
%			%								\item
%			%								در صورتی که در دیگر گروه ها تاثیر هم بستگی در تغییرات 
%			%								\lr{turnover}
%			%								کمتر است
%			%								
%			%								
%			%							\end{itemize}
%		%	\end{itemize}
%	%	
%	%	
%	%		\begin{LTR}
%		%						\lr{		\begin{table}[htbp]
%				%								\centering
%				%								\caption{heading}
%				%								\label{BigBusinessGroup}
%				%								\resizebox{\textwidth}{!}{
%					%									{
\def\sym#1{\ifmmode^{#1}\else\(^{#1}\)\fi}
\begin{tabular}{l*{4}{c}}
\hline\hline
                &\multicolumn{4}{c}{Dep. Var.: Future Monthly Cor.  of 4F+Ind. Res.}        \\\cmidrule(lr){2-5}
                &\multicolumn{1}{c}{(1)}         &\multicolumn{1}{c}{(2)}         &\multicolumn{1}{c}{(3)}         &\multicolumn{1}{c}{(4)}         \\
\hline
Same Group      &  0.00637\sym{*}  &   0.0169\sym{*}  &  0.00476         &   0.0127         \\
                &   (2.22)         &   (2.25)         &   (1.83)         &   (1.78)         \\
[1em]
$ \text{FCA*} $ &-0.000339         &-0.000551         &-0.000108         & -0.00121         \\
                &  (-0.80)         &  (-1.14)         &  (-0.19)         &  (-1.64)         \\
[1em]
 $ (\text{FCA}^*) \times {\text{SameGroup} }  $ &   0.0120\sym{***}&   0.0120\sym{***}&   0.0121\sym{***}&   0.0115\sym{***}\\
                &   (7.57)         &   (7.74)         &   (7.14)         &   (4.07)         \\
[1em]
 $ {\rho_t(\text{Turnover})} $ &  0.00515\sym{***}&  0.00609\sym{***}&  0.00373\sym{***}&  0.00638\sym{***}\\
                &   (8.45)         &   (5.86)         &   (3.52)         &   (6.12)         \\
[1em]
 $ {\rho_t} $   &   0.0246\sym{***}&   0.0245\sym{***}&   0.0246\sym{***}&   0.0243\sym{***}\\
                &  (17.07)         &  (17.07)         &  (17.07)         &  (10.96)         \\
[1em]
$ {\text{SameGroup} \times  {\rho_t(\text{Turnover})} } $ &                  &  -0.0104         &   0.0236\sym{***}&  -0.0129         \\
                &                  &  (-0.95)         &   (5.23)         &  (-1.19)         \\
[1em]
BigGroup        &                  & -0.00148         &                  &                  \\
                &                  &  (-1.67)         &                  &                  \\
[1em]
$ {\text{BigGroup} } \times {\text{SameGroup} }  $ &                  &  -0.0132\sym{*}  &                  &                  \\
                &                  &  (-2.08)         &                  &                  \\
[1em]
$ {\text{BigGroup} } \times  {\rho_t(\text{Turnover})}  $ &                  & -0.00233         &                  &                  \\
                &                  &  (-1.35)         &                  &                  \\
[1em]
$ {\text{BigGroup}}\times{\text{SameGroup}}\times  {\rho_t(\text{Turnover})}$ &                  &   0.0336\sym{**} &                  &                  \\
                &                  &   (3.15)         &                  &                  \\
\hline
Observations    &  1459585         &  1459585         &   957316         &   502269         \\
Controls        &      Yes         &      Yes         &      Yes         &      Yes         \\
Pari Size FE    &      Yes         &      Yes         &      Yes         &      Yes         \\
SubSample       &      All         &      All         &Big Groups         &   Others         \\
$ R^2$          &  0.00241         &  0.00284         &  0.00312         &  0.00399         \\
\hline\hline
\multicolumn{5}{l}{\footnotesize \textit{t} statistics in parentheses}\\
\multicolumn{5}{l}{\footnotesize \sym{*} \(p<0.05\), \sym{**} \(p<0.01\), \sym{***} \(p<0.001\)}\\
\end{tabular}
}

%					%								}
%				%						\end{table}}
%		%					\end{LTR}
%\end{itemize}

\FloatBarrier





\subsection{\lr{Institutional Imbalance}}
\begin{LTR}
	Furthermore, we should show that stocks in groups that trade together are traded in the same direction. So, for each firm, we calculate daily institutional imbalances, which is the net buying value of institutional investors relative to total traded value on that day ($ \text{InsImb} = \frac{\text{Buy}_{\text{value}} - \text{Sell}_{\text{value}}}{\text{Buy}_{\text{value}} + \text{Sell}_{\text{value}}} $).\cite{seasholes2007predictable} 
	We expect that institutional imbalances have a lower variation in groups due to the correlated tradings that the ultimate owner ordered to do. So, at first step, we calculate monthly institutional imbalances for firms. As we expected, firms in the business groups have a lower level of standard error in  imbalances (Table \ref{tab:ImbalanceInsMeanSummary}). In the second step, 	 we calculate the monthly standard deviation of the group's imbalances and compare them to unaffiliated ones. As we expected grouped standard error is  $12.9\%$ and significantly (with pvalue of 0) lower than ungrouped firms. 
	
	\lr{\begin{table}[htbp]
		\centering
		\caption{text}
		\resizebox{0.75\textwidth}{!}{
			\begin{tabular}{lrrrrrrrr}
\toprule
{} &  Group \$\textbackslash times\$ Month &   mean &    std &  min &    25\% &    50\% &    75\% &  max \\
Grouped   &                       &        &        &      &        &        &        &      \\
\midrule
Ungrouped &                 20197 &  0.010 &  0.630 & -1.0 & -0.474 &  0.016 &  0.479 &  1.0 \\
Grouped   &                 12021 & -0.041 &  0.581 & -1.0 & -0.462 & -0.009 &  0.341 &  1.0 \\
\bottomrule
\end{tabular}

		}
		\label{tab:ImbalanceInsMeanSummary}%
	\end{table}}
	\lr{\begin{table}[htbp]
		\centering
		\caption{text}
		\resizebox{0.75\textwidth}{!}{
			\begin{tabular}{lcccccccc}
\toprule
{} &  Group \$\textbackslash times\$ Month &   mean &    std &    min &    25\% &    50\% &    75\% &    max \\
\midrule
Ungrouped &                    72 &  0.619 &  0.054 &  0.481 &  0.594 &  0.627 &  0.655 &  0.734 \\
Grouped   &                  2062 &  0.497 &  0.247 &  0.000 &  0.334 &  0.495 &  0.636 &  1.414 \\
\bottomrule
\end{tabular}

		}
		\label{tab:ImbalanceInsStdSummary}%
\end{table}}
	\begin{figure}[htbp]
		\centering
		\includegraphics[width=0.85\linewidth]{Output/GroupedInsSTD.eps}
		\label{fig:GroupedInsSTD}
	\end{figure}
	
	According to the main hypothesis, we need to compare pairs in groups with low standard error and other pairs.
	For this purpose, we define \textbf{Low Imbalance std} dummy for groups whose average standard errors are lower than half of the sample. 
	So, this dummy is equal to one if at least one pair's firms belong to the low imbalance std business group.  We use the previous methodology for estimating that model this model:
	\textsl{}\begin{equation}
\begin{split}
\rho_{ij,t+1} = & \text{ 	}\beta_0 + \beta_1* \text{SameGroup}_{ij} + \beta_2* \text{Low Imbalance std} \\
&+  \beta_3 * \text{Low Imbalance std} \times \text{SameGroup}_{ij}  \\
%& +\beta_4*  \\
%& +\beta_5* \text{FCA}^*_{ij,t} \times \text{SameGroup}_{ij}  \\
%& +\beta_6* \text{Low Imbalance std} \times \text{FCA}^*_{ij,t}  \\
% & 	+\beta_4* \text{Low Imbalance std} \times \text{SameGroup}_{ij} \times \text{FCA}^*_{ij,t}   \\
  & + \sum_{k=1} ^{n} \alpha_k*\text{Control}_{ij,t} + \varepsilon_{ij,t+1}
\end{split}
\label{model1}
\end{equation}
	
	We expected pairs in the same business groups with a low standard error of buy-sell imbalance co-move more than other pairs. Table \ref{Imbalance} reports estimation results. In columns three and four, we use our defined dummy variable and the same group. These results show that pairs in the same group of low imbalance std will comove more than other pairs. Moreover, in the subsample of same business groups, pairs in the low imbalance std comove greater than others.  For detailed analysis, we use the interaction of three variables of interest. For using this triple interaction, we use all the interactions between variables. Columns seven and eight report our results. By increasing common ownership in the same groups, pairs in the same business group of low imbalance std will comove greater than others.
	
		\lr{\begin{table}[htbp]
			\centering
			\caption{text}
			\label{Imbalance}
			\resizebox{\textwidth}{!}{
				{
\def\sym#1{\ifmmode^{#1}\else\(^{#1}\)\fi}
\begin{tabular}{l*{7}{c}}
\hline\hline
                &\multicolumn{7}{c}{Future Monthly Corr. of 4F+Ind. Residuals}                                                                       \\\cmidrule(lr){2-8}
                &\multicolumn{1}{c}{(1)}         &\multicolumn{1}{c}{(2)}         &\multicolumn{1}{c}{(3)}         &\multicolumn{1}{c}{(4)}         &\multicolumn{1}{c}{(5)}         &\multicolumn{1}{c}{(6)}         &\multicolumn{1}{c}{(7)}         \\
\hline
$ \text{FCA*} $ &  0.00296\sym{***}&  0.00277\sym{***}&  0.00275\sym{***}&                  &  0.00611\sym{**} &  0.00244\sym{**} &  0.00284\sym{**} \\
                &   (3.77)         &   (3.57)         &   (3.55)         &                  &   (3.21)         &   (3.14)         &   (3.40)         \\
[1em]
Same Group      &  0.00978\sym{***}&  0.00981\sym{***}&  0.00858\sym{**} &   0.0110\sym{***}&                  &  0.00861\sym{**} &  0.00826\sym{**} \\
                &   (4.29)         &   (4.35)         &   (3.37)         &   (4.73)         &                  &   (3.38)         &   (3.05)         \\
[1em]
Low Imbalance std&                  & -0.00364\sym{**} & -0.00388\sym{**} & -0.00446\sym{**} & -0.00725\sym{*}  & -0.00393\sym{**} & 0.000437         \\
                &                  &  (-2.81)         &  (-2.83)         &  (-3.24)         &  (-2.47)         &  (-2.87)         &   (0.21)         \\
[1em]
 $ \text{Low Imbalance std} \times {\text{SameGroup} } $ &                  &                  &  0.00301         &  0.00365         &                  & -0.00904         & -0.00990\sym{*}  \\
                &                  &                  &   (0.81)         &   (0.98)         &                  &  (-1.84)         &  (-2.02)         \\
[1em]
 $ \text{Low Imbalance std} \times {\text{SameGroup} } \times \text{FCA}^*  $ &                  &                  &                  &                  &                  &   0.0104\sym{***}&  0.00941\sym{***}\\
                &                  &                  &                  &                  &                  &   (3.87)         &   (3.53)         \\
\hline
Observations    &   388492         &   388492         &   388492         &   388492         &    37114         &   388492         &   388492         \\
Group Effect    &       No         &       No         &       No         &       No         &       No         &       No         &      Yes         \\
Sub-sample      &    Total         &    Total         &    Total         &    Total         &Same Groups         &    Total         &    Total         \\
Controls        &      Yes         &      Yes         &      Yes         &      Yes         &      Yes         &      Yes         &      Yes         \\
$ R^2 $         &  0.00229         &  0.00255         &  0.00274         &  0.00246         &   0.0199         &  0.00290         &  0.00906         \\
\hline\hline
\multicolumn{8}{l}{\footnotesize \textit{t} statistics in parentheses}\\
\multicolumn{8}{l}{\footnotesize \sym{*} \(p<0.05\), \sym{**} \(p<0.01\), \sym{***} \(p<0.001\)}\\
\end{tabular}
}

			}
	\end{table}}
\end{LTR}
%
%\begin{itemize}
%	
%	
%	
%	
%	\item 
%	در قسمت قبل نشان دادیم که شرکت های درون گروه به همراه هم حرکت می شوند و این امر سبب می شود تا هم حرکتی قیمتی نیز با یدیگر داشته باشند
%	\item
%	حال بررسی می کنیم تا شرکت ها در یک جهت نیز معامله شوند
%	\item
%	یکی از ملاک های مورد استفاده در ادبیات برای بررسی رفتار معامله گران  ناترازی خرید و فروش است
%	
%	\lr{\cite{seasholes2007predictable}}
%	
%	\begin{equation}
%		Imbalance_{ins} = \frac{Buy_{ins} - Sell_{ins}}{Buy_{ins} + Sell_{ins}}
%	\end{equation}
%	
%	\item 
%	در سطح ماه ملاک ناترازی خرید و فروش را تعریف می کنیم
%	\item 
%	که در عبارت های ذکر شده مجموع خرید و فروش در سطح یک ماه در نظر گرفته شده است
%	\item 
%	مشخصات آماری ناترازی حقوقی در جدول 
%	\ref{tab:ImbalanceInsMeanSummary}
%	بیان شده است
%	\begin{LTR}
%		\lr{\begin{table}[htbp]
%				\centering
%				\caption{text}
%				\resizebox{0.75\textwidth}{!}{
%					\begin{tabular}{lrrrrrrrr}
\toprule
{} &  Group \$\textbackslash times\$ Month &   mean &    std &  min &    25\% &    50\% &    75\% &  max \\
Grouped   &                       &        &        &      &        &        &        &      \\
\midrule
Ungrouped &                 20197 &  0.010 &  0.630 & -1.0 & -0.474 &  0.016 &  0.479 &  1.0 \\
Grouped   &                 12021 & -0.041 &  0.581 & -1.0 & -0.462 & -0.009 &  0.341 &  1.0 \\
\bottomrule
\end{tabular}

%				}
%				\label{tab:ImbalanceInsMeanSummary}
%		\end{table}}
%	\end{LTR}
%	
%\end{itemize}
%\FloatBarrier
%
%\begin{itemize}
%	\item 
%	اگر شرکت های در یک گروه کسب و کار به همراه یکدیگر معامله شوند انتظار داریم تا انحراف معیار ناترازی خرید و فروش  حقوقی در گروه کمتر از شرکت های بیرون گروه باشد
%	\item 
%	انحراف معیار ناترازی حقوقی در شرکت های درون گروه و بیرون گروه را بررسی کرده ایم
%	\item 
%	جدول 
%	\ref{tab:ImbalanceInsStdSummary}
%	و
%	نتایج را نشان می دهد
%	\begin{LTR}
%		\lr{\begin{table}[htbp]
%				\centering
%				\caption{text}
%				\resizebox{0.75\textwidth}{!}{
%					\begin{tabular}{lcccccccc}
\toprule
{} &  Group \$\textbackslash times\$ Month &   mean &    std &    min &    25\% &    50\% &    75\% &    max \\
\midrule
Ungrouped &                    72 &  0.619 &  0.054 &  0.481 &  0.594 &  0.627 &  0.655 &  0.734 \\
Grouped   &                  2062 &  0.497 &  0.247 &  0.000 &  0.334 &  0.495 &  0.636 &  1.414 \\
\bottomrule
\end{tabular}

%				}
%				\label{tab:ImbalanceInsStdSummary}%
%		\end{table}}
%	\end{LTR}
%	
%	\item 
%	به صورت متوسط انحراف معیار ناترازی در شرکت های درون گروه از شرکت های بیرون گروه کمتر است	
%	
%	\item 
%	در شکل 
%	\ref{fig:GroupedInsSTD}
%	سری زمانی میانگین انحراف معیار ناترازی شرکت ها در گروه ها و بیرون گروه نشان داده شده است
%	\begin{figure}[htbp]
%		\centering
%		\caption{text}
%		\includegraphics[width=0.85\linewidth]{Output/GroupedInsSTD.eps}
%		\label{fig:GroupedInsSTD}
%	\end{figure}
%	
%	\item 
%	به صورت متوسط انحراف معیار نا ترازی برای حقوقی ها 12\%
%	از شرکت های بیرون گروه کمتر است. (از لحاظ آماری هم این اختلاف معنا دار است)
%	
%\end{itemize}
%
%
%\begin{itemize}
%	\item 
%	همانطور که انتظار داشتیم در گروه های کسب و کار انحراف معیار نا ترازی کم است
%	\item 
%	حال باید نشان دهیم که جفت های حاضر  در گروه های کسب و کار با انحراف معیار کمتر، هم حرکتی بالاتری نیز دارند
%	\item 
%	برای این هدف متغیر دامی 
%	\textbf{Imbalance std}
%	را برای گروه هایی که انحراف معیار ناترازی حقوقی برای آن ها از میانه کمتر است تعریف می کنیم
%	%	\item 
%	%	مدل زیر را با استفاده از روش مدل 
%	%	\ref{model1}
%	%	برآورد می کنیم
%	%	\textsl{}\begin{equation}
\begin{split}
\rho_{ij,t+1} = & \text{ 	}\beta_0 + \beta_1* \text{SameGroup}_{ij} + \beta_2* \text{Low Imbalance std} \\
&+  \beta_3 * \text{Low Imbalance std} \times \text{SameGroup}_{ij}  \\
%& +\beta_4*  \\
%& +\beta_5* \text{FCA}^*_{ij,t} \times \text{SameGroup}_{ij}  \\
%& +\beta_6* \text{Low Imbalance std} \times \text{FCA}^*_{ij,t}  \\
% & 	+\beta_4* \text{Low Imbalance std} \times \text{SameGroup}_{ij} \times \text{FCA}^*_{ij,t}   \\
  & + \sum_{k=1} ^{n} \alpha_k*\text{Control}_{ij,t} + \varepsilon_{ij,t+1}
\end{split}
\label{model1}
\end{equation}
%	\item 
%	انتظار داریم جفت های حاضر در گروه های با انحراف معیار کم هم حرکتی بیشتری داشته باشند
%	
%\end{itemize}
%
%\begin{itemize}
%	\item 
%	نتایج در جدول
%	\ref{Imbalance}
%	آورده شده است
%	\begin{LTR}
%		\lr{\begin{table}[htbp]
%				\centering
%				\caption{text}
%				\label{Imbalance}
%				\resizebox{\textwidth}{!}{
%					{
\def\sym#1{\ifmmode^{#1}\else\(^{#1}\)\fi}
\begin{tabular}{l*{7}{c}}
\hline\hline
                &\multicolumn{7}{c}{Future Monthly Corr. of 4F+Ind. Residuals}                                                                       \\\cmidrule(lr){2-8}
                &\multicolumn{1}{c}{(1)}         &\multicolumn{1}{c}{(2)}         &\multicolumn{1}{c}{(3)}         &\multicolumn{1}{c}{(4)}         &\multicolumn{1}{c}{(5)}         &\multicolumn{1}{c}{(6)}         &\multicolumn{1}{c}{(7)}         \\
\hline
$ \text{FCA*} $ &  0.00296\sym{***}&  0.00277\sym{***}&  0.00275\sym{***}&                  &  0.00611\sym{**} &  0.00244\sym{**} &  0.00284\sym{**} \\
                &   (3.77)         &   (3.57)         &   (3.55)         &                  &   (3.21)         &   (3.14)         &   (3.40)         \\
[1em]
Same Group      &  0.00978\sym{***}&  0.00981\sym{***}&  0.00858\sym{**} &   0.0110\sym{***}&                  &  0.00861\sym{**} &  0.00826\sym{**} \\
                &   (4.29)         &   (4.35)         &   (3.37)         &   (4.73)         &                  &   (3.38)         &   (3.05)         \\
[1em]
Low Imbalance std&                  & -0.00364\sym{**} & -0.00388\sym{**} & -0.00446\sym{**} & -0.00725\sym{*}  & -0.00393\sym{**} & 0.000437         \\
                &                  &  (-2.81)         &  (-2.83)         &  (-3.24)         &  (-2.47)         &  (-2.87)         &   (0.21)         \\
[1em]
 $ \text{Low Imbalance std} \times {\text{SameGroup} } $ &                  &                  &  0.00301         &  0.00365         &                  & -0.00904         & -0.00990\sym{*}  \\
                &                  &                  &   (0.81)         &   (0.98)         &                  &  (-1.84)         &  (-2.02)         \\
[1em]
 $ \text{Low Imbalance std} \times {\text{SameGroup} } \times \text{FCA}^*  $ &                  &                  &                  &                  &                  &   0.0104\sym{***}&  0.00941\sym{***}\\
                &                  &                  &                  &                  &                  &   (3.87)         &   (3.53)         \\
\hline
Observations    &   388492         &   388492         &   388492         &   388492         &    37114         &   388492         &   388492         \\
Group Effect    &       No         &       No         &       No         &       No         &       No         &       No         &      Yes         \\
Sub-sample      &    Total         &    Total         &    Total         &    Total         &Same Groups         &    Total         &    Total         \\
Controls        &      Yes         &      Yes         &      Yes         &      Yes         &      Yes         &      Yes         &      Yes         \\
$ R^2 $         &  0.00229         &  0.00255         &  0.00274         &  0.00246         &   0.0199         &  0.00290         &  0.00906         \\
\hline\hline
\multicolumn{8}{l}{\footnotesize \textit{t} statistics in parentheses}\\
\multicolumn{8}{l}{\footnotesize \sym{*} \(p<0.05\), \sym{**} \(p<0.01\), \sym{***} \(p<0.001\)}\\
\end{tabular}
}

%				}
%		\end{table}}
%	\end{LTR}
%	\begin{itemize}
%		\item 
%		همچنان جفت های در یک گروه کسب و کار هم حرکتی بیشتری دارند
%		\item 
%		اگر جفت های در یک گروه کسب و کار، در گروه های با انحراف معیار کم باشند $ 2.4\% $ هم حرکتی آن ها افزایش پیدا می کند
%		(میانگین هم حرکتی تقریبا $ 1.6 \%$ است )
%		
%		
%	\end{itemize}
%\end{itemize}

\FloatBarrier

